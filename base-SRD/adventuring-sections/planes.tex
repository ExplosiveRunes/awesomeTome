%%%%%%%%%%%%%%%%%%%%%%%%%%%%%%%%%%%%%%%%%%%%%%%%%%
\section{The Planes}\index{Planes}
%%%%%%%%%%%%%%%%%%%%%%%%%%%%%%%%%%%%%%%%%%%%%%%%%%

%%%%%%%%%%%%%%%%%%%%%%%%%
\subsection{What Is A Plane?}
%%%%%%%%%%%%%%%%%%%%%%%%%

The planes of existence are different realities with interwoven connections. Except 
for rare linking points, each plane is effectively its own universe with its own 
natural laws. 

The planes break down into a number of general types: the Material Plane, the Transitive 
Planes, the Inner Planes, the Outer Planes, and the demiplanes.

\textbf{\gameterm{Material Plane}:} The Material Plane tends to be the most Earthlike of all 
planes and operates under the same set of natural laws that our own real world 
does. This is the default plane for most adventures.

\textbf{\gameterm{Transitive Planes}:} These three planes have one important common characteristic: 
Each is used to get from one place to another. The Astral Plane is a conduit to 
all other planes, while the Ethereal Plane and the Plane of Shadow both serve as 
means of transportation within the Material Plane they're connected to. These planes 
have the strongest regular interaction with the Material Plane and are often accessed 
by using various spells. They have native inhabitants as well.

\textbf{\gameterm{Inner Planes}:} These six planes are manifestations of the basic building 
blocks of the universe. Each is made up of a single type of energy or element that 
overwhelms all others. The natives of a particular Inner Plane are made of the 
same energy or element as the plane itself.

\textbf{\gameterm{Outer Planes}:} The deities live on the Outer Planes, as do creatures such 
as celestials, demons, and devils. Each of the Outer Planes has an alignment, representing 
a particular moral or ethical outlook, and the natives of each plane tend to behave 
in agreement with that plane's alignment. The Outer Planes are also the final resting 
place of souls from the Material Plane, whether that final rest takes the form 
of calm introspection or eternal damnation.

\textbf{\gameterm{Demiplanes}:} This catch-all category covers all extradimensional spaces 
that function like planes but have measurable size and limited access. Other kinds 
of planes are theoretically infinite in size, but a demiplane might be only a few 
hundred feet across.

%%%%%%%%%%%%%%%%%%%%%%%%%
\subsection{Planar Traits}
%%%%%%%%%%%%%%%%%%%%%%%%%

Each plane of existence has its own properties -- the natural laws of its universe.

Planar traits are broken down into a number of general areas.

All planes have the following kinds of traits.

\textbf{Physical Traits:} These traits determine the laws of physics and nature 
on the plane, including how gravity and time function.

\textbf{Elemental and Energy Traits:} These traits determine the dominance of particular 
elemental or energy forces.

\textbf{Alignment Traits:} Just as characters may be lawful neutral or chaotic 
good, many planes are tied to a particular moral or ethical outlook.

\textbf{Magic Traits:} Magic works differently from plane to plane, and magic traits 
set the boundaries for what it can and can't do.

%%%
\subsubsection{Physical Traits}
%%%

The two most important natural laws set by physical traits are how gravity works 
and how time passes. Other physical traits pertain to the size and shape of a plane 
and how easily a plane's nature can be altered.

%%%
\subsubsection{Gravity}
%%%

The direction of gravity's pull may be unusual, and it might 
even change directions within the plane itself.

\textit{Normal Gravity:} Most planes have gravity similar to that of the Material 
Plane. The usual rules for ability scores, carrying capacity, and encumbrance apply. 
Unless otherwise noted in a description, it is assumed every plane has the normal 
gravity trait.

\textit{Heavy Gravity:} The gravity on a plane with this trait is much more intense 
than on the Material Plane. As a result, Balance, Climb, Jump, Ride, Swim, and 
Tumble checks incur a -2 circumstance penalty, as do all attack rolls. All item 
weights are effectively doubled, which might affect a character's speed. Weapon 
ranges are halved. A character's Strength and Dexterity scores are not affected. 
Characters who fall on a heavy gravity plane take 1d10 points of damage for each 
10 feet fallen, to a maximum of 20d10 points of damage.

\textit{Light Gravity:} The gravity on a plane with this trait is less intense 
than on the Material Plane. As a result, creatures find that they can lift more, 
but their movements tend to be ungainly. Characters on a plane with the light gravity 
trait take a -2 circumstance penalty on attack rolls and Balance, Ride, Swim, and 
Tumble checks. All items weigh half as much. Weapon ranges double, and characters 
gain a +2 circumstance bonus on Climb and Jump checks.

Strength and Dexterity don't change as a result of light gravity, but what you 
can do with such scores does change. These advantages apply to travelers from other 
planes as well as natives.

Falling characters on a light gravity plane take 1d4 points of damage for each 
10 feet of the fall (maximum 20d4).

\textit{No Gravity:} Individuals on a plane with this trait merely float in space, 
unless other resources are available to provide a direction for gravity's pull.

\textit{Objective Directional Gravity:} The strength of gravity on a plane with 
this trait is the same as on the Material Plane, but the direction is not the traditional 
"down" toward the ground. It may be down toward any solid object, at an angle 
to the surface of the plane itself, or even upward.

In addition, objective directional gravity may change from place to place. The 
direction of "down" may vary.

\textit{Subjective Directional Gravity:} The strength of gravity on a plane with 
this trait is the same as on the Material Plane, but each individual chooses the 
direction of gravity's pull. Such a plane has no gravity for unattended objects 
and non-sentient creatures. This sort of environment can be very disorienting to 
the newcomer, but is common on "weightless" planes.

Characters on a plane with subjective directional gravity can move normally along 
a solid surface by imagining "down" near their feet. If suspended in midair, 
a character "flies" by merely choosing a "down" direction and "falling" that 
way. Under such a procedure, an individual "falls" 150 feet in the first round 
and 300 feet in each succeeding round. Movement is straight-line only. In order 
to stop, one has to slow one's movement by changing the designated "down" direction 
(again, moving 150 feet in the new direction in the first round and 300 feet per 
round thereafter).

It takes a DC 16 Wisdom check to set a new direction of gravity as a free action; 
this check can be made once per round. Any character who fails this Wisdom check 
in successive rounds receives a +6 bonus on subsequent checks until he or she succeeds.

%%%
\subsubsection{Time}
%%%

The rate of time's passage can vary on different planes, though 
it remains constant within any particular plane. Time is always subjective for 
the viewer. The same subjectivity applies to various planes. Travelers may discover 
that they'll pick up or lose time while moving among the planes, but from their 
point of view, time always passes naturally.

\textit{Normal Time:} This trait describes the way time passes on the Material 
Plane. One hour on a plane with normal time equals one hour on the Material Plane. 
Unless otherwise noted in a description, every plane has the normal time trait.

\textit{Timeless:} On planes with this trait, time still passes, but the effects 
of time are diminished. How the timeless trait can affect certain activities or 
conditions such as hunger, thirst, aging, the effects of poison, and healing varies 
from plane to plane.

The danger of a timeless plane is that once one leaves such a plane for one where 
time flows normally, conditions such as hunger and aging do occur retroactively. 

\textit{Flowing Time:} On some planes, time can flow faster or slower. One may 
travel to another plane, spend a year there, then return to the Material Plane 
to find that only six seconds have elapsed. Everything on the plane returned to 
is only a few seconds older. But for that traveler and the items, spells, and effects 
working on him, that year away was entirely real.

When designating how time works on planes with flowing time, put the Material Plane's 
flow of time first, followed by the same flow in the other plane. 

\textit{Erratic Time:} Some planes have time that slows down and speeds up, so 
an individual may lose or gain time as he moves between the two planes. The following 
is provided as an example.

\begin{table}[htb]
\rowcolors{1}{white}{offyellow}
\caption{Erratic Time Passage}
\centering
\begin{tabular}{ccc}
\textbf{d\%} & \textbf{Time on Material Plane} & \textbf{Time on Erratic Time Plane}\\
01-10 & 1 day & 1 round\\
11-40 & 1 day & 1 hour\\
41-60 & 1 day & 1 day\\
61-90 & 1 hour & 1 day\\
91-100 & 1 round & 1 day\\
\end{tabular}
\end{table}

To the denizens of such a plane, time flows naturally and the shift is unnoticed.

If a plane is timeless with respect to magic, any spell cast with a non-instantaneous 
duration is permanent until dispelled.

%%%
\subsubsection{Shape and Size}
%%%

Planes come in a variety of sizes and shapes. Most planes 
are infinite, or at least so large that they may as well be infinite.

\textit{Infinite:} Planes with this trait go on forever, though they may have finite 
components within them. Or they may consist of ongoing expanses in two directions, 
like a map that stretches out infinitely.

\textit{Finite Shape:} A plane with this trait has defined edges or borders. These 
borders may adjoin other planes or hard, finite borders such as the edge of the 
world or a great wall. Demiplanes are often finite.

\textit{Self-Contained Shape:} On planes with this trait, the borders wrap in on 
themselves, depositing the traveler on the other side of the map. A spherical plane 
is an example of a self-contained, finite plane, but there can be cubes, toruses, 
and flat planes with magical edges that teleport the traveler to an opposite edge 
when he crosses them. 

Some demiplanes are self-contained.

%%%
\subsubsection{Morphic Traits}
%%%

This trait measures how easily the basic nature of a plane 
can be changed. Some planes are responsive to sentient thought, while others can 
be manipulated only by extremely powerful creatures. And some planes respond to 
physical or magical efforts.

\textit{Alterable Morphic:} On a plane with this trait, objects remain where they 
are (and what they are) unless affected by physical force or magic. You can change 
the immediate environment as a result of tangible effort. 

\textit{Highly Morphic:} On a plane with this trait, features of the plane change 
so frequently that it's difficult to keep a particular area stable. Such planes 
may react dramatically to specific spells, sentient thought, or the force of will. 
Others change for no reason. 

\textit{Magically Morphic:} Specific spells can alter the basic material of a plane 
with this trait.

\textit{Divinely Morphic:} Specific unique beings (deities or similar great powers) 
have the ability to alter objects, creatures, and the landscape on planes with 
this trait. Ordinary characters find these planes similar to alterable planes in 
that they may be affected by spells and physical effort. But the deities may cause 
these areas to change instantly and dramatically, creating great kingdoms for themselves. 

\textit{Static:} These planes are unchanging. Visitors cannot affect living residents 
of the plane, nor objects that the denizens possess. Any spells that would affect 
those on the plane have no effect unless the plane's static trait is somehow removed 
or suppressed. Spells cast before entering a plane with the static trait remain 
in effect, however.

Even moving an unattended object within a static plane requires a DC 16 Strength 
check. Particularly heavy objects may be impossible to move.

\textit{Sentient:} These planes are ones that respond to a single thought -- that 
of the plane itself. Travelers would find the plane's landscape changing as a result 
of what the plane thought of the travelers, either becoming more or less hospitable 
depending on its reaction.

%%%
\subsubsection{Elemental and Energy Traits}
%%%

Four basic elements and two types of energy together make up everything. The elements 
are earth, air, fire, and water. The types of energy are positive and negative.

The Material Plane reflects a balancing of those elements and energies; all are 
found there. Each of the Inner Planes is dominated by one element or type of energy. 
Other planes may show off various aspects of these elemental traits. Many planes 
have no elemental or energy traits; these traits are noted in a plane's description 
only when they are present.

\textit{Air-Dominant:} Mostly open space, planes with this trait have just a few 
bits of floating stone or other elements. They usually have a breathable atmosphere, 
though such a plane may include clouds of acidic or toxic gas. Creatures of the 
earth subtype are uncomfortable on air-dominant planes because they have little 
or no natural earth to connect with. They take no actual damage, however.

\textit{Earth-Dominant:} Planes with this trait are mostly solid. Travelers who 
arrive run the risk of suffocation if they don't reach a cavern or other pocket 
within the earth. Worse yet, individuals without the ability to burrow are entombed 
in the earth and must dig their way out (5 feet per turn). Creatures of the air 
subtype are uncomfortable on earth dominant planes because these planes are tight 
and claustrophobic to them. But they suffer no inconvenience beyond having difficulty 
moving.

\textit{Fire-Dominant:} Planes with this trait are composed of flames that continually 
burn without consuming their fuel source. Fire-dominant planes are extremely hostile 
to Material Plane creatures, and those without resistance or immunity to fire are 
soon immolated.

Unprotected wood, paper, cloth, and other flammable materials catch fire almost 
immediately, and those wearing unprotected flammable clothing catch on fire. In 
addition, individuals take 3d10 points of fire damage every round they are on a 
fire-dominant plane. Creatures of the water subtype are extremely uncomfortable 
on fire-dominant planes. Those that are made of water take double damage each round.

\textit{Water-Dominant:} Planes with this trait are mostly liquid. Visitors who 
can't breathe water or reach a pocket of air will likely drown. Creatures of the 
fire subtype are extremely uncomfortable on water-dominant planes. Those made of 
fire take 1d10 points of damage each round.

\textit{Positive-Dominant:} An abundance of life characterizes planes with this 
trait. The two kinds of positive-dominant traits are minor positive-dominant and 
major positive-dominant. A minor positive-dominant plane is a riotous explosion 
of life in all its forms. Colors are brighter, fires are hotter, noises are louder, 
and sensations are more intense as a result of the positive energy swirling through 
the plane. All individuals in a positive-dominant plane gain fast healing 2 as 
an extraordinary ability.

Major positive-dominant planes go even further. A creature on a major positive-dominant 
plane must make a DC 15 Fortitude save to avoid being blinded for 10 rounds by 
the brilliance of the surroundings. Simply being on the plane grants fast healing 
5 as an extraordinary ability. In addition, those at full hit points gain 5 additional 
temporary hit points per round. These temporary hit points fade 1d20 rounds after 
the creature leaves the major positive- dominant plane. However, a creature must 
make a DC 20 Fortitude save each round that its temporary hit points exceed its 
normal hit point total. Failing the saving throw results in the creature exploding 
in a riot of energy, killing it.

\textit{Negative-Dominant:} Planes with this trait are vast, empty reaches that 
suck the life out of travelers who cross them. They tend to be lonely, haunted 
planes, drained of color and filled with winds bearing the soft moans of those 
who died within them. As with positive-dominant planes, negative-dominant planes 
can be either minor or major. On minor negative-dominant planes, living creatures 
take 1d6 points of damage per round. At 0 hit points or lower, they crumble into 
ash.

Major negative-dominant planes are even more severe. Each round, those within must 
make a DC 25 Fortitude save or gain a negative level. A creature whose negative 
levels equal its current levels or Hit Dice is slain, becoming a wraith. The \linkspell{Death Ward}
spell protects a traveler from the damage and energy drain of a negative-dominant 
plane.

%%%
\subsubsection{Alignment Traits}
%%%

Some planes have a predisposition to a certain alignment. Most of the inhabitants 
of these planes also have the plane's particular alignment, even powerful creatures 
such as deities. In addition, creatures of alignments contrary to the plane have 
a tougher time dealing with its natives and situations.

The alignment trait of a plane affects social interactions there. Characters who 
follow other alignments than most of the inhabitants do may find life more difficult.

Alignment traits have multiple components. First are the moral (good or evil) and 
ethical (lawful or chaotic) components; a plane can have either a moral component, 
an ethical component, or one of each. Second, the specific alignment trait indicates 
whether each moral or ethical component is mildly or strongly evident.

\textit{Good-Aligned/Evil-Aligned:} These planes have chosen a side in the battle 
of good versus evil. No plane can be both good-aligned and evil-aligned.

\textit{Law-Aligned/Chaos-Aligned:} Law versus chaos is the key struggle for these 
planes and their residents. No plane can be both law-aligned and chaos-aligned.

\vspace{12pt}
Each part of the moral/ethical alignment trait has a descriptor, either "mildly" 
or "strongly," to show how powerful the influence of alignment is on the plane.

\textit{Mildly Aligned:} Creatures who have an alignment opposite that of a mildly 
aligned plane take a -2 circumstance penalty on all Charisma-based checks.

\textit{Strongly Aligned:} On planes that are strongly aligned, a -2 circumstance 
penalty applies on all Charisma-based checks made by all creatures not of the plane's 
alignment. In addition, the -2 penalty affects all Intelligence-based and Wisdom-based 
checks, too.

The penalties for the moral and ethical components of the alignment trait do stack.

\textit{Neutral-Aligned:} A mildly neutral-aligned plane does not apply a circumstance 
penalty to anyone.

The Material Plane is considered mildly neutral-aligned, though it may contain 
high concentrations of evil or good, law or chaos in places.

A strongly neutral-aligned plane would stand in opposition to all other moral and 
ethical principles: good, evil, law, and chaos. Such a plane may be more concerned 
with the balance of the alignments than with accommodating and accepting alternate 
points of view. In the same fashion as for other strongly aligned planes, strongly 
neutral-aligned planes apply a -2 circumstance penalty to Intelligence-, Wisdom-, 
or Charisma-based checks by any creature that isn't neutral. The penalty is applied 
twice (once for law/chaos, and once for good/evil), so neutral good, neutral evil, 
lawful neutral, and chaotic neutral creatures take a -2 penalty and lawful good, 
chaotic good, chaotic evil, and lawful evil creatures take a -4 penalty.

%%%
\subsubsection{Magic Traits}
%%%

A plane's magic trait describes how magic works on the plane compared to how it 
works on the Material Plane. Particular locations on a plane (such as those under 
the direct control of deities) may be pockets where a different magic trait applies.

\textit{Normal Magic:} This magic trait means that all spells and supernatural 
abilities function as written. Unless otherwise noted in a description, every plane 
has the normal magic trait.

\textit{Wild Magic:} On a plane with the wild magic trait spells and spell-like 
abilities function in radically different and sometimes dangerous ways. Any spell 
or spell-like ability used on a wild magic plane has a chance to go awry. The caster 
must make a level check (DC 15 + the level of the spell or effect) for the magic 
to function normally. For spell-like abilities, use the level or HD of the creature 
employing the ability for the caster level check and the level of the spell-like 
ability to set the DC for the caster level check. Failure on this check means that 
something strange happens; roll d\% and consult the following table.

\begin{table}[htb]
\rowcolors{1}{white}{offyellow}
\caption{Wild Magic Effects}
\centering
\begin{tabular}{c p{14cm}}
\textbf{d\%} & \textbf{Effect}\\
01-19 & Spell rebounds on caster with normal effect. If the spell cannot affect the caster, it simply fails.\\
20-23 & A circular pit 15 feet wide opens under the caster's feet; it is 10 feet deep per level of the caster.\\
24-27 & The spell fails, but the target or targets of the spell are pelted with a rain of small objects (anything from flowers to rotten fruit), which disappear upon striking. The barrage continues for 1 round. During this time the targets are blinded and must make Concentration checks (DC 15 + spell level) to cast spells.\\
28-31 & The spell affects a random target or area. Randomly choose a different target from among those in range of the spell or center the spell at a random place within range of the spell. To generate direction randomly, roll 1d8 and count clockwise around the compass, starting with south. To generate range randomly, roll 3d6. Multiply the result by 5 feet for close range spells, 20 feet for medium range spells, or 80 feet for long range spells.\\
32-35 & The spell functions normally, but any material components are not consumed. The spell is not expended from the caster's mind (a spell slot or prepared spell can be used again). An item does not lose charges, and the effect does not count against an item's or spell-like ability's use limit.\\
36-39 & The spell does not function. Instead, everyone (friend or foe) within 30 feet of the caster receives the effect of a heal spell.\\
40-43 & The spell does not function. Instead, a deeper darkness and a silence effect cover a 30-foot radius around the caster for 2d4 rounds.\\
44-47 & The spell does not function. Instead, a reverse gravity effect covers a 30-foot radius around the caster for 1 round.\\
48-51 & The spell functions, but shimmering colors swirl around the caster for 1d4 rounds. Treat this a glitterdust effect with a save DC of 10 + the level of the spell that generated this result.\\
52-59 & Nothing happens. The spell does not function. Any material components are used up. The spell or spell slot is used up, and charges or uses from an item are used up.\\
60-71 & Nothing happens. The spell does not function. Any material components are not consumed. The spell is not expended from the caster's mind (a spell slot or prepared spell can be used again). An item does not lose charges, and the effect does not count against an item's or spell-like ability's use limit.\\
72-98 & The spell functions normally.\\
99-100 & The spell functions strongly. Saving throws against the spell incur a -2 penalty. The spell has the maximum possible effect, as if it were cast with the Maximize Spell feat. If the spell is already maximized with the feat, there is no further effect.\\
\end{tabular}
\end{table}

\textit{Impeded Magic:} Particular spells and spell-like abilities are more difficult 
to cast on planes with this trait, often because the nature of the plane interferes 
with the spell.

To cast an impeded spell, the caster must make a Spellcraft check (DC 20 + the 
level of the spell). If the check fails, the spell does not function but is still 
lost as a prepared spell or spell slot. If the check succeeds, the spell functions 
normally.

\textit{Enhanced Magic:} Particular spells and spell-like abilities are easier 
to use or more powerful in effect on planes with this trait than they are on the 
Material Plane.

Natives of a plane with the enhanced magic trait are aware of which spells and 
spell-like abilities are enhanced, but planar travelers may have to discover this 
on their own.

If a spell is enhanced, certain metamagic feats can be applied to it without changing 
the spell slot required or the casting time. Spellcasters on the plane are considered 
to have that feat or feats for the purpose of applying them to that spell. Spellcasters 
native to the plane must gain the feat or feats normally if they want to use them 
on other planes as well.

\textit{Limited Magic:} Planes with this trait permit only the use of spells and 
spell-like abilities that meet particular qualifications.

Magic can be limited to effects from certain schools or subschools, to effects 
with certain descriptors, or to effects of a certain level (or any combination 
of these qualities). Spells and spell-like abilities that don't meet the qualifications 
simply don't work.

\textit{Dead Magic:} These planes have no magic at all. A plane with the dead magic 
trait functions in all respects like an \linkspell{Antimagic Field} spell. Divination 
spells cannot detect subjects within a dead magic plane, nor can a spellcaster 
use \linkspell{Plane Shift} or another spell to move in or out. The only exception to 
the "no magic" rule is permanent planar portals, which still function normally.

%%%%%%%%%%%%%%%%%%%%%%%%%
\subsection{How Planes Interact}
%%%%%%%%%%%%%%%%%%%%%%%%%

\textbf{Separate Planes:} Two planes that are separate do not overlap or directly 
connect to each other. They are like planets in different orbits. The only way 
to get from one separate plane to the other is to go through a third plane.

\textbf{\gameterm{Coterminous Planes}:} Planes that touch at specific points are coterminous. 
Where they touch, a connection exists, and travelers can leave one reality behind 
and enter the other.

\textbf{\gameterm{Coexistent Planes}:} If a link between two planes can be created at any 
point, the two planes are coexistent. These planes overlap each other completely. 
A coexistent plane can be reached from anywhere on the plane it overlaps. When 
moving on a coexistent plane, it is often possible to see into or interact with 
the plane it coexists with. 

%%%
\subsubsection{Layered Planes}
%%%

Infinities may be broken into smaller infinities, and planes into smaller, related 
planes. These layers are effectively separate planes of existence, and each layer 
can have its own planar traits. Layers are connected to each other through a variety 
of planar gates, natural vortices, paths, and shifting borders.

Access to a layered plane from elsewhere usually happens on a specific layer: the 
first layer of the plane, which can be either the top layer or the bottom layer, 
depending on the specific plane. Most fixed access points (such as portals and 
natural vortices) reach this layer, which makes it the gateway for other layers 
of the plane. The \linkspell{Plane Shift} spell also deposits the spellcaster on the 
first layer of the plane.

%%%%%%%%%%%%%%%%%%%%%%%%%
\subsection{Plane Descriptions}
%%%%%%%%%%%%%%%%%%%%%%%%%

%%%
\subsubsection{The Material Plane}\index{Material Plane}
%%%

The Material Plane is the center of most cosmologies and defines what is considered 
normal.

The Material Plane has the following traits:

\begin{itemize}
\item Normal gravity.
\item Normal Time
\item Alterable morphic.
\item No Elemental or Energy Traits (specific locations may have these traits, however)
\item Mildly neutral-aligned.
\item Normal magic. 
\end{itemize}

%%%
\subsubsection{The Ethereal Plane}\index{Ethereal Plane}
%%%

The Ethereal Plane is coexistent with the Material Plane and often other planes 
as well. The Material Plane itself is visible from the Ethereal Plane, but it appears 
muted and indistinct, its colors blurring into each other and its edges turning 
fuzzy.

While it is possible to see into the Material Plane from the Ethereal Plane, the 
Ethereal Plane is usually invisible to those on the Material Plane. Normally, creatures 
on the Ethereal Plane cannot attack creatures on the Material Plane, and vice versa. 
A traveler on the Ethereal Plane is invisible, incorporeal, and utterly silent 
to someone on the Material Plane. 

The Ethereal Plane is mostly empty of structures and impediments. However, the 
plane has its own inhabitants. Some of these are other ethereal travelers, but 
the ghosts found here pose a particular peril to those who walk the fog. 

It has the following traits.

\begin{itemize}
\item No gravity.
\item Alterable morphic. The plane contains little to alter, however.
\item Mildly neutral-aligned.
\item Normal magic. Spells function normally on the Ethereal Plane, though they do not 
cross into the Material Plane. 
\end{itemize}

The only exceptions are spells and spell-like abilities that have the force descriptor 
and abjuration spells that affect ethereal beings. Spellcasters on the Material 
Plane must have some way to detect foes on the Ethereal Plane before targeting 
them with force-based spells, of course. While it's possible to hit ethereal enemies 
with a force spell cast on the Material Plane, the reverse isn't possible. No magical 
attacks cross from the Ethereal Plane to the Material Plane, including force attacks.

%%%
\subsubsection{The Plane of Shadow}\index{Plane of Shadow}
%%%

The Plane of Shadow is a dimly lit dimension that is both coterminous to 
and coexistent with the Material Plane. It overlaps the Material Plane much as 
the Ethereal Plane does, so a planar traveler can use the Plane of Shadow to cover 
great distances quickly.

The Plane of Shadow is also coterminous to other planes. With the right spell, 
a character can use the Plane of Shadow to visit other realities.

The Plane of Shadow is a world of black and white; color itself has been bleached 
from the environment. It is otherwise appears similar to the Material Plane.

Despite the lack of light sources, various plants, animals, and humanoids call 
the Plane of Shadow home.

The Plane of Shadow is magically morphic, and parts continually flow onto other 
planes. As a result, creating a precise map of the plane is next to impossible, 
despite the presence of landmarks.

The Plane of Shadow has the following traits.

\begin{itemize}
\item Magically morphic. Certain spells modify the base material of the Plane of Shadow. 
The utility and power of these spells within the Plane of Shadow make them particularly 
useful for explorers and natives alike.
\item Mildly neutral-aligned.
\item Enhanced magic. Spells with the shadow descriptor are enhanced on the Plane of 
Shadow. Such spells are cast as though they were prepared with the Maximize Spell 
feat, though they don't require the higher spell slots. Furthermore, specific spells become more powerful on the Plane of Shadow. \linkspell{Shadow Conjuration} and \linkspell{Shadow Evocation} spells are 30\% as powerful as the conjurations and evocations they mimic (as opposed to 20\%). \linkspell{Greater Shadow Conjuration} and \linkspell{Greater Shadow Evocation} are 70\% as powerful (not 60\%), and a \linkspell{Shades} spell conjures at 90\% of the power of the original (not 80\%).
\item Impeded magic. Spells that use or generate light or fire may fizzle when cast on 
the Plane of Shadow. A spellcaster attempting a spell with the light or fire descriptor 
must succeed on a Spellcraft check (DC 20 + the level of the spell). Spells that 
produce light are less effective in general, because all light sources have their 
ranges halved on the Plane of Shadow.
\end{itemize}

Despite the dark nature of the Plane of Shadow, spells that produce, use, or manipulate 
darkness are unaffected by the plane.

%%%
\subsubsection{The Astral Plane}\index{Astral Plane}
%%%

The Astral Plane is the space between the planes. When a character moves through 
an interplanar portal or projects her spirit to a different plane of existence, 
she travels through the Astral Plane. Even spells that allow instantaneous movement 
across a plane briefly touch the Astral Plane.

The Astral Plane is a great, endless sphere of clear silvery sky, both above and 
below. Occasional bits of solid matter can be found here, but most of the Astral 
Plane is an endless, open domain.

Both planar travelers and refugees from other planes call the Astral Plane home. 

The Astral Plane has the following traits.

\begin{itemize}
\item Subjective directional gravity.
\item Timeless. Age, hunger, thirst, poison, and natural healing don't function in the 
Astral Plane, though they resume functioning when the traveler leaves the Astral 
Plane.
\item Mildly neutral-aligned.
\item Enhanced magic. All spells and spell-like abilities used within the Astral Plane 
may be employed as if they were improved by the \linkfeat{Quicken Spell} feat. Already quickened 
spells and spell-like abilities are unaffected, as are spells from magic items. 
Spells so quickened are still prepared and cast at their unmodified level. As with 
the Quicken Spell feat, only one quickened spell can be cast per round.
\end{itemize}

%%%
\subsubsection{Elemental Plane of Air}\index{Elemental Plane of Air}
%%%

The Elemental Plane of Air is an empty plane, consisting of sky above and sky below.

The Elemental Plane of Air is the most comfortable and survivable of the Inner 
Planes, and it is the home of all manner of airborne creatures. Indeed, flying 
creatures find themselves at a great advantage on this plane. While travelers without 
flight can survive easily here, they are at a disadvantage.

The Elemental Plane of Air has the following traits.

\begin{itemize}
\item Subjective directional gravity. Inhabitants of the plane determine their own "down" 
direction. Objects not under the motive force of others do not move.
\item Air-dominant.
\item Enhanced magic. Spells and spell-like abilities that use, manipulate, or create 
air (including spells of the Air domain) are both empowered and enlarged (as if 
the Empower Spell and Enlarge Spell metamagic feats had been used on them, but 
the spells don't require higher-level slots).
\item Impeded magic. Spells and spell-like abilities that use or create earth (including 
spells of the Earth domain and spells that summon earth elementals or outsiders 
with the earth subtype) are impeded.
\end{itemize}

%%%
\subsubsection{Elemental Plane of Earth}\index{Elemental Plane of Earth}
%%%

The Elemental Plane of Earth is a solid place made of rock, soil, and stone. An 
unwary and unprepared traveler may find himself entombed within this vast solidity 
of material and have his life crushed into nothingness, his powdered remains a 
warning to any foolish enough to follow.

Despite its solid, unyielding nature, the Elemental Plane of Earth is varied in 
its consistency, ranging from relatively soft soil to veins of heavier and more 
valuable metal. 

The Elemental Plane of Earth has the following traits.

\begin{itemize}
\item Earth-dominant.
\item Enhanced magic. Spells and spell-like abilities that use, manipulate, or create 
earth or stone (including those of the Earth domain) are both empowered and extended 
(as if the Empower Spell and Extend Spell metamagic feats had been used on them, 
but the spells don't require higher-level slots). Spells and spell-like abilities 
that are already empowered or extended are unaffected by this benefit.
\item Impeded magic. Spells and spell-like abilities that use or create air (including 
spells of the Air domain and spells that summon air elementals or outsiders with 
the air subtype) are impeded.
\end{itemize}

%%%
\subsubsection{Elemental Plane of Fire}\index{Elemental Plane of Fire}
%%%

Everything is alight on the Elemental Plane of Fire. The ground is nothing more 
than great, evershifting plates of compressed flame. The air ripples with the heat 
of continual firestorms, and the most common liquid is magma, not water. The oceans 
are made of liquid flame, and the mountains ooze with molten lava. Fire survives 
here without need for fuel or air, but flammables brought onto the plane are consumed 
readily. 

The Elemental Plane of Fire has the following traits.

\begin{itemize}
\item Fire-dominant.
\item Enhanced magic. Spells and spell-like abilities with the fire descriptor are both 
maximized and enlarged (as if the Maximize Spell and Enlarge Spell had been used 
on them, but the spells don't require higher-level slots). Spells and spell-like 
abilities that are already maximized or enlarged are unaffected by this benefit.
\item Impeded magic. Spells and spell-like abilities that use or create water (including 
spells of the Water domain and spells that summon water elementals or outsiders 
with the water subtype) are impeded. 
\end{itemize}

%%%
\subsubsection{Elemental Plane of Water}\index{Elemental Plane of Water}
%%%

The Elemental Plane of Water is a sea without a floor or a surface, an entirely 
fluid environment lit by a diffuse glow. It is one of the more hospitable of the 
Inner Planes once a traveler gets past the problem of breathing the local medium.

The eternal oceans of this plane vary between ice cold and boiling hot, between 
saline and fresh. They are perpetually in motion, wracked by currents and tides. 
The plane's permanent settlements form around bits of flotsam and jetsam suspended 
within this endless liquid. These settlements drift on the tides of the Elemental 
Plane of Water.

The Elemental Plane of Water has the following traits.

\begin{itemize}
\item Subjective directional gravity. The gravity here works similar to that of the Elemental 
Plane of Air. But sinking or rising on the Elemental Plane of Water is slower (and 
less dangerous) than on the Elemental Plane of Air.
\item Water-dominant.
\item Enhanced magic. Spells and spell-like abilities that use or create water are both 
extended and enlarged (as if the Extend Spell and Enlarge Spell metamagic feats 
had been used on them, but the spells don't require higher-level slots). Spells 
and spell-like abilities that are already extended or enlarged are unaffected by 
this benefit.
\item Impeded magic. Spells and spell-like abilities with the fire descriptor (including 
spells of the Fire domain) are impeded. 
\end{itemize}

%%%
\subsubsection{Negative Energy Plane}\index{Negative Energy Plane}
%%%

To an observer, there's little to see on the Negative Energy Plane. It is a dark, 
empty place, an eternal pit where a traveler can fall until the plane itself steals 
away all light and life. The Negative Energy Plane is the most hostile of the Inner 
Planes, and the most uncaring and intolerant of life. Only creatures immune to 
its life-draining energies can survive there. 

The Negative Energy Plane has the following traits.

\begin{itemize}
\item Subjective directional gravity.
\item Major negative-dominant. Some areas within the plane have only the minor negative-dominant 
trait, and these islands tend to be inhabited.
\item Enhanced magic. Spells and spell-like abilities that use negative energy are maximized 
(as if the Maximize Spell metamagic feat had been used on them, but the spells 
don't require higher-level slots). Spells and spell-like abilities that are already 
maximized are unaffected by this benefit. Class abilities that use negative energy, 
such as rebuking and controlling undead, gain a +10 bonus on the roll to determine 
Hit Dice affected. 
\item Impeded magic. Spells and spell-like abilities that use positive energy, including 
\textit{cure} spells, are impeded. Characters on this plane take a -10 penalty 
on Fortitude saving throws made to remove negative levels bestowed by an energy 
drain attack.
\end{itemize}

Random Encounters: Because the Negative Energy Plane is virtually devoid of creatures, 
random encounters on the plane are exceedingly rare.

%%%
\subsubsection{Positive Energy Plane}\index{Positive Energy Plane}
%%%

The Positive Energy Plane has no surface and is akin to the Elemental Plane of 
Air with its wide-open nature. However, every bit of this plane glows brightly 
with innate power. This power is dangerous to mortal forms, which are not made 
to handle it. Despite the beneficial effects of the plane, it is one of the most 
hostile of the Inner Planes. An unprotected character on this plane swells with 
power as positive energy is force-fed into her. Then, her mortal frame unable to 
contain that power, she immolates as if she were a small planet caught at the edge 
of a supernova. Visits to the Positive Energy Plane are brief, and even then travelers 
must be heavily protected.

The Positive Energy Plane has the following traits.

\begin{itemize}
\item Subjective directional gravity.
\item Major positive-dominant. Some regions of the plane have the minor positive-dominant 
trait instead, and those islands tend to be inhabited.
\item Enhanced magic. Spells and spell-like abilities that use positive energy, including 
\textit{cure} spells, are maximized (as if the Maximize Spell metamagic feat had 
been used on them, but the spells don't require higher-level slots). Spells and 
spell-like abilities that are already maximized are unaffected by this benefit. 
Class abilities that use positive energy, such as turning and destroying undead, 
gain a +10 bonus on the roll to determine Hit Dice affected. (Undead are almost 
impossible to find on this plane, however.)
\item Impeded magic. Spells and spell-like abilities that use negative energy (including 
\textit{inflict} spells) are impeded.
\end{itemize}

Random Encounters: Because the Positive Energy Plane is virtually devoid of creatures, 
random encounters on the plane are exceedingly rare.

