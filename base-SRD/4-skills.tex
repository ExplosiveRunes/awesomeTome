%%%%%%%%%%%%%%%%%%%%%%%%%%%%%%%%%%%%%%%%%%%%%%%%%%
%%%%%%%%%%%%%%%%%%%%%%%%%%%%%%%%%%%%%%%%%%%%%%%%%%
\chapter{Skills}\label{chapter:Skills}
%%%%%%%%%%%%%%%%%%%%%%%%%%%%%%%%%%%%%%%%%%%%%%%%%%
%%%%%%%%%%%%%%%%%%%%%%%%%%%%%%%%%%%%%%%%%%%%%%%%%%

%%%%%%%%%%%%%%%%%%%%%%%%%%%%%%%%%%%%%%%%%%%%%%%%%%
\section{Skills Summary}
%%%%%%%%%%%%%%%%%%%%%%%%%%%%%%%%%%%%%%%%%%%%%%%%%%

If you buy a class skill, your character gets 1 rank (equal to a +1 bonus on checks 
with that skill) for each skill point. If you buy other classes' skills (cross-class 
skills), you get 1/2 rank per skill point.

Your maximum rank in a class skill is your character level + 3.

Your maximum rank in a cross-class skill is one-half of this number (do not round 
up or down).

\textbf{Using Skills:} To make a skill check, roll: 1d20 + skill modifier (Skill 
modifier = skill rank + ability modifier + miscellaneous modifiers)

This roll works just like an attack roll or a saving throw -- the higher the roll, 
the better. Either you're trying to match or exceed a certain Difficulty Class 
(DC), or you're trying to beat another character's check result.

\textbf{Skill Ranks:} A character's number of ranks in a skill is based on how 
many skill points a character has invested in a skill. Many skills can be used 
even if the character has no ranks in them; doing this is called making an untrained 
skill check.

\textbf{Ability Modifier:} The ability modifier used in a skill check is the modifier 
for the skill's key ability (the ability associated with the skill's use). The 
key ability of each skill is noted in its description.

\textbf{Miscellaneous Modifiers:} Miscellaneous modifiers include racial bonuses, 
armor check penalties, and bonuses provided by feats, among others.

Each skill point you spend on a class skill gets you 1 rank in that skill. Class 
skills are the skills found on your character's class skill list. Each skill point 
you spend on a cross-class skill gets your character 1/2 rank in that skill. Cross-class 
skills are skills not found on your character's class skill list. (Half ranks do 
not improve your skill check, but two 1/2 ranks make 1 rank.) You can't save skill 
points to spend later.

The maximum rank in a class skill is the character's level + 3. If it's a cross-class 
skill, the maximum rank is half of that number (do not round up or down).

Regardless of whether a skill is purchased as a class skill or a cross-class skill, 
if it is a class skill for any of your classes, your maximum rank equals your total 
character level + 3.

%%%%%%%%%%%%%%%%%%%%%%%%%%%%%%%%%%%%%%%%%%%%%%%%%%
\section{Using Skills}
%%%%%%%%%%%%%%%%%%%%%%%%%%%%%%%%%%%%%%%%%%%%%%%%%%

When your character uses a skill, you make a skill check to see how well he or 
she does. The higher the result of the skill check, the better. Based on the circumstances, 
your result must match or beat a particular number (a DC or the result of an opposed 
skill check) for the check to be successful. The harder the task, the higher the 
number you need to roll.

Circumstances can affect your check. A character who is free to work without distractions 
can make a careful attempt and avoid simple mistakes. A character who has lots 
of time can try over and over again, thereby assuring the best outcome. If others 
help, the character may succeed where otherwise he or she would fail.

%%%%%%%%%%%%%%%%%%%%%%%%%
\subsection{Skill Checks}
%%%%%%%%%%%%%%%%%%%%%%%%%

A skill check takes into account a character's training (skill rank), natural talent 
(ability modifier), and luck (the die roll). It may also take into account his 
or her race's knack for doing certain things (racial bonus) or what armor he or 
she is wearing (armor check penalty), or a certain feat the character possesses, 
among other things. 

To make a skill check, roll 1d20 and add your character's skill modifier for that 
skill. The skill modifier incorporates the character's ranks in that skill and 
the ability modifier for that skill's key ability, plus any other miscellaneous 
modifiers that may apply, including racial bonuses and armor check penalties. The 
higher the result, the better. Unlike with attack rolls and saving throws, a natural 
roll of 20 on the d20 is not an automatic success, and a natural roll of 1 is not 
an automatic failure.

%%%
\subsubsection{Difficulty Class}
%%%

Some checks are made against a Difficulty Class (DC). The DC is a number (set using 
the skill rules as a guideline) that you must score as a result on your skill check 
in order to succeed.

\begin{table}[htb]
\rowcolors{1}{white}{offyellow}
\caption{Difficulty Class Examples}
\centering
\begin{tabular}{l l}
\textbf{Difficulty (DC)} & \textbf{Example (Skill Used)}\\
Very Easy (0) & Notice Something large in plain sight (\linkskill{Spot})\\
Easy (5) & Climb a knotted rope (\linkskill{Climb})\\
Average (10) & Hear an approaching guard (\linkskill{Listen})\\
Tough (15) & Rig a wagon wheel to fall off (\linkskill{Disable Device})\\
Challenging (20) & Swim in stormy water (\linkskill{Swim})\\
Formidable (25) & Open an average lock (\linkskill{Open Lock})\\
Heroic (30) & Leap across a 30-foot chasm (\linkskill{Jump})\\
Nearly Impossible (40) & Track a squad of orcs across hard ground acter 24 hours of rainfall (\linkskill{Survival})\\
\end{tabular}
\end{table}

%%%
\subsubsection{Opposed Checks}
%%%

An opposed check is a check whose success or failure is determined by comparing 
the check result to another character's check result. In an opposed check, the 
higher result succeeds, while the lower result fails. In case of a tie, the higher 
skill modifier wins. If these scores are the same, roll again to break the tie.

\begin{table}[htb]
\rowcolors{1}{white}{offyellow}
\caption{Example Opposed Checks}
\centering
\begin{tabular}{l l l}
\textbf{Task} & \textbf{Skills (Key Ability)} & \textbf{Opposing Skill (Key Ability)}\\
Con someone & \linkskill{Bluff} (Cha) & \linkskill{Sense Motive} (Wis)\\
Pretend to be someone else & \linkskill{Disguise} (Cha) & \linkskill{Spot} (Wis)\\
Create a false map & \linkskill{Forgery} (Int) & \linkskill{Forgery} (Int)\\
Hide form someone & \linkskill{Hide} (Dex) & \linkskill{Spot} (Wis)\\
Make a bully back down & \linkskill{Intimidate} (Cha) & Special\textsuperscript{1}\\
Sneak up on someone & \linkskill{Move Silently} (Dex) & \linkskill{Listen} (Wis)\\
Steal a coin pouch & \linkskill{Sleight of Hand} (Dex) & \linkskill{Spot} (Wis)\\
Tie a prisoner securely & \linkskill{Use Rope} (Dex) & \linkskill{Escape Artist} (Dex)\\
\multicolumn{3}{p{11cm}}{\textsuperscript{1} An Intimidate check is opposed by the target's level check, not a skill check. See the Intimidate skill description for ore information.}\\
\end{tabular}
\end{table}

%%%
\subsubsection{Trying Again}
%%%

In general, you can try a skill check again if you fail, and you can keep trying 
indefinitely. Some skills, however, have consequences of failure that must be taken 
into account. A few skills are virtually useless once a check has failed on an 
attempt to accomplish a particular task. For most skills, when a character has 
succeeded once at a given task, additional successes are meaningless.

%%%
\subsubsection{Untrained Skill Checks}
%%%

Generally, if your character attempts to use a skill he or she does not possess, 
you make a skill check as normal. The skill modifier doesn't have a skill rank 
added in because the character has no ranks in the skill. Any other applicable 
modifiers, such as the modifier for the skill's key ability, are applied to the 
check.

Many skills can be used only by someone who is trained in them.

%%%
\subsubsection{Favorable and Unfavorable Conditions}
%%%

Some situations may make a skill easier or harder to use, resulting in a bonus 
or penalty to the skill modifier for a skill check or a change to the DC of the 
skill check.

The chance of success can be altered in four ways to take into account exceptional 
circumstances.

\begin{enumerate*}
\item Give the skill user a +2 circumstance bonus to represent conditions that improve 
performance, such as having the perfect tool for the job, getting help from another 
character (see \linksec{Combining Skill Attempts}), or possessing unusually accurate information. 

\item Give the skill user a -2 circumstance penalty to represent conditions that hamper 
performance, such as being forced to use improvised tools or having misleading 
information.

\item Reduce the DC by 2 to represent circumstances that make the task easier, such 
as having a friendly audience or doing work that can be subpar.

\item Increase the DC by 2 to represent circumstances that make the task harder, such 
as having an uncooperative audience or doing work that must be flawless.
\end{enumerate*}

Conditions that affect your character's ability to perform the skill change the 
skill modifier. Conditions that modify how well the character has to perform the 
skill to succeed change the DC. A bonus to the skill modifier and a reduction in 
the check's DC have the same result: They create a better chance of success. But 
they represent different circumstances, and sometimes that difference is important.



Using a skill might take a round, take no time, or take several rounds or even 
longer. Most skill uses are standard actions, move actions, or full-round actions. 
Types of actions define how long activities take to perform within the framework 
of a combat round (6 seconds) and how movement is treated with respect to the activity. 
Some skill checks are instant and represent reactions to an event, or are included 
as part of an action.

These skill checks are not actions. Other skill checks represent part of movement.

%%%
\subsubsection{Checks without Rolls}
%%%

A skill check represents an attempt to accomplish some goal, usually while under 
some sort of time pressure or distraction. Sometimes, though, a character can use 
a skill under more favorable conditions and eliminate the luck factor.

\textbf{Taking 10:} When your character is not being threatened or distracted, 
you may choose to take 10. Instead of rolling 1d20 for the skill check, calculate 
your result as if you had rolled a 10. For many routine tasks, taking 10 makes 
them automatically successful. Distractions or threats (such as combat) make it 
impossible for a character to take 10. In most cases, taking 10 is purely a safety 
measure -- you know (or expect) that an average roll will succeed but fear that 
a poor roll might fail, so you elect to settle for the average roll (a 10). Taking 
10 is especially useful in situations where a particularly high roll wouldn't help.

\textbf{Taking 20:} When you have plenty of time (generally 2 minutes for a skill 
that can normally be checked in 1 round, one full-round action, or one standard 
action), you are faced with no threats or distractions, and the skill being attempted 
carries no penalties for failure, you can take 20. In other words, eventually you 
will get a 20 on 1d20 if you roll enough times. Instead of rolling 1d20 for the 
skill check, just calculate your result as if you had rolled a 20.

Taking 20 means you are trying until you get it right, and it assumes that you 
fail many times before succeeding. Taking 20 takes twenty times as long as making 
a single check would take.

Since taking 20 assumes that the character will fail many times before succeeding, 
if you did attempt to take 20 on a skill that carries penalties for failure, your 
character would automatically incur those penalties before he or she could complete 
the task. Common "take 20" skills include Escape Artist, Open Lock, and Search.

\textbf{Ability Checks and Caster Level Checks:} The normal take 10 and take 20 
rules apply for ability checks. Neither rule applies to caster level checks.

%%%%%%%%%%%%%%%%%%%%%%%%%
\subsection{Combining Skill Attempts}
%%%%%%%%%%%%%%%%%%%%%%%%%

When more than one character tries the same skill at the same time and for the 
same purpose, their efforts may overlap.

%%%
\subsubsection{Individual Events}
%%%

Often, several characters attempt some action and each succeeds or fails independently. 
 The result of one character's Climb check does not influence the results of other 
characters Climb check.

%%%
\subsubsection{Aid Another and Skills}
%%%

You can help another character achieve success on his or her skill check by making 
the same kind of skill check in a cooperative effort. If you roll a 10 or higher 
on your check, the character you are helping gets a +2 bonus to his or her check, 
as per the rule for favorable conditions. (You can't take 10 on a skill check to 
aid another.) In many cases, a character's help won't be beneficial, or only a 
limited number of characters can help at once. 

In cases where the skill restricts who can achieve certain results you can't aid 
another to grant a bonus to a task that your character couldn't achieve alone.

%%%
\subsubsection{Skill Synergy}
%%%

It's possible for a character to have two skills that work well together. In general, 
having 5 or more ranks in one skill gives the character a +2 bonus on skill checks 
with each of its synergistic skills, as noted in the skill description. In some 
cases, this bonus applies only to specific uses of the skill in question, and not 
to all checks. Some skills provide benefits on other checks made by a character, 
such as those checks required to use certain class features.

%%%%%%%%%%%%%%%%%%%%%%%%%
\subsection{Ability Checks}
%%%%%%%%%%%%%%%%%%%%%%%%%

Sometimes a character tries to do something to which no specific skill really applies. 
In these cases, you make an ability check. An ability check is a roll of 1d20 plus 
the appropriate ability modifier. Essentially, you're making an untrained skill 
check. 

In some cases, an action is a straight test of one's ability with no luck involved. 
Just as you wouldn't make a height check to see who is taller, you don't make a 
Strength check to see who is stronger.

%%%%%%%%%%%%%%%%%%%%%%%%%%%%%%%%%%%%%%%%%%%%%%%%%%
\section{Skill Descriptions}
%%%%%%%%%%%%%%%%%%%%%%%%%%%%%%%%%%%%%%%%%%%%%%%%%%

This section describes each skill, including common uses and typical modifiers. 
Characters can sometimes use skills for purposes other than those noted here.

Here is the format for skill descriptions.

%%%%%%%%%%%%%%%%%%%%%%%%%
\subsection{Skill Name}
%%%%%%%%%%%%%%%%%%%%%%%%%

The skill name line includes (in addition to the name of the skill) the following 
information.

\textbf{Key Ability:} The abbreviation of the ability whose modifier applies to 
the skill check. \textit{Exception:} Speak Language has "None" as its key ability 
because the use of this skill does not require a check.

\textbf{Trained Only:} If this notation is included in the skill name line, you 
must have at least 1 rank in the skill to use it. If it is omitted, the skill can 
be used untrained (with a rank of 0). If any special notes apply to trained or 
untrained use, they are covered in the Untrained section (see below).

\textbf{Armor Check Penalty:} If this notation is included in the skill name line, 
an armor check penalty applies (when appropriate) to checks using this skill. If 
this entry is absent, an armor check penalty does not apply.

\vspace{12pt}
The skill name line is followed by a general description of what using the skill 
represents. After the description are a few other types of information:

\textbf{Check:} What a character ("you" in the skill description) can do with 
a successful skill check and the check's DC.

\textbf{Action:} The type of action using the skill requires, or the amount of 
time required for a check.

\textbf{Try Again:} Any conditions that apply to successive attempts to use the 
skill successfully. If the skill doesn't allow you to attempt the same task more 
than once, or if failure carries an inherent penalty (such as with the Climb skill), 
you can't take 20. If this paragraph is omitted, the skill can be retried without 
any inherent penalty, other than the additional time required.

\textbf{Special:} Any extra facts that apply to the skill, such as special effects 
deriving from its use or bonuses that certain characters receive because of class, 
feat choices, or race.

\textbf{Synergy:} Some skills grant a bonus to the use of one or more other skills 
because of a synergistic effect. This entry, when present, indicates what bonuses 
this skill may grant or receive because of such synergies. See Table 4-5 for a 
complete list of bonuses granted by synergy between skills (or between a skill 
and a class feature).

\textbf{Restriction:} The full utility of certain skills is restricted to characters 
of certain classes or characters who possess certain feats. This entry indicates 
whether any such restrictions exist for the skill.

\textbf{Untrained:} This entry indicates what a character without at least 1 rank 
in the skill can do with it. If this entry doesn't appear, it means that the skill 
functions normally for untrained characters (if it can be used untrained) or that 
an untrained character can't attempt checks with this skill (for skills that are 
designated as "Trained Only").

%%%%%%%%%%%%%%%%%%%%%%%%%
\skillentry{Appraise}{(Int)}
%%%%%%%%%%%%%%%%%%%%%%%%%

\textbf{Check:} You can appraise common or well-known objects with a DC 12 Appraise 
check. Failure means that you estimate the value at 50\% to 150\% (2d6+3 times 
10\%,) of its actual value.

Appraising a rare or exotic item requires a successful check against DC 15, 20, 
or higher. If the check is successful, you estimate the value correctly; failure 
means you cannot estimate the item's value.

A magnifying glass gives you a +2 circumstance bonus on Appraise checks involving 
any item that is small or highly detailed, such as a gem. A merchant's scale gives 
you a +2 circumstance bonus on Appraise checks involving any items that are valued 
by weight, including anything made of precious metals.

These bonuses stack.

\textbf{Action:} Appraising an item takes 1 minute (ten consecutive full-round 
actions).

\textbf{Try Again:} No. You cannot try again on the same object, regardless of 
success.

\textbf{Special:} A \linkrace{Dwarf} gets a +2 racial bonus on Appraise checks that are related 
to stone or metal items because dwarves are familiar with valuable items of all 
kinds (especially those made of stone or metal).

The master of a raven familiar gains a +3 bonus on Appraise checks.

A character with the \linkfeat{Diligent} feat gets a +2 bonus on Appraise checks.

\textbf{Synergy:} If you have 5 ranks in any \linkskill{Craft} skill, you gain a +2 bonus on 
Appraise checks related to items made with that Craft skill.

\textbf{Untrained:} For common items, failure on an untrained check means no estimate. 
For rare items, success means an estimate of 50\% to 150\% (2d6+3 times 10\%).

%%%%%%%%%%%%%%%%%%%%%%%%%
\skillentry{Balance}{(Dex; Armor Check Penalty)}
%%%%%%%%%%%%%%%%%%%%%%%%%

\textbf{Check:} You can walk on a precarious surface. A successful check lets you 
move at half your speed along the surface for 1 round. A failure by 4 or less means 
you can't move for 1 round. A failure by 5 or more means you fall. The difficulty 
varies with the surface, as follows:

\begin{table}[htb]
\rowcolors{1}{white}{offyellow}
\caption{Balance DCs}
\centering
\begin{tabular}{l c l c}
\textbf{Narrow Surface} & \textbf{Balance DC\textsuperscript{1}} & \textbf{Difficult Surface} & \textbf{Balance DC}\\
7-12 inches wide & 10 & Uneven Flagstone & 10\textsuperscript{2}\\
2-6 inches wide & 15 & Hewn Stone Floor & 10\textsuperscript{2}\\
Less than 2 inches wide & 20 & Sloped or Angled Floor & 10\textsuperscript{2}\\
\multicolumn{4}{p{11cm}}{\textsuperscript{1} Add modifiers from Narrow Surface Modifiers, below, as appropriate.}\\
\multicolumn{4}{p{11cm}}{\textsuperscript{2} Only if running or charging. Failure by 4 or less means the character can't run or charge, but my otherwise act normally.}\\
\end{tabular}
\end{table}

\begin{table}[htb]
\rowcolors{1}{white}{offyellow}
\caption{Narrow Surface Modifiers}
\centering
\begin{tabular}{l c}
\textbf{Surface} & \textbf{DC Modifer\textsuperscript{1}}\\
Lightly Obsctructed & +2\\
Severely Obstructed & +5\\
Lightly Slippery & +2\\
Severly Slipper & +5\\
Sloped or Angled & +2\\
\multicolumn{2}{p{7cm}}{\textsuperscript{1} Add the appropriate modifier to the Balance DC of a narrow surface. These modifiers stack.}\\
\end{tabular}
\end{table}

\textit{Being Attacked while Balancing:} You are considered flat-footed while balancing, 
since you can't move to avoid a blow, and thus you lose your Dexterity bonus to 
AC (if any). If you have 5 or more ranks in Balance, you aren't considered flat-footed 
while balancing. If you take damage while balancing, you must make another Balance 
check against the same DC to remain standing.

\textit{Accelerated Movement:} You can try to walk across a precarious surface 
more quickly than normal. If you accept a -5 penalty, you can move your full speed 
as a move action. (Moving twice your speed in a round requires two Balance checks, 
one for each move action used.) You may also accept this penalty in order to charge 
across a precarious surface; charging requires one Balance check for each multiple 
of your speed (or fraction thereof) that you charge.

\textbf{Action:} None. A Balance check doesn't require an action; it is made as 
part of another action or as a reaction to a situation.

\textbf{Special:} If you have the \linkfeat{Agile} feat, you get a +2 bonus on Balance checks.

\textbf{Synergy:} If you have 5 or more ranks in \linkskill{Tumble}, you get a +2 bonus on 
Balance checks.

%%%%%%%%%%%%%%%%%%%%%%%%%
\skillentry{Bluff}{(Cha)}
%%%%%%%%%%%%%%%%%%%%%%%%%

\textbf{Check:} A Bluff check is opposed by the target's \linkskill{Sense Motive} check. See 
the accompanying table for examples of different kinds of bluffs and the modifier 
to the target's Sense Motive check for each one.

Favorable and unfavorable circumstances weigh heavily on the outcome of a bluff. 
Two circumstances can weigh against you: The bluff is hard to believe, or the action 
that the target is asked to take goes against its self-interest, nature, personality, 
orders, or the like. If it's important, you can distinguish between a bluff that 
fails because the target doesn't believe it and one that fails because it just 
asks too much of the target. For instance, if the target gets a +10 bonus on its 
Sense Motive check because the bluff demands something risky, and the Sense Motive 
check succeeds by 10 or less, then the target didn't so much see through the bluff 
as prove reluctant to go along with it. A target that succeeds by 11 or more has 
seen through the bluff.

A successful Bluff check indicates that the target reacts as you wish, at least 
for a short time (usually 1 round or less) or believes something that you want 
it to believe. Bluff, however, is not a \linkspell{Suggestion} spell. 

A bluff requires interaction between you and the target. Creatures unaware of you 
cannot be bluffed.

\textit{Feinting in Combat:} You can also use Bluff to mislead an opponent in melee 
combat (so that it can't dodge your next attack effectively). To feint, make a 
Bluff check opposed by your target's Sense Motive check, but in this case, the 
target may add its base attack bonus to the roll along with any other applicable 
modifiers.

If your Bluff check result exceeds this special Sense Motive check result, your 
target is denied its Dexterity bonus to AC (if any) for the next melee attack you 
make against it. This attack must be made on or before your next turn.

Feinting in this way against a nonhumanoid is difficult because it's harder to 
read a strange creature's body language; you take a -4 penalty on your Bluff check. 
Against a creature of animal Intelligence (1 or 2) it's even harder; you take a 
-8 penalty. Against a nonintelligent creature, it's impossible.

Feinting in combat does not provoke an attack of opportunity.

\textit{Creating a Diversion to Hide:} You can use the Bluff skill to help you 
hide. A successful Bluff check gives you the momentary diversion you need to attempt 
a Hide check while people are aware of you. This usage does not provoke an attack 
of opportunity.

\textit{Delivering a Secret Message:} You can use Bluff to get a message across 
to another character without others understanding it. The DC is 15 for simple messages, 
or 20 for complex messages, especially those that rely on getting across new information. 
Failure by 4 or less means you can't get the message across. Failure by 5 or more 
means that some false information has been implied or inferred. Anyone listening 
to the exchange can make a Sense Motive check opposed by the Bluff check you made 
to transmit in order to intercept your message (see Sense Motive).

\textbf{Action:} Varies. A Bluff check made as part of general interaction always 
takes at least 1 round (and is at least a full-round action), but it can take much 
longer if you try something elaborate. A Bluff check made to feint in combat or 
create a diversion to hide is a standard action. A Bluff check made to deliver 
a secret message doesn't take an action; it is part of normal communication.

\textbf{Try Again:} Varies. Generally, a failed Bluff check in social interaction 
makes the target too suspicious for you to try again in the same circumstances, 
but you may retry freely on Bluff checks made to feint in combat. Retries are also 
allowed when you are trying to send a message, but you may attempt such a retry 
only once per round.

Each retry carries the same chance of miscommunication.

\textbf{Special:} A \linkclass{Ranger} gains a bonus on Bluff checks when using this skill 
against a favored enemy.

The master of a snake familiar gains a +3 bonus on Bluff checks.

If you have the \linkfeat{Persuasive} feat, you get a +2 bonus on Bluff checks.

\textbf{Synergy:} If you have 5 or more ranks in Bluff, you get a +2 bonus on \linkskill{Diplomacy}, 
\linkskill{Intimidate}, and \linkskill{Sleight of Hand} checks, as well as on \linkskill{Disguise} checks made when 
you know you're being observed and you try to act in character.

\begin{table}[htb]
\rowcolors{1}{white}{offyellow}
\caption{Bluff Examples}
\centering
\begin{tabular}{l c l c}
\textbf{Example Circumstances} & \textbf{Sense Motive Modifier}\\
The target wants to believe you. & -5\\
The bluff is believable and doesn't affect the target much. & +0\\
The bluff is a little hard to believe or puts the target at some risk. & +5\\
The bluff is hard to believe or puts the target as significant risk. & +10\\
The bluff is way tout there, almost too incredible to consider. & +20\\
\end{tabular}
\end{table}
%%%%%%%%%%%%%%%%%%%%%%%%%
\skillentry{Climb}{(Str; Armor Check Penalty)}
%%%%%%%%%%%%%%%%%%%%%%%%%

\textbf{Check:} With a successful Climb check, you can advance up, down, or across 
a slope, a wall, or some other steep incline (or even a ceiling with handholds) 
at one-quarter your normal speed. A slope is considered to be any incline at an 
angle measuring less than 60 degrees; a wall is any incline at an angle measuring 
60 degrees or more.

A Climb check that fails by 4 or less means that you make no progress, and one 
that fails by 5 or more means that you fall from whatever height you have already 
attained.

A climber's kit gives you a +2 circumstance bonus on Climb checks.

The DC of the check depends on the conditions of the climb. Compare the task with 
those on the following table to determine an appropriate DC.

\begin{table}[htb]
\rowcolors{1}{white}{offyellow}
\caption{Example Climb DCs}
\centering
\begin{tabular}{c p{12cm}}
\textbf{Climb DC} & \textbf{Example Surface or Activity}\\
0 & A slope too steep to walk up, or a knotted rope with a wall to brace against.\\
5 & A rope with a wall to brace against, or a knotted rope, or a rope affected by the \linkspell{Rope Trick} spell.\\
10 & A surface with ledges to hold on to and stand on, such as a very rough wall or a ship's rigging.\\
15 & Any surface with adequate handholds and footholds (natural or artificial), such as a very rough natural rock surface or a tree, or an unknotted rope, or pulling yourself up when dangling by your hands. \\
20 & An uneven surface with some narrow handholds and footholds, such as a typical wall in a dungeon or ruins. \\
25 & A rough surface, such as a natural rock wall or a brick wall. \\
25 & An overhang or ceiling with handholds but no footholds. \\
-- & A perfectly smooth, flat, vertical surface cannot be climbed.\\
\end{tabular}
\end{table}

\begin{table}[htb]
\rowcolors{1}{white}{offyellow}
\caption{Example Climb DCs Modifiers}
\centering
\begin{tabular}{c p{12.5cm}}
\textbf{Climb DC Modifier\textsuperscript{1}} & \textbf{Example Surface or Activity}\\
-10 & Climbing a chimney (artificial or natural) or other location where you can brace against two opposite walls (reduces DC by 10).\\
-5 & Climbing a corner where you can brace against perpendicular walls (reduces DC by 5).\\
+5 & Surface is slippery (increases DC by 5).\\
\multicolumn{2}{l}{\textsuperscript{1}These modifiers are cumulative; use any that apply.}
\end{tabular}
\end{table}

You need both hands free to climb, but you may cling to a wall with one hand while 
you cast a spell or take some other action that requires only one hand. While climbing, 
you can't move to avoid a blow, so you lose your Dexterity bonus to AC (if any). 
You also can't use a shield while climbing.

Any time you take damage while climbing, make a Climb check against the DC of the 
slope or wall. Failure means you fall from your current height and sustain the 
appropriate falling damage.

\textit{Accelerated Climbing:} You try to climb more quickly than normal. By accepting 
a -5 penalty, you can move half your speed (instead of one-quarter your speed).

\textit{Making Your Own Handholds and Footholds:} You can make your own handholds 
and footholds by pounding pitons into a wall. Doing so takes 1 minute per piton, 
and one piton is needed per 3 feet of distance. As with any surface that offers 
handholds and footholds, a wall with pitons in it has a DC of 15. In the same way, 
a climber with a handaxe or similar implement can cut handholds in an ice wall.

\textit{Catching Yourself When Falling:} It's practically impossible to catch yourself 
on a wall while falling. Make a Climb check (DC = wall's DC + 20) to do so. It's 
much easier to catch yourself on a slope (DC = slope's DC + 10).

\textit{Catching a Falling Character While Climbing:} If someone climbing above 
you or adjacent to you falls, you can attempt to catch the falling character if 
he or she is within your reach. Doing so requires a successful melee touch attack 
against the falling character (though he or she can voluntarily forego any Dexterity 
bonus to AC if desired). If you hit, you must immediately attempt a Climb check 
(DC = wall's DC + 10). Success indicates that you catch the falling character, 
but his or her total weight, including equipment, cannot exceed your heavy load 
limit or you automatically fall. If you fail your Climb check by 4 or less, you 
fail to stop the character's fall but don't lose your grip on the wall. If you 
fail by 5 or more, you fail to stop the character's fall and begin falling as well.

\textbf{Action:} Climbing is part of movement, so it's generally part of a move 
action (and may be combined with other types of movement in a move action). Each 
move action that includes any climbing requires a separate Climb check. Catching 
yourself or another falling character doesn't take an action.

\textbf{Special:} You can use a rope to haul a character upward (or lower a character) 
through sheer strength. You can lift double your maximum load in this manner.

A \linkrace{Halfling} has a +2 racial bonus on Climb checks because halflings are agile and 
surefooted.

The master of a lizard familiar gains a +3 bonus on Climb checks.

If you have the \linkfeat{Athletic} feat, you get a +2 bonus on Climb checks.

A creature with a climb speed has a +8 racial bonus on all Climb checks. The creature 
must make a Climb check to climb any wall or slope with a DC higher than 0, but 
it always can choose to take 10, even if rushed or threatened while climbing. If 
a creature with a climb speed chooses an accelerated climb (see above), it moves 
at double its climb speed (or at its land speed, whichever is slower) and makes 
a single Climb check at a -5 penalty. Such a creature retains its Dexterity bonus 
to Armor Class (if any) while climbing, and opponents get no special bonus to their 
attacks against it. It cannot, however, use the run action while climbing.

\textbf{Synergy:} If you have 5 or more ranks in \linkskill{Use Rope}, you get a +2 bonus on 
Climb checks made to climb a rope, a knotted rope, or a rope-and-wall combination.

%%%%%%%%%%%%%%%%%%%%%%%%%
\skillentry{Concentration}{(Con)}
%%%%%%%%%%%%%%%%%%%%%%%%%

\textbf{Check:} You must make a Concentration check whenever you might potentially 
be distracted (by taking damage, by harsh weather, and so on) while engaged in 
some action that requires your full attention. Such actions include casting a spell, 
concentrating on an active spell, directing a spell, using a spell-like ability, 
or using a skill that would provoke an attack of opportunity. In general, if an 
action wouldn't normally provoke an attack of opportunity, you need not make a 
Concentration check to avoid being distracted.

If the Concentration check succeeds, you may continue with the action as normal. 
If the check fails, the action automatically fails and is wasted. If you were in 
the process of casting a spell, the spell is lost. If you were concentrating on 
an active spell, the spell ends as if you had ceased concentrating on it. If you 
were directing a spell, the direction fails but the spell remains active. If you 
were using a spell-like ability, that use of the ability is lost. A skill use also 
fails, and in some cases a failed skill check may have other ramifications as well.

The table below summarizes various types of distractions that cause you to make 
a Concentration check. If the distraction occurs while you are trying to cast a 
spell, you must add the level of the spell you are trying to cast to the appropriate 
Concentration DC. If more than one type of distraction is present, make a check 
for each one; any failed Concentration check indicates that the task is not completed.

\begin{table}[htb]
\rowcolors{1}{white}{offyellow}
\caption{Concentration DCs}
\centering
\begin{tabular}{l p{12cm}}
\textbf{Concentration DC} & \textbf{Distraction}\\
10 + damage dealt & Damaged during the action.\textsuperscript{2}\\
10 + half of continuous & Taking continuous damage during the damage last dealt action.\textsuperscript{3}\\
Distracting spell's save DC & Distracted by nondamaging spell.\textsuperscript{4}\\
10 & Vigorous motion (on a moving mount, taking a bouncy wagon ride, in a small boat in rough water, belowdecks in a stormtossed ship).\\
15 & Violent motion (on a galloping horse, taking a very rough wagon ride, in a small boat in rapids, on the deck of a storm-tossed ship).\\
20 & Extraordinarily violent motion (earthquake).\\
15 & Entangled.\\
20 & Grappling or pinned. (You can cast only spells without somatic components for which you have any required material component in hand.)\\
5 & Weather is a high wind carrying blinding rain or sleet.\\
10 & Weather is wind-driven hail, dust, or debris.\\
Distracting spell's save DC & Weather caused by a spell, such as \linkspell{Storm of Vengeance}.\textsuperscript{4}\\
\multicolumn{2}{p{\linewidth}}{\textsuperscript{1} If you are trying to cast, concentrate on, or direct a spell when the distraction occurs, add the level of the spell to the indicated DC.}\\
\multicolumn{2}{p{\linewidth}}{\textsuperscript{2} Such as during the casting of a spell with a casting time of 1 round or more, or the execution of an activity that takes more than a single full-round action (such as Disable Device). Also, damage stemming from an attack of opportunity or readied attack made in response to the spell being cast (for spells with a casting time of 1 action) or the action being taken (for activities requiring no more than a full-round action).}\\
\multicolumn{2}{p{\linewidth}}{\textsuperscript{3} Such as from \linkspell{Acid Arrow}.}\\
\multicolumn{2}{p{\linewidth}}{\textsuperscript{4} If the spell allows no save, use the save DC it would have if it did allow a save.}\\
\end{tabular}
\end{table}

\textbf{Action:} None. Making a Concentration check doesn't take an action; it 
is either a free action (when attempted reactively) or part of another action (when 
attempted actively).

\textbf{Try Again:} Yes, though a success doesn't cancel the effect of a previous 
failure, such as the loss of a spell you were casting or the disruption of a spell 
you were concentrating on.

\textbf{Special:} You can use Concentration to cast a spell, use a spell-like ability, 
or use a skill defensively, so as to avoid attacks of opportunity altogether. This 
doesn't apply to other actions that might provoke attacks of opportunity.

The DC of the check is 15 (plus the spell's level, if casting a spell or using 
a spell-like ability defensively). If the Concentration check succeeds, you may 
attempt the action normally without provoking any attacks of opportunity. A successful 
Concentration check still doesn't allow you to take 10 on another check if you 
are in a stressful situation; you must make the check normally. If the Concentration 
check fails, the related action also automatically fails (with any appropriate 
ramifications), and the action is wasted, just as if your concentration had been 
disrupted by a distraction. 

A character with the \linkfeat{Combat Casting} feat gets a +4 bonus on Concentration checks 
made to cast a spell or use a spell-like ability while on the defensive or while 
grappling or pinned.

%%%%%%%%%%%%%%%%%%%%%%%%%
\skillentry{Craft}{(Int)}
%%%%%%%%%%%%%%%%%%%%%%%%%

Like \linkskill{Knowledge}, \linkskill{Perform}, and \linkskill{Profession}, Craft is actually a number of separate 
skills. You could have several Craft skills, each with its own ranks, each purchased 
as a separate skill.

A Craft skill is specifically focused on creating something. If nothing is created 
by the endeavor, it probably falls under the heading of a \linkskill{Profession} skill.

\textbf{Check:} You can practice your trade and make a decent living, earning about 
half your check result in gold pieces per week of dedicated work. You know how 
to use the tools of your trade, how to perform the craft's daily tasks, how to 
supervise untrained helpers, and how to handle common problems. (Untrained laborers 
and assistants earn an average of 1 silver piece per day.)

The basic function of the Craft skill, however, is to allow you to make an item 
of the appropriate type. The DC depends on the complexity of the item to be created. 
The DC, your check results, and the price of the item determine how long it takes 
to make a particular item. The item's finished price also determines the cost of 
raw materials.

In some cases, the \linkspell{Fabricate} spell can be used to achieve the results 
of a Craft check with no actual check involved. However, you must make an appropriate 
Craft check when using the spell to make articles requiring a high degree of craftsmanship.

A successful Craft check related to woodworking in conjunction with the casting 
of the \linkspell{Ironwood} spell enables you to make wooden items that have the strength 
of steel.

When casting the spell \linkspell{Minor Creation}, you must succeed on an appropriate 
Craft check to make a complex item.

All crafts require artisan's tools to give the best chance of success. If improvised 
tools are used, the check is made with a -2 circumstance penalty. On the other 
hand, masterwork artisan's tools provide a +2 circumstance bonus on the check.

To determine how much time and money it takes to make an item, follow these steps.

\begin{enumerate}
\item Find the item's price. Put the price in silver pieces (1 gp = 10 sp).
\item Find the DC from the table below.
\item Pay one-third of the item's price for the cost of raw materials.
\item Make an appropriate Craft check representing one week's work. If the check succeeds, 
multiply your check result by the DC. If the result x the DC 
equals the price of the item in sp, then you have completed the item. (If the result 
x the DC equals double or triple the price of the item in silver 
pieces, then you've completed the task in one-half or one-third of the time. Other 
multiples of the DC reduce the time in the same manner.) If the result x 
the DC doesn't equal the price, then it represents the progress you've made this 
week. Record the result and make a new Craft check for the next week. Each week, 
you make more progress until your total reaches the price of the item in silver 
pieces.
\end{enumerate}

If you fail a check by 4 or less, you make no progress this week.

If you fail by 5 or more, you ruin half the raw materials and have to pay half 
the original raw material cost again.

\textit{Progress by the Day:} You can make checks by the day instead of by the 
week. In this case your progress (check result x DC) is in copper 
pieces instead of silver pieces.

\textit{Creating Masterwork Items:} You can make a masterwork item---a weapon, 
suit of armor, shield, or tool that conveys a bonus on its use through its exceptional 
craftsmanship, not through being magical. To create a masterwork item, you create 
the masterwork component as if it were a separate item in addition to the standard 
item. The masterwork component has its own price (300 gp for a weapon or 150 gp 
for a suit of armor or a shield) and a Craft DC of 20. Once both the standard component 
and the masterwork component are completed, the masterwork item is finished. \textit{Note:} The cost you pay for the masterwork component is one-third of the given amount, 
just as it is for the cost in raw materials.

\textit{Repairing Items:} Generally, you can repair an item by making checks against 
the same DC that it took to make the item in the first place. The cost of repairing 
an item is one-fifth of the item's price. 

When you use the Craft skill to make a particular sort of item, the DC for checks 
involving the creation of that item are typically as given on the following table.

\begin{table}[htb]
\rowcolors{1}{white}{offyellow}
\caption{Craft DCs}
\centering
\begin{tabular}{l l c}
\textbf{Item} & \textbf{Craft Skill} & \textbf{Craft DC}\\
Acid & Alchemy\textsuperscript{1} & 15	\\
Alchemist's fire, smokestick, or tindertwig & Alchemy\textsuperscript{1} & 20\\
Antitoxin, sunrod, tanglefoot bag, or thunderstone & Alchemy\textsuperscript{1} & 25\\
Armor or shield  & Armorsmithing & 10 + AC bonus\\
Longbow or shortbow & Bowmaking & 12\\
Composite longbow or composite shortbow & Bowmaking & 15\\
Composite longbow or composite shortbow with high strength rating & Bowmaking & 15 + (2 x rating)\\
Crossbow & Weaponsmithing & 15\\
Simple melee or thrown weapon & Weaponsmithing & 12\\
Martial melee or thrown weapon & Weaponsmithing & 15\\
Exotic melee or thrown weapon & Weaponsmithing & 18\\
Mechanical trap & 	Trapmaking & Varies\textsuperscript{2}\\
Very simple item (wooden spoon) & Varies & 5\\
Typical item (iron pot) & Varies & 10\\
High-quality item (bell) & Varies & 15\\
Complex or superior item (lock) & Varies & 20\\
\multicolumn{3}{l}{\textsuperscript{1} You must be a spellcaster to craft any of these items.}\\
\multicolumn{3}{l}{\textsuperscript{2} Traps have their own rules for construction.}\\
\end{tabular}
\end{table}

\textbf{Action:} Does not apply. Craft checks are made by the day or week (see 
above).

\textbf{Try Again:} Yes, but each time you miss by 5 or more, you ruin half the 
raw materials and have to pay half the original raw material cost again.

\textbf{Special:} A \linkrace{Dwarf} has a +2 racial bonus on Craft checks that are related 
to stone or metal, because dwarves are especially capable with stonework and metalwork.

A \linkrace{Gnome} has a +2 racial bonus on Craft (alchemy) checks because gnomes have sensitive 
noses.

You may voluntarily add +10 to the indicated DC to craft an item. This allows you 
to create the item more quickly (since you'll be multiplying this higher DC by 
your Craft check result to determine progress). You must decide whether to increase 
the DC before you make each weekly or daily check.

To make an item using Craft (alchemy), you must have alchemical equipment and be 
a spellcaster. If you are working in a city, you can buy what you need as part 
of the raw materials cost to make the item, but alchemical equipment is difficult 
or impossible to come by in some places. Purchasing and maintaining an alchemist's 
lab grants a +2 circumstance bonus on Craft (alchemy) checks because you have the 
perfect tools for the job, but it does not affect the cost of any items made using 
the skill.

\textbf{Synergy:} If you have 5 ranks in a Craft skill, you get a +2 bonus on \linkskill{Appraise} 
checks related to items made with that Craft skill.

%%%%%%%%%%%%%%%%%%%%%%%%%
\skillentry{Decipher Script}{(Int; Trained Only)}
%%%%%%%%%%%%%%%%%%%%%%%%%

\textbf{Check:} You can decipher writing in an unfamiliar language or a message 
written in an incomplete or archaic form. The base DC is 20 for the simplest messages, 
25 for standard texts, and 30 or higher for intricate, exotic, or very old writing.

If the check succeeds, you understand the general content of a piece of writing 
about one page long (or the equivalent). If the check fails, make a DC 5 Wisdom 
check to see if you avoid drawing a false conclusion about the text. (Success means 
that you do not draw a false conclusion; failure means that you do.)

Both the Decipher Script check and (if necessary) the Wisdom check are made secretly, 
so that you can't tell whether the conclusion you draw is true or false.

\textbf{Action:} Deciphering the equivalent of a single page of script takes 1 
minute (ten consecutive full-round actions).

\textbf{Try Again:} No.

\textbf{Special:} A character with the \linkfeat{Diligent} feat gets a +2 bonus on Decipher 
Script checks.

\textbf{Synergy:} If you have 5 or more ranks in Decipher Script, you get a +2 
bonus on \linkskill{Use Magic Device} checks involving scrolls.

%%%%%%%%%%%%%%%%%%%%%%%%%
\skillentry{Diplomacy}{(Cha)}
%%%%%%%%%%%%%%%%%%%%%%%%%

\textbf{Check:} You can change the attitudes of others (nonplayer characters) with 
a successful Diplomacy check; see the Influencing NPC Attitudes sidebar, below, 
for basic DCs. In negotiations, participants roll opposed Diplomacy checks, and 
the winner gains the advantage. Opposed checks also resolve situations when two 
advocates or diplomats plead opposite cases in a hearing before a third party.

\textbf{Action:} Changing others' attitudes with Diplomacy generally takes at least 
1 full minute (10 consecutive full-round actions). In some situations, this time 
requirement may greatly increase. A rushed Diplomacy check can be made as a full-round 
action, but you take a -10 penalty on the check.

\textbf{Try Again:} Optional, but not recommended because retries usually do not 
work. Even if the initial Diplomacy check succeeds, the other character can be 
persuaded only so far, and a retry may do more harm than good. If the initial check 
fails, the other character has probably become more firmly committed to his position, 
and a retry is futile.

\textbf{Special:} A \linkrace{Half-Elf} has a +2 racial bonus on Diplomacy checks.

If you have the \linkfeat{Negotiator} feat, you get a +2 bonus on Diplomacy checks.

\textbf{Synergy:} If you have 5 or more ranks in \linkskill{Bluff}, \linkskill{Knowledge} (nobility and 
royalty), or \linkskill{Sense Motive}, you get a +2 bonus on Diplomacy checks.

%%%
\subsubsection{Influencing NPC Attitudes}
%%%

Use the table below to determine the effectiveness of Diplomacy checks (or Charisma 
checks) made to influence the attitude of a nonplayer character, or wild empathy 
checks made to influence the attitude of an animal or magical beast.

\begin{table}[htb]
\rowcolors{1}{white}{offyellow}
\caption{New Attitude DCs}
\centering
\begin{tabular}{l *{5}{c}}
\textbf{Initial Attitude} & \textbf{Hostile} & \textbf{Unfriendly} & \textbf{Indifferent} & \textbf{Friendly} & \textbf{Hlepful}\\
Hostile & Less than 20 & 20 & 25 & 35 & 50\\
Unfriendly & Less than 5 & 5 & 15 & 25 & 40\\
Indifferent & -- & Less than 1 & 1 & 15 & 30\\
Hostile & -- & -- & Less than 1 & 1 & 20\\
Hostile & -- & -- & -- & Less than 1 & 1\\
\end{tabular}
\end{table}

\begin{table}[htb]
\rowcolors{1}{white}{offyellow}
\caption{NPC Attitude Explanations}
\centering
\begin{tabular}{l l l}
\textbf{Attitude} & \textbf{Means} & \textbf{Possible Actions}\\
Hostile & Will take risks to hurt you & Attack, interfere, berate, flee\\
Unfriendly & Wishes you ill & Mislead, gossip, avoid, watch suspiciously, insult\\
Indifferent & Doesn't much care & Socially expected interaction\\
Friendly & Wishes you well & Chat, advise, offer limited help, advocate\\
Helpful & Will take risks to help you & Protect, back up, heal, aid\\
\end{tabular}
\end{table}
%%%%%%%%%%%%%%%%%%%%%%%%%
\skillentry{Disable Device}{(Int; Trained Only)}
%%%%%%%%%%%%%%%%%%%%%%%%%

\textbf{Check:} The Disable Device check is made secretly, so that you don't necessarily 
know whether you've succeeded.

The DC depends on how tricky the device is. Disabling (or rigging or jamming) a 
fairly simple device has a DC of 10; more intricate and complex devices have higher 
DCs.

If the check succeeds, you disable the device. If it fails by 4 or less, you have 
failed but can try again. If you fail by 5 or more, something goes wrong. If the 
device is a trap, you spring it. If you're attempting some sort of sabotage, you 
think the device is disabled, but it still works normally.

You also can rig simple devices such as saddles or wagon wheels to work normally 
for a while and then fail or fall off some time later (usually after 1d4 rounds 
or minutes of use).

\begin{table}[htb]
\rowcolors{1}{white}{offyellow}
\caption{NPC Attitude Explanations}
\centering
\begin{tabular}{l l c l}
\textbf{Device} & \textbf{Time} & \textbf{Disable Device DC\textsuperscript{1}} & \textbf{Example}\\
Simple & 1 round & 10 & Jam a lock\\
Tricky & 1d4 rounds & 15 & Sabotage a wagon wheel\\
Difficult & 2d4 rounds & 20 & Disarm a trap, reset a trap\\
Wicked & 2d4 rounds & 25 & Disarm a complex trap, cleverly sabotage a clockwork device\\
\multicolumn{4}{l}{\textsuperscript{1}If you attempt to leave behind no trace of your tampering, add 5 to the DC.}\\
\end{tabular}
\end{table}

\textbf{Action:} The amount of time needed to make a Disable Device check depends 
on the task, as noted above. Disabling a simple device takes 1 round and is a full-round 
action. An intricate or complex device requires 1d4 or 2d4 rounds.

\textbf{Try Again:} Varies. You can retry if you have missed the check by 4 or 
less, though you must be aware that you have failed in order to try again.

\textbf{Special:} If you have the \linkfeat{Nimble Fingers} feat, you get a +2 bonus on Disable 
Device checks.

A \linkclass{Rogue} who beats a trap's DC by 10 or more can study the trap, figure out how 
it works, and bypass it (along with her companions) without disarming it.

\textbf{Restriction:} Rogues (and other characters with the Trapfinding class feature) 
can disarm magic traps. A magic trap generally has a DC of 25 + the spell level 
of the magic used to create it.

The spells \linkspell{Fire Trap}, \linkspell{Glyph of Warding}, \linkspell{Symbol of Death}{Symbol} (all of them), and \linkspell{Teleportation Circle} also create traps that a rogue can disarm with a successful Disable Device 
check. \linkspell{Spike Growth} and \linkspell{Spike Stones}, however, create magic traps 
against which Disable Device checks do not succeed. See the individual spell descriptions 
for details.

%%%
\subsubsection{Other Ways To Beat A Trap}
%%%

It's possible to ruin many traps without making a Disable Device check.

\textbf{Ranged Attack Traps:} Once a trap's location is known, the obvious way 
to ruin it is to smash the mechanism -- assuming the mechanism can be accessed. 
Failing that, it's possible to plug up the holes from which the projectiles emerge. 
Doing this prevents the trap from firing unless its ammunition does enough damage 
to break through the plugs.

\textbf{Melee Attack Traps:} These devices can be thwarted by smashing the mechanism 
or blocking the weapons, as noted above. Alternatively, if a character studies 
the trap as it triggers, he might be able to time his dodges just right to avoid 
damage. A character who is doing nothing but studying a trap when it first goes 
off gains a +4 dodge bonus against its attacks if it is triggered again within 
the next minute.

\textbf{Pits:} Disabling a pit trap generally ruins only the trapdoor, making it 
an uncovered pit. Filling in the pit or building a makeshift bridge across it is 
an application of manual labor, not the Disable Device skill. Characters could 
neutralize any spikes at the bottom of a pit by attacking them---they break just 
as daggers do.

\textbf{Magic Traps:} \linkspell{Dispel Magic} helps here. Someone who succeeds on 
a caster level check against the level of the trap's creator suppresses the trap 
for 1d4 rounds. This works only with a targeted \textit{dispel magic}, not the 
area version (see the spell description).

%%%%%%%%%%%%%%%%%%%%%%%%%
\skillentry{Disguise}{(Cha)}
%%%%%%%%%%%%%%%%%%%%%%%%%

\textbf{Check:} Your Disguise check result determines how good the disguise is, 
and it is opposed by others' \linkskill{Spot} check results. If you don't draw any attention 
to yourself, others do not get to make Spot checks. If you come to the attention 
of people who are suspicious (such as a guard who is watching commoners walking 
through a city gate), it can be assumed that such observers are taking 10 on their 
Spot checks.

You get only one Disguise check per use of the skill, even if several people are 
making Spot checks against it. The Disguise check is made secretly, so that you 
can't be sure how good the result is.

The effectiveness of your disguise depends in part on how much you're attempting 
to change your appearance.

\begin{table}[htb]
\rowcolors{1}{white}{offyellow}
\caption{Disguise Modifiers}
\centering
\begin{tabular}{l c}
\textbf{Disguise} & \textbf{Disguise Check Modifier} \\
Minor details only & +5\\
Disguised as different gender\textsuperscript{1} & -2\\
Disguised as different race\textsuperscript{1} & -2\\
Disguised as different age category\textsuperscript{1} & -2\textsuperscript{2}\\
\multicolumn{2}{l}{\textsuperscript{1}These modifiers are cumulative; use any that apply.}\\
\multicolumn{2}{p{9cm}}{\textsuperscript{2}Per step of difference between your actual age category and your disguised age category. The steps are: young (younger than adulthood), adulthood, middle age, old, and venerable.}\\
\end{tabular}
\end{table}

If you are impersonating a particular individual, those who know what that person 
looks like get a bonus on their Spot checks according to the table below. Furthermore, 
they are automatically considered to be suspicious of you, so opposed checks are 
always called for.

\begin{table}[htb]
\rowcolors{1}{white}{offyellow}
\caption{Impersonation Modifiers}
\centering
\begin{tabular}{l c}
\textbf{Familiarity} & \textbf{Viewer's Spot Check Bonus}\\
Recognizes on sight & +4\\
Friends or accociates & +6\\
Close friends & +8\\
Intimiate & +10\\
\end{tabular}
\end{table}

Usually, an individual makes a Spot check to see through your disguise immediately 
upon meeting you and each hour thereafter. If you casually meet many different 
creatures, each for a short time, check once per day or hour, using an average 
Spot modifier for the group. 

\textbf{Action:} Creating a disguise requires 1d3x10 minutes 
of work.

\textbf{Try Again:} Yes. You may try to redo a failed disguise, but once others 
know that a disguise was attempted, they'll be more suspicious.

\textbf{Special:} Magic that alters your form, such as \linkspell{Alter Self}, \linkspell{Disguise Self}, \linkspell{Polymorph}, or \linkspell{Shapechange}, grants you a +10 bonus on Disguise checks 
(see the individual spell descriptions). You must succeed on a Disguise check with 
a +10 bonus to duplicate the appearance of a specific individual using the \linkspell{Veil} spell. Divination magic that allows people to see through illusions (such as \linkspell{True Seeing}) does not penetrate a mundane disguise, but it can negate the magical component 
of a magically enhanced one.

You must make a Disguise check when you cast a \linkspell{Simulacrum} spell to determine 
how good the likeness is.

If you have the \linkfeat{Deceitful} feat, you get a +2 bonus on Disguise checks.

\textbf{Synergy:} If you have 5 or more ranks in \linkskill{Bluff}, you get a +2 bonus on Disguise 
checks when you know that you're being observed and you try to act in character.

%%%%%%%%%%%%%%%%%%%%%%%%%
\skillentry{Escape Artist}{(Dex; Armor Check Penalty)}
%%%%%%%%%%%%%%%%%%%%%%%%%

\textbf{Check:} The table below gives the DCs to escape various forms of restraints.

\textit{Ropes:} Your Escape Artist check is opposed by the binder's Use Rope check. 
Since it's easier to tie someone up than to escape from being tied up, the binder 
gets a +10 bonus on his or her check.

\textit{Manacles and Masterwork Manacles:} The DC for manacles is set by their 
construction.

\textit{Tight Space:} The DC noted on the table is for getting through a space 
where your head fits but your shoulders don't. If the space is long you may need 
to make multiple checks. You can't get through a space that your head does not 
fit through.

\textit{Grappler:} You can make an Escape Artist check opposed by your enemy's 
grapple check to get out of a grapple or out of a pinned condition (so that you're 
only grappling).

\begin{table}[htb]
\rowcolors{1}{white}{offyellow}
\caption{Escape Artist DCs}
\centering
\begin{tabular}{p{7cm} c}
\textbf{Restraint} & \textbf{Escape Artist DC}\\
Ropes & Binder's Use Rope check at +10\\
Net, \linkspell{Animate Rope} spell, \linkspell{Command Plants} spell, \linkspell{Control Plants} spell, or \linkspell{Entangle} spell & 20\\
\linkspell{Snare} spell & 23\\
Manacles & 30\\
Tight space & 30\\
Masterwork manacles & 35\\
Grappler & Grappler's grapple check result\\
\end{tabular}
\end{table}

\textbf{Action:} Making an Escape Artist check to escape from rope bindings, manacles, 
or other restraints (except a grappler) requires 1 minute of work. Escaping from 
a net or an \linkspell{Animate Rope}, \linkspell{Command Plants}, \linkspell{Control Plants}, or \linkspell{Entangle} spell is a full-round action. Escaping from a grapple or pin is a standard action. 
Squeezing through a tight space takes at least 1 minute, maybe longer, depending 
on how long the space is.

\textbf{Try Again:} Varies. You can make another check after a failed check if 
you're squeezing your way through a tight space, making multiple checks. If the 
situation permits, you can make additional checks, or even take 20, as long as 
you're not being actively opposed.

\textbf{Special:} If you have the \linkfeat{Agile} feat, you get a +2 bonus on Escape Artist 
checks.

\textbf{Synergy:} If you have 5 or more ranks in Escape Artist, you get a +2 bonus 
on \linkskill{Use Rope} checks to bind someone.

If you have 5 or more ranks in Use Rope, you get a +2 bonus on Escape Artist checks 
when escaping from rope bonds.

%%%%%%%%%%%%%%%%%%%%%%%%%
\skillentry{Forgery}{(Int)}
%%%%%%%%%%%%%%%%%%%%%%%%%

\textbf{Check:} Forgery requires writing materials appropriate to the document 
being forged, enough light or sufficient visual acuity to see the details of what 
you're writing, wax for seals (if appropriate), and some time. To forge a document 
on which the handwriting is not specific to a person (military orders, a government 
decree, a business ledger, or the like), you need only to have seen a similar document 
before, and you gain a +8 bonus on your check. To forge a signature, you need an 
autograph of that person to copy, and you gain a +4 bonus on the check. To forge 
a longer document written in the hand of some particular person, a large sample 
of that person's handwriting is needed.

The Forgery check is made secretly, so that you're not sure how good your forgery 
is. As with Disguise, you don't even need to make a check until someone examines 
the work. Your Forgery check is opposed by the Forgery check of the person who 
examines the document to check its authenticity. The examiner gains modifiers on 
his or her check if any of the conditions on the table below exist.

\begin{table}[htb]
\rowcolors{1}{white}{offyellow}
\caption{Forgery Situation Modifiers}
\centering
\begin{tabular}{l p{2.7cm}}
\textbf{Condition} & \textbf{Reader's Forgery Check Modifier}\\
Type of document unknown to reader & -2\\
Type of document somewhat known to reader & +0\\
Type of document well known to reader & +2\\
Handwriting not known to reader & -2\\
Handwriting somewhat known to reader & +0\\
Handwriting intimately known to reader & +2\\
Reader only casually reviews the document & -2\\
\end{tabular}
\end{table}

A document that contradicts procedure, orders, or previous knowledge, or one that 
requires sacrifice on the part of the person checking the document can increase 
that character's suspicion (and thus create favorable circumstances for the checker's 
opposing Forgery check).

\textbf{Action:} Forging a very short and simple document takes about 1 minute. 
A longer or more complex document takes 1d4 minutes per page.

\textbf{Try Again:} Usually, no. A retry is never possible after a particular reader 
detects a particular forgery. But the document created by the forger might still 
fool someone else. The result of a Forgery check for a particular document must 
be used for every instance of a different reader examining the document. No reader 
can attempt to detect a particular forgery more than once; if that one opposed 
check goes in favor of the forger, then the reader can't try using his own skill 
again, even if he's suspicious about the document.

\textbf{Special:} If you have the \linkfeat{Deceitful} feat, you get a +2 bonus on Forgery 
checks.

\textbf{Restriction:} Forgery is language-dependent; thus, to forge documents and 
detect forgeries, you must be able to read and write the language in question. 
A barbarian can't learn the Forgery skill unless he has learned to read and write.

%%%%%%%%%%%%%%%%%%%%%%%%%
\skillentry{Gather Information}{(Cha)}
%%%%%%%%%%%%%%%%%%%%%%%%%

\textbf{Check:} An evening's time, a few gold pieces for buying drinks and making 
friends, and a DC 10 Gather Information check get you a general idea of a city's 
major news items, assuming there are no obvious reasons why the information would 
be withheld. The higher your check result, the better the information.

If you want to find out about a specific rumor, or a specific item, or obtain a 
map, or do something else along those lines, the DC for the check is 15 to 25, 
or even higher.

\textbf{Action:} A typical Gather Information check takes 1d4+1 hours.

\textbf{Try Again:} Yes, but it takes time for each check. Furthermore, you may 
draw attention to yourself if you repeatedly pursue a certain type of information.

\textbf{Special:} A \linkrace{Half-Elf} has a +2 racial bonus on Gather Information checks.

If you have the \linkfeat{Investigator} feat, you get a +2 bonus on Gather Information checks.

\textbf{Synergy:} If you have 5 or more ranks in \linkskill{Knowledge} (local), you get a +2 
bonus on Gather Information checks.

%%%%%%%%%%%%%%%%%%%%%%%%%
\skillentry{Handle Animal}{(Cha; Trained Only)}
%%%%%%%%%%%%%%%%%%%%%%%%%

\textbf{Check:} The DC depends on what you are trying to do.

\begin{table}[htb]
\rowcolors{1}{white}{offyellow}
\caption{Handle Animal Tasks}
\centering
\begin{tabular}{l c}
\textbf{Task} & \textbf{Handle Animal DC}\\
Handle an animal & 10\\
"Push" and animal & 25\\
Teach an animal a trick & 15 or 20\textsuperscript{1}\\
Train an animal for a general purpose & 15 or 20\textsuperscript{1}\\
Rear a wild animal & 15 + HD of animal\\
\multicolumn{2}{l}{\textsuperscript{1}See the specific trick or purpose below.}\\
\end{tabular}
\end{table}

\begin{table}[htb]
\rowcolors{1}{white}{offyellow}
\caption{Handle Animal General Purposes}
\centering
\begin{tabular}{l c l c}
\textbf{General Purpose} & \textbf{DC} & \textbf{General Purpose} & \textbf{DC}\\
Combat Riding & 20 & Hunting & 20\\
Fighting & 20 & Performance & 15\\
Guarding & 20 & Riding & 15\\
Heavy Labor & 15 & &\\
\end{tabular}
\end{table}

\textit{Handle an Animal:} This task involves commanding an animal to perform a 
task or trick that it knows. If the animal is wounded or has taken any nonlethal 
damage or ability score damage, the DC increases by 2. If your check succeeds, 
the animal performs the task or trick on its next action.

\textit{"Push" an Animal:} To push an animal means to get it to perform a task 
or trick that it doesn't know but is physically capable of performing. This category 
also covers making an animal perform a forced march or forcing it to hustle for 
more than 1 hour between sleep cycles. If the animal is wounded or has taken any 
nonlethal damage or ability score damage, the DC increases by 2. If your check 
succeeds, the animal performs the task or trick on its next action.

\textit{Teach an Animal a Trick:} You can teach an animal a specific trick with 
one week of work and a successful Handle Animal check against the indicated DC. 
An animal with an Intelligence score of 1 can learn a maximum of three tricks, 
while an animal with an Intelligence score of 2 can learn a maximum of six tricks. 
Possible tricks (and their associated DCs) include, but are not necessarily limited 
to, the following.

Attack (DC 20): The animal attacks apparent enemies. You may point to a particular 
creature that you wish the animal to attack, and it will comply if able. Normally, 
an animal will attack only humanoids, monstrous humanoids, giants, or other animals. 
Teaching an animal to attack all creatures (including such unnatural creatures 
as undead and aberrations) counts as two tricks.

Come (DC 15): The animal comes to you, even if it normally would not do so.

Defend (DC 20): The animal defends you (or is ready to defend you if no threat 
is present), even without any command being given. Alternatively, you can command 
the animal to defend a specific other character.

Down (DC 15): The animal breaks off from combat or otherwise backs down. An animal 
that doesn't know this trick continues to fight until it must flee (due to injury, 
a fear effect, or the like) or its opponent is defeated.

Fetch (DC 15): The animal goes and gets something. If you do not point out a specific 
item, the animal fetches some random object.

Guard (DC 20): The animal stays in place and prevents others from approaching.

Heel (DC 15): The animal follows you closely, even to places where it normally 
wouldn't go.

Perform (DC 15): The animal performs a variety of simple tricks, such as sitting 
up, rolling over, roaring or barking, and so on.

Seek (DC 15): The animal moves into an area and looks around for anything that 
is obviously alive or animate.

Stay (DC 15): The animal stays in place, waiting for you to return. It does not 
challenge other creatures that come by,

though it still defends itself if it needs to.

Track (DC 20): The animal tracks the scent presented to it. (This requires the 
animal to have the scent ability)

Work (DC 15): The animal pulls or pushes a medium or heavy load.

\vspace{12pt}
\textit{Train an Animal for a Purpose:} Rather than teaching an animal individual 
tricks, you can simply train it for a general purpose. Essentially, an animal's 
purpose represents a preselected set of known tricks that fit into a common scheme, 
such as guarding or heavy labor. The animal must meet all the normal prerequisites 
for all tricks included in the training package. If the package includes more than 
three tricks, the animal must have an Intelligence score of 2.

An animal can be trained for only one general purpose, though if the creature is 
capable of learning additional tricks (above and beyond those included in its general 
purpose), it may do so. Training an animal for a purpose requires fewer checks 
than teaching individual tricks does, but no less time. 

Combat Riding (DC 20): An animal trained to bear a rider into combat knows the 
tricks attack, come, defend, down, guard, and heel. Training an animal for combat 
riding takes six weeks. You may also "upgrade" an animal trained for riding to 
one trained for combat riding by spending three weeks and making a successful DC 
20 Handle Animal check. The new general purpose and tricks completely replace the 
animal's previous purpose and any tricks it once knew. Warhorses and riding dogs 
are already trained to bear riders into combat, and they don't require any additional 
training for this purpose.

Fighting (DC 20): An animal trained to engage in combat knows the tricks attack, 
down, and stay. Training an animal for fighting takes three weeks.

Guarding (DC 20): An animal trained to guard knows the tricks attack, defend, down, 
and guard. Training an animal for guarding takes four weeks.

Heavy Labor (DC 15): An animal trained for heavy labor knows the tricks come and 
work. Training an animal for heavy labor takes two weeks.

Hunting (DC 20): An animal trained for hunting knows the tricks attack, down, fetch, 
heel, seek, and track. Training an animal for hunting takes six weeks.

Performance (DC 15): An animal trained for performance knows the tricks come, fetch, 
heel, perform, and stay. Training an animal for performance takes five weeks.

Riding (DC 15): An animal trained to bear a rider knows the tricks come, heel, 
and stay. Training an animal for riding takes three weeks.

\vspace{12pt}
\textit{Rear a Wild Animal:} To rear an animal means to raise a wild creature from 
infancy so that it becomes domesticated. A handler can rear as many as three creatures 
of the same kind at once.

A successfully domesticated animal can be taught tricks at the same time it's being 
raised, or it can be taught as a domesticated animal later.

\textbf{Action:} Varies. Handling an animal is a move action, while pushing an 
animal is a full-round action. (A druid or ranger can handle her animal companion 
as a free action or push it as a move action.) For tasks with specific time frames 
noted above, you must spend half this time (at the rate of 3 hours per day per 
animal being handled) working toward completion of the task before you attempt 
the Handle Animal check. If the check fails, your attempt to teach, rear, or train 
the animal fails and you need not complete the teaching, rearing, or training time. 
If the check succeeds, you must invest the remainder of the time to complete the 
teaching, rearing, or training. If the time is interrupted or the task is not followed 
through to completion, the attempt to teach, rear, or train the animal automatically 
fails.

\textbf{Try Again:} Yes, except for rearing an animal.

\textbf{Special:} You can use this skill on a creature with an Intelligence score 
of 1 or 2 that is not an animal, but the DC of any such check increases by 5. Such 
creatures have the same limit on tricks known as animals do.

A \linkclass{Druid} or \linkclass{Ranger} gains a +4 circumstance bonus on Handle Animal checks involving 
her animal companion.

In addition, a druid's or ranger's animal companion knows one or more bonus tricks, 
which don't count against the normal limit on tricks known and don't require any 
training time or Handle Animal checks to teach.

If you have the \linkfeat{Animal Affinity} feat, you get a +2 bonus on Handle Animal checks.

\textbf{Synergy:} If you have 5 or more ranks in Handle Animal, you get a +2 bonus 
on \linkskill{Ride} checks and wild empathy checks.

\textbf{Untrained:} If you have no ranks in Handle Animal, you can use a Charisma 
check to handle and push domestic animals, but you can't teach, rear, or train 
animals. A druid or ranger with no ranks in Handle Animal can use a Charisma check 
to handle and push her animal companion, but she can't teach, rear, or train other 
nondomestic animals.

%%%%%%%%%%%%%%%%%%%%%%%%%
\skillentry{Heal}{(Wis)}
%%%%%%%%%%%%%%%%%%%%%%%%%

\textbf{Check:} The DC and effect depend on the task you attempt.

\begin{table}[htb]
\rowcolors{1}{white}{offyellow}
\caption{Heal Tasks}
\centering
\begin{tabular}{p{4.5cm} c}
\textbf{Task} & \textbf{Heal DC} \\
First Aid & 15\\
Long-term care & 15\\
\raggedright{}Treat wound from caltrop, \linkspell{Spike Growth}, or \linkspell{Spike Stones} & 15\\
Treat Poison & Poison's save DC\\
Treat Disease & Disease's save DC\\
\end{tabular}
\end{table}

\textit{First Aid:} You usually use first aid to save a dying character. If a character 
has negative hit points and is losing hit points (at the rate of 1 per round, 1 
per hour, or 1 per day), you can make him or her stable. A stable character regains 
no hit points but stops losing them.

\textit{Long-Term Care:} Providing long-term care means treating a wounded person 
for a day or more. If your Heal check is successful, the patient recovers hit points 
or ability score points (lost to ability damage) at twice the normal rate: 2 hit 
points per level for a full 8 hours of rest in a day, or 4 hit points per level 
for each full day of complete rest; 2 ability score points for a full 8 hours of 
rest in a day, or 4 ability score points for each full day of complete rest.

You can tend as many as six patients at a time. You need a few items and supplies 
(bandages, salves, and so on) that are easy to come by in settled lands. Giving 
long-term care counts as light activity for the healer. You cannot give long-term 
care to yourself.

\textit{Treat Wound from Caltrop, Spike Growth, or Spike Stones:} A creature wounded 
by stepping on a caltrop moves at one-half normal speed. A successful Heal check 
removes this movement penalty.

A creature wounded by a \linkspell{Spike Growth} or \linkspell{Spike Stones} spell must 
succeed on a Reflex save or take injuries that reduce his speed by one-third. Another 
character can remove this penalty by taking 10 minutes to dress the victim's injuries 
and succeeding on a Heal check against the spell's save DC.

\textit{Treat Poison:} To treat poison means to tend a single character who has 
been poisoned and who is going to take more damage from the poison (or suffer some 
other effect). Every time the poisoned character makes a saving throw against the 
poison, you make a Heal check. The poisoned character uses your check result or 
his or her saving throw, whichever is higher.

\textit{Treat Disease:} To treat a disease means to tend a single diseased character. 
Every time he or she makes a saving throw against disease effects, you make a Heal 
check. The diseased character uses your check result or his or her saving throw, 
whichever is higher.

\textbf{Action:} Providing first aid, treating a wound, or treating poison is a 
standard action. Treating a disease or tending a creature wounded by a \textit{Spike 
Growth} or \textit{Spike Stones} spell takes 10 minutes of work. Providing long-term 
care requires 8 hours of light activity.

\textbf{Try Again:} Varies. Generally speaking, you can't try a Heal check again 
without proof of the original check's failure. You can always retry a check to 
provide first aid, assuming the target of the previous attempt is still alive.

\textbf{Special:} A character with the \linkfeat{Self-Sufficient} feat gets a +2 bonus on 
Heal checks.

A healer's kit gives you a +2 circumstance bonus on Heal checks.

%%%%%%%%%%%%%%%%%%%%%%%%%
\skillentry{Hide}{(Dex; Armor Check Penalty)}
%%%%%%%%%%%%%%%%%%%%%%%%%

\textbf{Check:} Your Hide check is opposed by the \linkskill{Spot} check of anyone who might 
see you. You can move up to one-half your normal speed and hide at no penalty. 
When moving at a speed greater than one-half but less than your normal speed, you 
take a -5 penalty. It's practically impossible (-20 penalty) to hide while attacking, 
running, or charging.

A creature larger or smaller than Medium takes a size bonus or penalty on Hide 
checks depending on its size category: Fine +16, Diminutive +12, Tiny +8, Small 
+4, Large -4, Huge -8, Gargantuan -12, Colossal -16.

You need cover or concealment in order to attempt a Hide check. Total cover or 
total concealment usually (but not always; see Special, below) obviates the need 
for a Hide check, since nothing can see you anyway.

If people are observing you, even casually, you can't hide. You can run around 
a corner or behind cover so that you're out of sight and then hide, but the others 
then know at least where you went.

If your observers are momentarily distracted (such as by a \linkskill{Bluff} check; see below), 
though, you can attempt to hide. While the others turn their attention from you, 
you can attempt a Hide check if you can get to a hiding place of some kind. (As 
a general guideline, the hiding place has to be within 1 foot per rank you have 
in Hide.) This check, however, is made at a -10 penalty because you have to move 
fast.

\textit{Sniping:} If you've already successfully hidden at least 10 feet from your 
target, you can make one ranged attack, then immediately hide again. You take a 
-20 penalty on your Hide check to conceal yourself after the shot.

\textit{Creating a Diversion to Hide:} You can use Bluff to help you hide. A successful 
Bluff check can give you the momentary diversion you need to attempt a Hide check 
while people are aware of you.

\textbf{Action:} Usually none. Normally, you make a Hide check as part of movement, 
so it doesn't take a separate action. However, hiding immediately after a ranged 
attack (see Sniping, above) is a move action.

\textbf{Special:} If you are invisible, you gain a +40 bonus on Hide checks if 
you are immobile, or a +20 bonus on Hide checks if you're moving.

If you have the \linkfeat{Stealthy} feat, you get a +2 bonus on Hide checks.

A 13th-level \linkclass{Ranger} can attempt a Hide check in any sort of natural terrain, even 
if it doesn't grant cover or concealment. A 17th-level ranger can do this even while 
being observed.

%%%%%%%%%%%%%%%%%%%%%%%%%
\skillentry{Intimidate}{(Cha)}
%%%%%%%%%%%%%%%%%%%%%%%%%

\textbf{Check:} You can change another's behavior with a successful check. Your 
Intimidate check is opposed by the target's modified level check (1d20 + character 
level or Hit Dice + target's Wisdom bonus [if any] + target's modifiers on saves 
against fear). If you beat your target's check result, you may treat the target 
as friendly, but only for the purpose of actions taken while it remains intimidated. 
(That is, the target retains its normal attitude, but will chat, advise, offer 
limited help, or advocate on your behalf while intimidated. See the Diplomacy skill, 
above, for additional details.) The effect lasts as long as the target remains 
in your presence, and for 1d6x10 minutes afterward. After this 
time, the target's default attitude toward you shifts to unfriendly (or, if normally 
unfriendly, to hostile).

If you fail the check by 5 or more, the target provides you with incorrect or useless 
information, or otherwise frustrates your efforts.

\textit{Demoralize Opponent:} You can also use Intimidate to weaken an opponent's 
resolve in combat. To do so, make an Intimidate check opposed by the target's modified 
level check (see above). If you win, the target becomes shaken for 1 round. A shaken 
character takes a -2 penalty on attack rolls, ability checks, and saving throws. 
You can intimidate only an opponent that you threaten in melee combat and that 
can see you.

\textbf{Action:} Varies. Changing another's behavior requires 1 minute of interaction. 
Intimidating an opponent in combat is a standard action.

\textbf{Try Again:} Optional, but not recommended because retries usually do not 
work. Even if the initial check succeeds, the other character can be intimidated 
only so far, and a retry doesn't help. If the initial check fails, the other character 
has probably become more firmly resolved to resist the intimidator, and a retry 
is futile.

\textbf{Special:} You gain a +4 bonus on your Intimidate check for every size category 
that you are larger than your target. Conversely, you take a -4 penalty on your 
Intimidate check for every size category that you are smaller than your target.

A character immune to fear can't be intimidated, nor can nonintelligent creatures.

If you have the \linkfeat{Persuasive} feat, you get a +2 bonus on Intimidate checks.

\textbf{Synergy:} If you have 5 or more ranks in \linkskill{Bluff}, you get a +2 bonus on Intimidate 
checks.

%%%%%%%%%%%%%%%%%%%%%%%%%
\skillentry{Jump}{(Str; Armor Check Penalty)}
%%%%%%%%%%%%%%%%%%%%%%%%%

\textbf{Check:} The DC and the distance you can cover vary according to the type 
of jump you are attempting (see below).

Your Jump check is modified by your speed. If your speed is 30 feet then no modifier 
based on speed applies to the check. If your speed is less than 30 feet, you take 
a -6 penalty for every 10 feet of speed less than 30 feet. If your speed is greater 
than 30 feet, you gain a +4 bonus for every 10 feet beyond 30 feet.

All Jump DCs given here assume that you get a running start, which requires that 
you move at least 20 feet in a straight line before attempting the jump. If you 
do not get a running start, the DC for the jump is doubled.

Distance moved by jumping is counted against your normal maximum movement in a 
round.

If you have ranks in Jump and you succeed on a Jump check, you land on your feet 
(when appropriate). If you attempt a Jump check untrained, you land prone unless 
you beat the DC by 5 or more.

\textit{Long Jump:} A long jump is a horizontal jump, made across a gap like a 
chasm or stream. At the midpoint of the jump, you attain a vertical height equal 
to one-quarter of the horizontal distance. The DC for the jump is equal to the 
distance jumped (in feet).

If your check succeeds, you land on your feet at the far end. If you fail the check 
by less than 5, you don't clear the distance, but you can make a DC 15 Reflex save 
to grab the far edge of the gap. You end your movement grasping the far edge. If 
that leaves you dangling over a chasm or gap, getting up requires a move action 
and a DC 15 Climb check.

\textit{High Jump:} A high jump is a vertical leap made to reach a ledge high above 
or to grasp something overhead. The DC is equal to 4 times the distance to be cleared.

If you jumped up to grab something, a successful check indicates that you reached 
the desired height. If you wish to pull yourself up, you can do so with a move 
action and a DC 15 Climb check. If you fail the Jump check, you do not reach the 
height, and you land on your feet in the same spot from which you jumped. As with 
a long jump, the DC is doubled if you do not get a running start of at least 20 
feet.


Obviously, the difficulty of reaching a given height varies according to the size 
of the character or creature. The maximum vertical reach (height the creature can 
reach without jumping) for an average creature of a given size is shown on the 
table below. (As a Medium creature, a typical human can reach 8 feet without jumping.)

Quadrupedal creatures don't have the same vertical reach as a bipedal creature; 
treat them as being one size category smaller.

\begin{table}[htb]
\rowcolors{1}{white}{offyellow}
\caption{Vertical Reach By Size}
\centering
\begin{tabular}{l c}
\textbf{Size} & \textbf{Vertical Reach} \\
Colossal & 128ft\\
Gargantuan & 64ft\\
Huge & 32ft\\
Large & 16ft\\
Medium & 8ft\\
Small & 4ft\\
Tiny & 2ft\\
Diminutive & 1ft\\
Fine & \sfrac{1}{2}ft\\
\end{tabular}
\end{table}

\textit{Hop Up:} You can jump up onto an object as tall as your waist, such as 
a table or small boulder, with a DC 10 Jump check. Doing so counts as 10 feet of 
movement, so if your speed is 30 feet, you could move 20 feet, then hop up onto 
a counter. You do not need to get a running start to hop up, so the DC is not doubled 
if you do not get a running start.

\textit{Jumping Down:} If you intentionally jump from a height, you take less damage 
than you would if you just fell. The DC to jump down from a height is 15. You do 
not have to get a running start to jump down, so the DC is not doubled if you do 
not get a running start.

If you succeed on the check, you take falling damage as if you had dropped 10 fewer 
feet than you actually did.

\textbf{Action:} None. A Jump check is included in your movement, so it is part 
of a move action. If you run out of movement mid-jump, your next action (either 
on this turn or, if necessary, on your next turn) must be a move action to complete 
the jump.

\textbf{Special:} Effects that increase your movement also increase your jumping 
distance, since your check is modified by your speed.

If you have the \linkfeat{Run} feat, you get a +4 bonus on Jump checks for any jumps made 
after a running start.

A \linkrace{Halfling} has a +2 racial bonus on Jump checks because halflings are agile and 
athletic.

If you have the \linkfeat{Acrobatic} feat, you get a +2 bonus on Jump checks.

\textbf{Synergy:} If you have 5 or more ranks in \linkskill{Tumble}, you get a +2 bonus on 
Jump checks.

If you have 5 or more ranks in Jump, you get a +2 bonus on Tumble checks.

%%%%%%%%%%%%%%%%%%%%%%%%%
\skillentry{Knowledge}{(Int; Trained Only)}
%%%%%%%%%%%%%%%%%%%%%%%%%

Like the Craft and Profession skills, Knowledge actually encompasses a number of 
unrelated skills. Knowledge represents a study of some body of lore, possibly an 
academic or even scientific discipline.

Below are listed typical fields of study.

\begin{itemize*}
\item Arcana (ancient mysteries, magic traditions, arcane symbols, cryptic phrases, constructs, dragons, magical beasts)
\item Architecture and engineering (buildings, aqueducts, bridges, fortifications)
\item Dungeoneering (aberrations, caverns, oozes, spelunking)
\item Geography (lands, terrain, climate, people)
\item History (royalty, wars, colonies, migrations, founding of cities)
\item Local (legends, personalities, inhabitants, laws, customs, traditions, humanoids)
\item Nature (animals, fey, giants, monstrous humanoids, plants, seasons and cycles, weather, vermin)
\item Nobility and royalty (lineages, heraldry, family trees, mottoes, personalities)
\item Religion (gods and goddesses, mythic history, ecclesiastic tradition, holy symbols, undead)
\end{itemize*}

The planes (the Inner Planes, the Outer Planes, the Astral Plane, the Ethereal 
Plane, outsiders, elementals, magic related to the planes)

\textbf{Check:} Answering a question within your field of study has a DC of 10 
(for really easy questions), 15 (for basic questions), or 20 to 30 (for really 
tough questions).

In many cases, you can use this skill to identify monsters and their special powers 
or vulnerabilities. In general, the DC of such a check equals 10 + the monster's 
HD. A successful check allows you to remember a bit of useful information about 
that monster.

For every 5 points by which your check result exceeds the DC, you recall another 
piece of useful information.

\textbf{Action:} Usually none. In most cases, making a Knowledge check doesn't 
take an action -- you simply know the answer or you don't.

\textbf{Try Again:} No. The check represents what you know, and thinking about 
a topic a second time doesn't let you know something that you never learned in 
the first place.

\textbf{Synergy:} If you have 5 or more ranks in Knowledge (arcana), you get a 
+2 bonus on \linkskill{Spellcraft} checks.

If you have 5 or more ranks in Knowledge (architecture and engineering), you get 
a +2 bonus on \linkskill{Search} checks made to find secret doors or hidden compartments.

If you have 5 or more ranks in Knowledge (geography), you get a +2 bonus on \linkskill{Survival} 
checks made to keep from getting lost or to avoid natural hazards.

If you have 5 or more ranks in Knowledge (history), you get a +2 bonus on bardic 
knowledge checks.

If you have 5 or more ranks in Knowledge (local), you get a +2 bonus on \linkskill{Gather Information} checks.

If you have 5 or more ranks in Knowledge (nature), you get a +2 bonus on \linkskill{Survival} 
checks made in aboveground natural environments (aquatic, desert, forest, hill, 
marsh, mountains, or plains).

If you have 5 or more ranks in Knowledge (nobility and royalty), you get a +2 bonus 
on \linkskill{Diplomacy} checks.

If you have 5 or more ranks in Knowledge (religion), you get a +2 bonus on turning 
checks against undead.

If you have 5 or more ranks in Knowledge (the planes), you get a +2 bonus on \linkskill{Survival} 
checks made while on other planes.

If you have 5 or more ranks in Knowledge (dungeoneering), you get a +2 bonus on 
\linkskill{Survival} checks made while underground.

If you have 5 or more ranks in \linkskill{Survival}, you get a +2 bonus on Knowledge (nature) 
checks.

\textbf{Untrained:} An untrained Knowledge check is simply an Intelligence check. 
Without actual training, you know only common knowledge (DC 10 or lower).

%%%%%%%%%%%%%%%%%%%%%%%%%
\skillentry{Listen}{(Wis)}
%%%%%%%%%%%%%%%%%%%%%%%%%

\textbf{Check:} Your Listen check is either made against a DC that reflects how 
quiet the noise is that you might hear, or it is opposed by your target's Move 
Silently check.

\begin{table}[htb]
\rowcolors{1}{white}{offyellow}
\caption{Example Listen DCs}
\centering
\begin{tabular}{c l}
\textbf{Listen DC} & \textbf{Sound}\\
-10 & A battle\\
0 & People talking\textsuperscript{1}\\
5 & A person in medium armor walking at a slow pace (10 ft./round) trying not to make any noise\\
10 & An unarmored person walking at a slow pace (15 ft./round) trying not to make any noise\\
15 & A 1st-level rogue using Move Silently to sneak past the listener\\
15 & People whispering\textsuperscript{1}\\
19 & A cat stalking\\
30 & An owl gliding in for a kill\\
\multicolumn{2}{p{11cm}}{\textsuperscript{1} If you beat the DC by 10 or more, you can make out what's being said, assuming that you understand the language.}\\
\end{tabular}
\end{table}

\begin{table}[htb]
\rowcolors{1}{white}{offyellow}
\caption{Listen DC Modifiers}
\centering
\begin{tabular}{c l}
\textbf{Listen DC Modifier} & \textbf{Condition}\\
+5 & Through a door\\
+15 & Through a stone wall\\
+1 & Per 10ft of distance\\
+5 & Listener distracted\\
\end{tabular}
\end{table}

In the case of people trying to be quiet, the DCs given on the table could be replaced 
by Move Silently checks, in which case the indicated DC would be their average 
check result. 

\textbf{Action:} Varies. Every time you have a chance to hear something in a reactive 
manner (such as when someone makes a noise or you move into a new area), you can 
make a Listen check without using an action. Trying to hear something you failed 
to hear previously is a move action.

\textbf{Try Again:} Yes. You can try to hear something that you failed to hear 
previously with no penalty.

\textbf{Special:} When several characters are listening to the same thing, a single 
1d20 roll can be used for all the individuals' Listen checks.

A \linksec{Fascinated} creature takes a -4 penalty on Listen checks made as reactions.

If you have the \linkfeat{Alertness} feat, you get a +2 bonus on Listen checks.

A \linkclass{Ranger} gains a bonus on Listen checks when using this skill against a favored 
enemy.

An \linkrace{Elf}, \linkrace{Gnome}, or \linkrace{Halfling} has a +2 racial bonus on Listen checks. 

A \linkrace{Half-Elf} has a +1 racial bonus on Listen checks..

A sleeping character may make Listen checks at a -10 penalty. A successful check 
awakens the sleeper.

%%%%%%%%%%%%%%%%%%%%%%%%%
\skillentry{Move Silently}{(Dex; Armor Check Penalty)}
%%%%%%%%%%%%%%%%%%%%%%%%%

\textbf{Check:} Your Move Silently check is opposed by the Listen check of anyone 
who might hear you. You can move up to one-half your normal speed at no penalty. 
When moving at a speed greater than one-half but less than your full speed, you 
take a -5 penalty. It's practically impossible (-20 penalty) to move silently while 
running or charging.

Noisy surfaces, such as bogs or undergrowth, are tough to move silently across. 
When you try to sneak across such a surface, you take a penalty on your Move Silently 
check as indicated below.

\begin{table}[htb]
\rowcolors{1}{white}{offyellow}
\caption{Move Silently Surface Modifiers}
\centering
\begin{tabular}{lc}
\textbf{Surface} & \textbf{Check Modifier}\\
Noisy (scree, shallow or deep bog, undergrowth, dense rubble) & -2 \\
Very noisy (dense undergrowth, deep snow) & -5 \\
\end{tabular}
\end{table}

\textbf{Action:}None. A Move Silently check is included in your movement or other 
activity, so it is part of another action.

\textbf{Special:} The master of a cat familiar gains a +3 bonus on Move Silently 
checks.

A halfling has a +2 racial bonus on Move Silently checks.

If you have the \linkfeat{Stealthy} feat, you get a +2 bonus on Move Silently checks.

%%%%%%%%%%%%%%%%%%%%%%%%%
\skillentry{Open Lock}{(Dex; Trained Only)}
%%%%%%%%%%%%%%%%%%%%%%%%%

Attempting an Open Lock check without a set of thieves' tools imposes a -2 circumstance 
penalty on the check, even if a simple tool is employed. If you use masterwork 
thieves' tools, you gain a +2 circumstance bonus on the check.

\textbf{Check:} The DC for opening a lock varies from 20 to 40, depending on the 
quality of the lock, as given on the table below.

\begin{table}[htb]
\rowcolors{1}{white}{offyellow}
\caption{Open Lock DCs}
\centering
\begin{tabular}{l c l c}
\textbf{Lock} & \textbf{DC} & \textbf{Lock} & \textbf{DC}\\
Very Simple Lock & 20 & Good Lock & 30\\
Average Lock & 25 & Amazing Lock & 40\\
\end{tabular}
\end{table}

\textbf{Action:} Opening a lock is a full-round action.

\textbf{Special:} If you have the Nimble Fingers feat, you get a +2 bonus on Open 
Lock checks.

\textbf{Untrained:} You cannot pick locks untrained, but you might successfully 
force them open.

%%%%%%%%%%%%%%%%%%%%%%%%%
\skillentry{Perform}{(Cha)}
%%%%%%%%%%%%%%%%%%%%%%%%%

Like Craft, Knowledge, and Profession, Perform is actually a number of separate 
skills.

You could have several Perform skills, each with its own ranks, each purchased 
as a separate skill.

Each of the nine categories of the Perform skill includes a variety of methods, 
instruments, or techniques, a small list of which is provided for each category 
below.

\begin{itemize*}
\item Act (comedy, drama, mime)
\item Comedy (buffoonery, limericks, joke-telling)
\item Dance (ballet, waltz, jig)
\item Keyboard instruments (harpsichord, piano, pipe organ)
\item Oratory (epic, ode, storytelling)
\item Percussion instruments (bells, chimes, drums, gong)
\item String instruments (fiddle, harp, lute, mandolin)
\item Wind instruments (flute, pan pipes, recorder, shawm, trumpet)
\item Sing (ballad, chant, melody)
\end{itemize*}

\textbf{Check:} You can impress audiences with your talent and skill.

\begin{table}[htb]
\rowcolors{1}{white}{offyellow}
\caption{Performance Results}
\centering
\begin{tabular}{c p{12cm}}
\textbf{Perform DC} & \textbf{Performance}\\
10 & Routine performance. Trying to earn money by playing in public is essentially begging. You can earn 1d10 cp/day.\\
15 & Enjoyable performance. In a prosperous city, you can earn 1d10 sp/day.\\
20 & Great performance. In a prosperous city, you can earn 3d10 sp/day. In time, you may be invited to join a professional troupe and may develop a regional reputation.\\
25 & Memorable performance. In a prosperous city, you can earn 1d6 gp/day. In time, you may come to the attention of noble patrons and develop a national reputation.\\
30 & Extraordinary performance. In a prosperous city, you can earn 3d6 gp/day. In time, you may draw attention from distant potential patrons, or even from extraplanar beings.\\
\end{tabular}
\end{table}

A masterwork musical instrument gives you a +2 circumstance bonus on Perform checks 
that involve its use.

\textbf{Action:} Varies. Trying to earn money by playing in public requires anywhere 
from an evening's work to a full day's performance. The bard's special Perform-based 
abilities are described in that class's description.

\textbf{Try Again:} Yes. Retries are allowed, but they don't negate previous failures, 
and an audience that has been unimpressed in the past is likely to be prejudiced 
against future performances. (Increase the DC by 2 for each previous failure.)

\textbf{Special:} A bard must have at least 3 ranks in a Perform skill to inspire 
courage in his allies, or to use his countersong or his \textit{fascinate }ability. 
A bard needs 6 ranks in a Perform skill to inspire competence, 9 ranks to use his 
\textit{suggestion }ability, 12 ranks to inspire greatness, 15 ranks to use his 
\textit{song of freedom }ability, 18 ranks to inspire heroics, and 21 ranks to 
use his \textit{mass suggestion }ability. See Bardic Music in the bard class description.

In addition to using the Perform skill, you can entertain people with \linkskill{Sleight of Hand}, \linkskill{Tumble}{Tumbling}, \linkskill{Balance}{Tightrope walking}, and spells (especially illusions).

%%%%%%%%%%%%%%%%%%%%%%%%%
\skillentry{Profession}{(Wis; Trained Only)}
%%%%%%%%%%%%%%%%%%%%%%%%%

Like \linkskill{Craft}, \linkskill{Knowledge}, and \linkskill{Perform}, Profession is actually a number of separate 
skills. You could have several Profession skills, each with its own ranks, each 
purchased as a separate skill. While a Craft skill represents ability in creating 
or making an item, a Profession skill represents an aptitude in a vocation requiring 
a broader range of less specific knowledge. 

\textbf{Check:} You can practice your trade and make a decent living, earning about 
half your Profession check result in gold pieces per week of dedicated work. You 
know how to use the tools of your trade, how to perform the profession's daily 
tasks, how to supervise helpers, and how to handle common problems.

\textbf{Action:} Not applicable. A single check generally represents a week of 
work.

\textbf{Try Again:} Varies. An attempt to use a Profession skill to earn an income 
cannot be retried. You are stuck with whatever weekly wage your check result brought 
you. Another check may be made after a week to determine a new income for the next 
period of time. An attempt to accomplish some specific task can usually be retried.

\textbf{Untrained:} Untrained laborers and assistants (that is, characters without 
any ranks in Profession) earn an average of 1 silver piece per day.

%%%%%%%%%%%%%%%%%%%%%%%%%
\skillentry{Ride}{(Dex)}
%%%%%%%%%%%%%%%%%%%%%%%%%

If you attempt to ride a creature that is ill suited as a mount, you take a -5 
penalty on your Ride checks.

\textbf{Check:} Typical riding actions don't require checks. You can saddle, mount, 
ride, and dismount from a mount without a problem.

The following tasks do require checks.

\begin{table}[htb]
\rowcolors{1}{white}{offyellow}
\caption{Riding DCs}
\centering
\begin{tabular}{l c l c}
\textbf{Task} & \textbf{DC} & \textbf{Task} & \textbf{DC}\\
Guide with knees & 5 & Leap & 15\\
Stay in saddle & 5 & Spur mount & 15\\
Fight with warhorse & 10 & Fast mount or dismount & 20\textsuperscript{1}\\
Soft fall & 15 & & \\
\multicolumn{4}{l}{\textsuperscript{1} Armor check penalty applies.}\\
\end{tabular}
\end{table}

\textit{Guide with Knees:} You can react instantly to guide your mount with your 
knees so that you can use both hands in combat. Make your Ride check at the start 
of your turn. If you fail, you can use only one hand this round because you need 
to use the other to control your mount.

\textit{Stay in Saddle:} You can react instantly to try to avoid falling when your 
mount rears or bolts unexpectedly or when you take damage. This usage does not 
take an action.

\textit{Fight with Warhorse:} If you direct your war-trained mount to attack in 
battle, you can still make your own attack or attacks normally. This usage is a 
free action.

\textit{Cover:} You can react instantly to drop down and hang alongside your mount, 
using it as cover. You can't attack or cast spells while using your mount as cover. 
If you fail your Ride check, you don't get the cover benefit. This usage does not 
take an action.

\textit{Soft Fall:} You can react instantly to try to take no damage when you fall 
off a mount---when it is killed or when it falls, for example. If you fail your 
Ride check, you take 1d6 points of falling damage. This usage does not take an 
action.

\textit{Leap:} You can get your mount to leap obstacles as part of its movement. 
Use your Ride modifier or the mount's Jump modifier, whichever is lower, to see 
how far the creature can jump. If you fail your Ride check, you fall off the mount 
when it leaps and take the appropriate falling damage (at least 1d6 points). This 
usage does not take an action, but is part of the mount's movement.

\textit{Spur Mount:} You can spur your mount to greater speed with a move action. 
A successful Ride check increases the mount's speed by 10 feet for 1 round but 
deals 1 point of damage to the creature. You can use this ability every round, 
but each consecutive round of additional speed deals twice as much damage to the 
mount as the previous round (2 points, 4 points, 8 points, and so on).

\textit{Control Mount in Battle:} As a move action, you can attempt to control 
a light horse, pony, heavy horse, or other mount not trained for combat riding 
while in battle. If you fail the Ride check, you can do nothing else in that round. 
You do not need to roll for warhorses or warponies.

\textit{Fast Mount or Dismount:} You can attempt to mount or dismount from a mount 
of up to one size category larger than yourself as a free action, provided that 
you still have a move action available that round. If you fail the Ride check, 
mounting or dismounting is a move action. You can't use fast mount or dismount 
on a mount more than one size category larger than yourself.

\textbf{Action:} Varies. Mounting or dismounting normally is a move action. Other 
checks are a move action, a free action, or no action at all, as noted above.

\textbf{Special:} If you are riding bareback, you take a -5 penalty on Ride checks.

If your mount has a military saddle you get a +2 circumstance bonus on Ride checks 
related to staying in the saddle.

The Ride skill is a prerequisite for the feats \linkfeat{Mounted Archery}, \linkfeat{Mounted Combat}, 
\linkfeat{Ride-by Attack}, \linkfeat{Spirited Charge}, and \linkfeat{Trample}.

If you have the \linkfeat{Animal Affinity} feat, you get a +2 bonus on Ride checks.

\textbf{Synergy:} If you have 5 or more ranks in \linkskill{Handle Animal}, you get a +2 bonus 
on Ride checks.

%%%%%%%%%%%%%%%%%%%%%%%%%
\skillentry{Search}{(Int)}
%%%%%%%%%%%%%%%%%%%%%%%%%

\textbf{Check:} You generally must be within 10 feet of the object or surface to 
be searched. The table below gives DCs for typical tasks involving the Search skill.

\begin{table}[htb]
\rowcolors{1}{white}{offyellow}
\caption{Search DCs}
\centering
\begin{tabular}{l l}
\textbf{Task} & \textbf{Search DC}\\
Ransack a chest full of junk to find a certain item & 10\\
Notice a typical secret door or a simple trap & 20\\
Find a difficult nonmagical trap (rogue only)\textsuperscript{1} & 21 or higher\\
Find a magic trap (rogue only)\textsuperscript{1} & 25 + level of spell used to create trap\\
Notice a well-hidden secret door & 30\\
Find a footprint & Varies\textsuperscript{2}\\
\multicolumn{2}{l}{\textsuperscript{1}Dwarves (even if they are not rogues) can use Search to find traps built into or out of stone.}\\
\multicolumn{2}{p{14cm}}{\textsuperscript{2}A successful Search check can find a footprint or similar sign of a creature's passage, but it won't let you find or follow a trail. See the Track feat for the appropriate DC.}\\
\end{tabular}
\end{table}

\textbf{Action:} It takes a full-round action to search a 5-foot-by-5-foot area 
or a volume of goods 5 feet on a side.

\textbf{Special:} An \linkrace{Elf} has a +2 racial bonus on Search checks, and a \linkrace{Half-Elf} 
has a +1 racial bonus. An elf (but not a half-elf) who simply passes within 5 feet 
of a secret or concealed door can make a Search check to find that door.

If you have the \linkfeat{Investigator} feat, you get a +2 bonus on Search checks.

The spells \linkspell{Explosive Runes}, \linkspell{Fire Trap}, \linkspell{Glyph of Warding}, \linkspell{Symbol of Death}{Symbol} (all of them), and \linkspell{Teleportation Circle} create magic traps that a rogue can find by making a successful Search 
check and then can attempt to disarm by using Disable Device. Identifying the location 
of a \linkspell{Snare} spell has a DC of 23. \linkspell{Spike Growth} and \linkspell{Spike Stones} create magic traps that can be found using Search, but against which \linkskill{Disable Device} checks do not succeed. See the individual spell descriptions for details.

Active abjuration spells within 10 feet of each other for 24 hours or more create 
barely visible energy fluctuations. These fluctuations give you a +4 bonus on Search 
checks to locate such abjuration spells.

\textbf{Synergy:} If you have 5 or more ranks in Search, you get a +2 bonus on 
\linkskill{Survival} checks to find or follow tracks.

If you have 5 or more ranks in \linkskill{Knowledge} (architecture and engineering), you get 
a +2 bonus on Search checks to find secret doors or hidden compartments.

\textbf{Restriction:} While anyone can use Search to find a trap whose DC is 20 
or lower, only a \linkclass{Rogue} can use Search to locate traps with higher DCs. (\textit{Exception:} The spell \linkspell{Find Traps} temporarily enables a \linkclass{Cleric} to use the Search skill 
as if he were a rogue.)

A \linkrace{Dwarf}, even one who is not a rogue, can use the Search skill to find a difficult 
trap (one with a DC higher than 20) if the trap is built into or out of stone. 
He gains a +2 racial bonus on the Search check from his stonecunning ability.

%%%%%%%%%%%%%%%%%%%%%%%%%
\skillentry{Sense Motive}{(Wis)}
%%%%%%%%%%%%%%%%%%%%%%%%%

\textbf{Check:} A successful check lets you avoid being bluffed (see the \linkskill{Bluff} 
skill). You can also use this skill to determine when "something is up" (that 
is, something odd is going on) or to assess someone's trustworthiness. 

\begin{table}[htb]
\rowcolors{1}{white}{offyellow}
\caption{Sense Motive DCs}
\centering
\begin{tabular}{c l}
\textbf{Task} & \textbf{Sense Motive DC}\\
Hunch & 20 \\
Sense enchantment & 25 or 15\\
Discern secret message & Varies\\
\end{tabular}
\end{table}

\textit{Hunch:} This use of the skill involves making a gut assessment of the social 
situation. You can get the feeling from another's behavior that something is wrong, 
such as when you're talking to an impostor. Alternatively, you can get the feeling 
that someone is trustworthy.

\textit{Sense Enchantment:} You can tell that someone's behavior is being influenced 
by an enchantment effect (by definition, a mind-affecting effect), even 
if that person isn't aware of it. The usual DC is 25, but if the target is dominated 
(see \linkspell{Dominate Person}), the DC is only 15 because of the limited range 
of the target's activities.

\textit{Discern Secret Message:} You may use \linkskill{Sense Motive} to detect that a hidden 
message is being transmitted via the Bluff skill. In this case, your Sense Motive 
check is opposed by the Bluff check of the character transmitting the message. 
For each piece of information relating to the message that you are missing, you 
take a -2 penalty on your Sense Motive check. If you succeed by 4 or less, you 
know that something hidden is being communicated, but you can't learn anything 
specific about its content. If you beat the DC by 5 or more, you intercept and 
understand the message. If you fail by 4 or less, you don't detect any hidden communication. 
If you fail by 5 or more, you infer some false information.

\textbf{Action:} Trying to gain information with Sense Motive generally takes at 
least 1 minute, and you could spend a whole evening trying to get a sense of the 
people around you.

\textbf{Try Again:} No, though you may make a Sense Motive check for each Bluff 
check made against you.

\textbf{Special:} A ranger gains a bonus on Sense Motive checks when using this 
skill against a favored enemy.

If you have the \linkfeat{Negotiator} feat, you get a +2 bonus on Sense Motive checks.

\textbf{Synergy:} If you have 5 or more ranks in Sense Motive, you get a +2 bonus 
on Diplomacy checks.

%%%%%%%%%%%%%%%%%%%%%%%%%
\skillentry{Sleight of Hand}{(Dex; Trained Only; Armor Check Penalty)}\label{skill:Sleight Of Hand}
%%%%%%%%%%%%%%%%%%%%%%%%%

\textbf{Check:} A DC 10 Sleight of Hand check lets you palm a coin-sized, unattended 
object. Performing a minor feat of legerdemain, such as making a coin disappear, 
also has a DC of 10 unless an observer is determined to note where the item went.

When you use this skill under close observation, your skill check is opposed by 
the observer's Spot check. The observer's success doesn't prevent you from performing 
the action, just from doing it unnoticed.

You can hide a small object (including a light weapon or an easily concealed ranged 
weapon, such as a dart, sling, or hand crossbow) on your body. Your Sleight of 
Hand check is opposed by the Spot check of anyone observing you or the Search check 
of anyone frisking you. In the latter case, the searcher gains a +4 bonus on the 
Search check, since it's generally easier to find such an object than to hide it. 
A dagger is easier to hide than most light weapons, and grants you a +2 bonus on 
your Sleight of Hand check to conceal it. An extraordinarily small object, such 
as a coin, shuriken, or ring, grants you a +4 bonus on your Sleight of Hand check 
to conceal it, and heavy or baggy clothing (such as a cloak) grants you a +2 bonus 
on the check.

Drawing a hidden weapon is a standard action and doesn't provoke an attack of opportunity.

If you try to take something from another creature, you must make a DC 20 Sleight 
of Hand check to obtain it. The opponent makes a Spot check to detect the attempt, 
opposed by the same Sleight of Hand check result you achieved when you tried to 
grab the item. An opponent who succeeds on this check notices the attempt, regardless 
of whether you got the item.

You can also use Sleight of Hand to entertain an audience as though you were using 
the Perform skill. In such a case, your "act" encompasses elements of legerdemain, 
juggling, and the like.

\begin{table}[htb]
\rowcolors{1}{white}{offyellow}
\caption{Sleight of Hand DCs}
\centering
\begin{tabular}{c l}
\textbf{Sleight of Hand DC} & \textbf{Task}\\
10 & Palm a coin-sized object, make a coin disappear\\
20 & Lift a small object from a person\\
\end{tabular}
\end{table}

\textbf{Action:} Any Sleight of Hand check normally is a standard action. However, 
you may perform a Sleight of Hand check as a free action by taking a -20 penalty 
on the check.

\textbf{Try Again:} Yes, but after an initial failure, a second Sleight of Hand 
attempt against the same target (or while you are being watched by the same observer 
who noticed your previous attempt) increases the DC for the task by 10.

\textbf{Special:} If you have the \linkfeat{Deft Hands} feat, you get a +2 bonus on Sleight 
of Hand checks.

\textbf{Synergy:} If you have 5 or more ranks in \linkskill{Bluff}, you get a +2 bonus on Sleight 
of Hand checks.

\textbf{Untrained:} An untrained Sleight of Hand check is simply a Dexterity check. 
Without actual training, you can't succeed on any Sleight of Hand check with a 
DC higher than 10, except for hiding an object on your body.

%%%%%%%%%%%%%%%%%%%%%%%%%
\skillentry{Speak Language}{(None; Trained Only)}
%%%%%%%%%%%%%%%%%%%%%%%%%

\begin{table}[htb]
\rowcolors{1}{white}{offyellow}
\caption{Common Languages and Their Alphabets}
\centering
\begin{tabular}{l l l}
\textbf{Language} & \textbf{Typical Speakers} & \textbf{Alphabet}\\
Abyssal & Demons, chaotic evil outsiders & Infernal\\
Aquan & Water-based creatures & Elven\\
Auran & Air-based creatures & Draconic\\
Celestial & Good outsiders & Celestial\\
Common & Humans, halflings, half-elves, half-orcs & Common\\
Draconic & Kobolds, troglodytes, lizardfolk, dragons & Draconic\\
Druidic & Druids (only) & Druidic\\
Dwarven & Dwarves & Dwarven\\
Elven & Elves & Elven\\
Giant & Ogres, giants & Dwarven\\
Gnome & Gnomes & Dwarven\\
Goblin & Goblins, hobgoblins, bugbears & Dwarven\\
Gnoll & Gnolls & Common\\
Halfling & Halflings & Common\\
Ignan & Fire-based creatures & Draconic\\
Infernal & Devils, lawful evil outsiders & Infernal\\
Orc & Orcs & Dwarven\\
Sylvan & Dryads, brownies, leprechauns & Elven\\
Terran & Xorns and other earth-based creatures & Dwarven\\
Undercommon & Drow & Elven\\
\end{tabular}
\end{table}

\textbf{Action:} Not applicable.

\textbf{Try Again:} Not applicable. There are no Speak Language checks to fail.

The Speak Language skill doesn't work like other skills. Languages work as follows.

\begin{itemize*}
\item You start at 1st level knowing one or two languages (based on your race), plus 
an additional number of languages equal to your starting Intelligence bonus.
\item You can purchase Speak Language just like any other skill, but instead of buying 
a rank in it, you choose a new language that you can speak.
\item You don't make Speak Language checks. You either know a language or you don't.
\end{itemize*}

A literate character (anyone but a \linkclass{Barbarian} who has not spent skill points to 
become literate) can read and write any language she speaks. Each language has 
an alphabet, though sometimes several spoken languages share a single alphabet.

%%%%%%%%%%%%%%%%%%%%%%%%%
\skillentry{Spellcraft}{(Int; Trained Only)}
%%%%%%%%%%%%%%%%%%%%%%%%%

Use this skill to identify spells as they are cast or spells already in place.

\begin{table}[htb]
\rowcolors{1}{white}{offyellow}
\caption{Spellcraft DCs}
\centering
\begin{tabular}{l p{13cm}}
\textbf{Spellcraft DC} & \textbf{Task}\\
13 & When using \linkspell{Read Magic}, identify a \linkspell{Glyph of Warding}. No action required.\\
15 + spell level & Identify a spell being cast. (You must see or hear the spell's verbal or somatic components.) No action required. No retry.\\
15 + spell level & Learn a spell from a spellbook or scroll (wizard only). No retry for that spell until you gain at least 1 rank in Spellcraft (even if you find another source to try to learn the spell from). Requires 8 hours.\\
15 + spell level & Prepare a spell from a borrowed spellbook (wizard only). One try per day. No extra time required.\\
15 + spell level & When casting \linkspell{Detect Magic}, determine the school of magic involved in the aura of a single item or creature you can see. (If the aura is not a spell effect, the DC is 15 + one-half caster level.) No action required.\\
19 & When using \linkspell{Read Magic}, identify a \textit{Symbol}. No action required.\\
20 + spell level & Identify a spell that's already in place and in effect. You must be able to see or detect the effects of the spell. No action required. No retry.\\
20 + spell level & Identify materials created or shaped by magic, such as noting that an iron wall is the result of a \linkspell{Wall of Iron} spell. No action required. No retry.\\
20 + spell level & Decipher a written spell (such as a scroll) without using \linkspell{Read Magic}. One try per day. Requires a full-round action.\\
25 + spell level & After rolling a saving throw against a spell targeted on you, determine what that spell was. No action required. No retry.\\
25 & Identify a potion. Requires 1 minute. No retry.\\
20 & Draw a diagram to allow \linkspell{Dimensional Anchor} to be cast on a \textit{Magic Circle} spell. Requires 10 minutes. No retry. This check is made secretly so you do not know the result.\\
30 or higher & Understand a strange or unique magical effect, such as the effects of a magic stream. Time required varies. No retry.\\
\end{tabular}
\end{table}

\textbf{Check:} You can identify spells and magic effects. The DCs for Spellcraft 
checks relating to various tasks are summarized on the table above.

\textbf{Action:} Varies, as noted above.

\textbf{Try Again:} See above.

\textbf{Special:} If you are a specialist \linkclass{Wizard}, you get a +2 bonus on Spellcraft 
checks when dealing with a spell or effect from your specialty school. You take 
a -5 penalty when dealing with a spell or effect from a prohibited school (and 
some tasks, such as learning a prohibited spell, are just impossible).

If you have the \linkfeat{Magical Aptitude} feat, you get a +2 bonus on Spellcraft checks.

\textbf{Synergy:} If you have 5 or more ranks in \linkskill{Knowledge} (arcana), you get a 
+2 bonus on Spellcraft checks.

If you have 5 or more ranks in \linkskill{Use Magic Device}, you get a +2 bonus on Spellcraft 
checks to decipher spells on scrolls.

If you have 5 or more ranks in Spellcraft, you get a +2 bonus on Use Magic Device 
checks related to scrolls.

Additionally, certain spells allow you to gain information about magic, provided 
that you make a successful Spellcraft check as detailed in the spell description.

%%%%%%%%%%%%%%%%%%%%%%%%%
\skillentry{Spot}{(Wis)}
%%%%%%%%%%%%%%%%%%%%%%%%%

\textbf{Check:} The Spot skill is used primarily to detect characters or creatures 
who are hiding. Typically, your Spot check is opposed by the \linkskill{Hide} check of the 
creature trying not to be seen. Sometimes a creature isn't intentionally hiding 
but is still difficult to see, so a successful Spot check is necessary to notice 
it.

A Spot check result higher than 20 generally lets you become aware of an invisible 
creature near you, though you can't actually see it.

Spot is also used to detect someone in \linkskill{Disguise} (see the Disguise skill), and to 
read lips when you can't hear or understand what someone is saying.

Spot checks may be called for to determine the distance at which an encounter begins. 
A penalty applies on such checks, depending on the distance between the two individuals 
or groups, and an additional penalty may apply if the character making the Spot 
check is distracted (not concentrating on being observant).

\begin{table}[htb]
\rowcolors{1}{white}{offyellow}
\caption{Spot Modifiers}
\centering
\begin{tabular}{lc}
\textbf{Condition} & \textbf{Penalty}\\
Per 10 feet of distance & -1\\
Spotter distracted & -5\\
\end{tabular}
\end{table}

\textit{Read Lips:} To understand what someone is saying by reading lips, you must 
be within 30 feet of the speaker, be able to see him or her speak, and understand 
the speaker's language. (This use of the skill is language-dependent.) The base 
DC is 15, but it increases for complex speech or an inarticulate speaker. You must 
maintain a line of sight to the lips being read.

If your Spot check succeeds, you can understand the general content of a minute's 
worth of speaking, but you usually still miss certain details. If the check fails 
by 4 or less, you can't read the speaker's lips. If the check fails by 5 or more, 
you draw some incorrect conclusion about the speech. The check is rolled secretly 
in this case, so that you don't know whether you succeeded or missed by 5.

\textbf{Action:} Varies. Every time you have a chance to spot something in a reactive 
manner you can make a Spot check without using an action. Trying to spot something 
you failed to see previously is a move action. To read lips, you must concentrate 
for a full minute before making a Spot check, and you can't perform any other action 
(other than moving at up to half speed) during this minute.

\textbf{Try Again:} Yes. You can try to spot something that you failed to see previously 
at no penalty. You can attempt to read lips once per minute.

\textbf{Special:} A \linksec{Fascinated} creature takes a -4 penalty on Spot checks made 
as reactions.

If you have the \linkfeat{Alertness} feat, you get a +2 bonus on Spot checks.

A \linkclass{Ranger} gains a bonus on Spot checks when using this skill against a favored enemy.

An \linkrace{Elf} has a +2 racial bonus on Spot checks.

A \linkrace{Half-Elf} has a +1 racial bonus on Spot checks.

The master of a hawk familiar gains a +3 bonus on Spot checks in daylight or other 
lighted areas.

The master of an owl familiar gains a +3 bonus on Spot checks in shadowy or other 
darkened areas.

%%%%%%%%%%%%%%%%%%%%%%%%%
\skillentry{Survival}{(Wis)}
%%%%%%%%%%%%%%%%%%%%%%%%%

\textbf{Check:} You can keep yourself and others safe and fed in the wild. The 
table below gives the DCs for various tasks that require Survival checks.

Survival does not allow you to follow difficult tracks unless you are a \linkclass{Ranger} 
or have the \linkfeat{Track} feat (see the Restriction section below).

\begin{table}[htb]
\rowcolors{1}{white}{offyellow}
\caption{Survival DCs}
\centering
\begin{tabular}{l p{14cm}}
\textbf{Survival DC} & \textbf{Task}\\
10 & Get along in the wild. Move up to one-half your overland speed while hunting and foraging (no food or water supplies needed). You can provide food and water for one other person for every 2 points by which your check result exceeds 10.\\
15 & Gain a +2 bonus on all Fortitude saves against severe weather while moving up to one-half your overland speed, or gain a +4 bonus if you remain stationary. You may grant the same bonus to one other character for every 1 point by which your Survival check result exceeds 15.\\
15 & Keep from getting lost or avoid natural hazards, such as quicksand.\\
15 & Predict the weather up to 24 hours in advance. For every 5 points by which your Survival check result exceeds 15, you can predict the weather for one additional day in advance.\\
Varies & Follow tracks (see the \linkfeat{Track} feat).\\
\end{tabular}
\end{table}

\textbf{Action:} Varies. A single Survival check may represent activity over the 
course of hours or a full day. A Survival check made to find tracks is at least 
a full-round action, and it may take even longer.

\textbf{Try Again:} Varies. For getting along in the wild or for gaining the Fortitude 
save bonus noted in the table above, you make a Survival check once every 24 hours. 
The result of that check applies until the next check is made. To avoid getting 
lost or avoid natural hazards, you make a Survival check whenever the situation 
calls for one. Retries to avoid getting lost in a specific situation or to avoid 
a specific natural hazard are not allowed. For finding tracks, you can retry a 
failed check after 1 hour (outdoors) or 10 minutes(indoors) of searching.

\textbf{Restriction:} While anyone can use Survival to find tracks (regardless 
of the DC), or to follow tracks when the DC for the task is 10 or lower, only a 
ranger (or a character with the Track feat) can use Survival to follow tracks when 
the task has a higher DC.

\textbf{Special:} If you have 5 or more ranks in Survival, you can automatically 
determine where true north lies in relation to yourself.

A \linkclass{Ranger} gains a bonus on Survival checks when using this skill to find or follow 
the tracks of a favored enemy.

If you have the \linkfeat{Self-Sufficient} feat, you get a +2 bonus on Survival checks.

\textbf{Synergy:} If you have 5 or more ranks in Survival, you get a +2 bonus on 
\linkskill{Knowledge} (nature) checks.

If you have 5 or more ranks in Knowledge (dungeoneering), you get a +2 bonus on 
Survival checks made while underground.

If you have 5 or more ranks in Knowledge (nature), you get a +2 bonus on Survival 
checks in aboveground natural environments (aquatic, desert, forest, hill, marsh, 
mountains, and plains).

If you have 5 or more ranks in Knowledge (geography), you get a +2 bonus on Survival 
checks made to keep from getting lost or to avoid natural hazards.

If you have 5 or more ranks in Knowledge (the planes), you get a +2 bonus on Survival 
checks made while on other planes.

If you have 5 or more ranks in \linkskill{Search}, you get a +2 bonus on Survival checks to 
find or follow tracks.

%%%%%%%%%%%%%%%%%%%%%%%%%
\skillentry{Swim}{(Str; Armor Check Penalty)}
%%%%%%%%%%%%%%%%%%%%%%%%%

\textbf{Check:} Make a Swim check once per round while you are in the water. Success 
means you may swim at up to one-half your speed (as a full-round action) or at 
one-quarter your speed (as a move action). If you fail by 4 or less, you make no 
progress through the water. If you fail by 5 or more, you go underwater.

If you are underwater, either because you failed a Swim check or because you are 
swimming underwater intentionally, you must hold your breath. You can hold your 
breath for a number of rounds equal to your Constitution score, but only if you 
do nothing other than take move actions or free actions. If you take a standard 
action or a full-round action (such as making an attack), the remainder of the 
duration for which you can hold your breath is reduced by 1 round. (Effectively, 
a character in combat can hold his or her breath only half as long as normal.) 
After that period of time, you must make a DC 10 Constitution check every round 
to continue holding your breath. Each round, the DC for that check increases by 
1. If you fail the Constitution check, you begin to drown.

The DC for the Swim check depends on the water, as given on the table below.

\begin{table}[htb]
\rowcolors{1}{white}{offyellow}
\caption{Swim DCs}
\centering
\begin{tabular}{l l}
\textbf{Task} & \textbf{Swim DC}\\
Calm water & 10\\
Rough water & 15\\
Stormy water & 20\textsuperscript{1}\\
\multicolumn{2}{p{6cm}}{\textsuperscript{1} You can't take 10 on a Swim check in stormy water, even if you aren't otherwise being threatened or distracted.}\\
\end{tabular}
\end{table}

Each hour that you swim, you must make a DC 20 Swim check or take 1d6 points of 
nonlethal damage from fatigue.

\textbf{Action:} A successful Swim check allows you to swim one-quarter of your 
speed as a move action or one-half your speed as a full-round action.

\textbf{Special:} Swim checks are subject to double the normal armor check penalty 
and encumbrance penalty.

If you have the \linkfeat{Athletic} feat, you get a +2 bonus on Swim checks.

If you have the \linkfeat{Endurance} feat, you get a +4 bonus on Swim checks made to avoid 
taking nonlethal damage from fatigue.

A creature with a swim speed can move through water at its indicated speed without 
making Swim checks. It gains a +8 racial bonus on any Swim check to perform a special 
action or avoid a hazard. The creature always can choose to take 10 on a Swim check, 
even if distracted or endangered when swimming. Such a creature can use the run 
action while swimming, provided that it swims in a straight line.

%%%%%%%%%%%%%%%%%%%%%%%%%
\skillentry{Tumble}{(Dex; Trained Only; Armor Check Penalty)}
%%%%%%%%%%%%%%%%%%%%%%%%%

You can't use this skill if your speed has been reduced by armor, excess equipment, 
or loot.

\textbf{Check:} You can land softly when you fall or tumble past opponents. You 
can also tumble to entertain an audience (as though using the Perform skill). The 
DCs for various tasks involving the Tumble skill are given on the table below.

\begin{table}[htb]
\rowcolors{1}{white}{offyellow}
\caption{Tumble DCs}
\centering
\begin{tabular}{l p{14cm}}
\textbf{Tumble DC} & \textbf{Task}\\
15 & Treat a fall as if it were 10 feet shorter than it really is when determining damage.\\
15 & Tumble at one-half speed as part of normal movement, provoking no attacks of opportunity while doing so. Failure means you provoke attacks of opportunity normally. Check separately for each opponent you move past, in the order in which you pass them (player’s choice of order in case of a tie). Each additional enemy after the first adds +2 to the Tumble DC.\\
25 & Tumble at one-half speed through an area occupied by an enemy (over, under, or around the opponent) as part of normal movement, provoking no attacks of opportunity while doing so. Failure means you stop before entering the enemy-occupied area and provoke an attack of opportunity from that enemy. Check separately for each opponent. Each additional enemy after the first adds +2 to the Tumble DC.\\
\end{tabular}
\end{table}

Obstructed or otherwise treacherous surfaces, such as natural cavern floors or 
undergrowth, are tough to tumble through. The DC for any Tumble check made to tumble 
into such a square is modified as indicated below.

\begin{table}[htb]
\rowcolors{1}{white}{offyellow}
\caption{Tumble Surface Modifiers}
\centering
\begin{tabular}{l c}
\textbf{Surface Is \ldots{}} & \textbf{DC Modifier}\\
Lightly obstructed (scree, light rubble, shallow bog\textsuperscript{1}, undergrowth) & +2\\
Severely obstructed (natural cavern floor, dense rubble, dense undergrowth) & +5\\
Lightly slippery (wet floor) & +2\\
Severely slippery (ice sheet) & +5\\
Sloped or angled & +2\\
\multicolumn{2}{p{6cm}}{\textsuperscript{1} Tumbling is impossible in a deep bog.}\\
\end{tabular}
\end{table}

\textit{Accelerated Tumbling:} You try to tumble past or through enemies more quickly 
than normal. By accepting a -10 penalty on your Tumble checks, you can move at 
your full speed instead of one-half your speed.

\textbf{Action:} Not applicable. Tumbling is part of movement, so a Tumble check 
is part of a move action.

\textbf{Try Again:} Usually no. An audience, once it has judged a tumbler as an 
uninteresting performer, is not receptive to repeat performances.

You can try to reduce damage from a fall as an instant reaction only once per fall.

\textbf{Special:} If you have 5 or more ranks in Tumble, you gain a +3 dodge bonus 
to AC when fighting defensively instead of the usual +2 dodge bonus to AC.

If you have 5 or more ranks in Tumble, you gain a +6 dodge bonus to AC when executing 
the total defense standard action instead of the usual +4 dodge bonus to AC.

If you have the \linkfeat{Acrobatic} feat, you get a +2 bonus on Tumble checks.

\textbf{Synergy:} If you have 5 or more ranks in Tumble, you get a +2 bonus on 
\linkskill{Balance} and \linkskill{Jump} checks.

If you have 5 or more ranks in Jump, you get a +2 bonus on Tumble checks.

%%%%%%%%%%%%%%%%%%%%%%%%%
\skillentry{Use Magic Device}{(Cha; Trained Only)}
%%%%%%%%%%%%%%%%%%%%%%%%%

Use this skill to activate magic

\textbf{Check:} You can use this skill to read a spell or to activate a magic item. 
Use Magic Device lets you use a magic item as if you had the spell ability or class 
features of another class, as if you were a different race, or as if you were of 
a different alignment.

You make a Use Magic Device check each time you activate a device such as a wand. 
If you are using the check to emulate an alignment or some other quality in an 
ongoing manner, you need to make the relevant Use Magic Device check once per hour.

You must consciously choose which requirement to emulate. That is, you must know 
what you are trying to emulate when you make a Use Magic Device check for that 
purpose. The DCs for various tasks involving Use Magic Device checks are summarized 
on the table below.

\begin{table}[htb]
\rowcolors{1}{white}{offyellow}
\caption{Use Magic Device DCs}
\centering
\begin{tabular}{l c}
\textbf{Task} & \textbf{Use Magic Device DC}\\
Activate Blindly & 25\\
Decipher a written spell & 25 + spell level\\
Use a scroll & 20 + caster level\\
Use a wand & 20\\
Emulate a class feature & 20\\
Emulate an ability score & See Text\\
Emulate a race & 25\\
Emulate an alignment & 30\\
\end{tabular}
\end{table}

\textit{Activate Blindly:} Some magic items are activated by special words, thoughts, 
or actions. You can activate such an item as if you were using the activation word, 
thought, or action, even when you're not and even if you don't know it. You do 
have to perform some equivalent activity in order to make the check. That is, you 
must speak, wave the item around, or otherwise attempt to get it to activate. You 
get a special +2 bonus on your Use Magic Device check if you've activated the item 
in question at least once before. If you fail by 9 or less, you can't activate 
the device. If you fail by 10 or more, you suffer a mishap. A mishap means that 
magical energy gets released but it doesn't do what you wanted it to do. The default 
mishaps are that the item affects the wrong target or that uncontrolled magical 
energy is released, dealing 2d6 points of damage to you. This mishap is in addition 
to the chance for a mishap that you normally run when you cast a spell from a scroll 
that you could not otherwise cast yourself.

\textit{Decipher a Written Spell:} This usage works just like deciphering a written 
spell with the Spellcraft skill, except that the DC is 5 points higher. Deciphering 
a written spell requires 1 minute of concentration.

\textit{Emulate an Ability Score:} To cast a spell from a scroll, you need a high 
score in the appropriate ability (Intelligence for wizard spells, Wisdom for divine 
spells, or Charisma for sorcerer or bard spells). Your effective ability score 
(appropriate to the class you're emulating when you try to cast the spell from 
the scroll) is your Use Magic Device check result minus 15. If you already have 
a high enough score in the appropriate ability, you don't need to make this check.

\textit{Emulate an Alignment:} Some magic items have positive or negative effects 
based on the user's alignment. Use Magic Device lets you use these items as if 
you were of an alignment of your choice. You can emulate only one alignment at 
a time.

\textit{Emulate a Class Feature:} Sometimes you need to use a class feature to 
activate a magic item. In this case, your effective level in the emulated class 
equals your Use Magic Device check result minus 20.  This skill does not let you 
actually use the class feature of another class. It just lets you activate items 
as if you had that class feature. If the class whose feature you are emulating 
has an alignment requirement, you must meet it, either honestly or by emulating 
an appropriate alignment with a separate Use Magic Device check (see above).

\textit{Emulate a Race:} Some magic items work only for members of certain races, 
or work better for members of those races. You can use such an item as if you were 
a race of your choice. You can emulate only one race at a time.

\textit{Use a Scroll:} If you are casting a spell from a scroll, you have to decipher 
it first. Normally, to cast a spell from a scroll, you must have the scroll's spell 
on your class spell list. Use Magic Device allows you to use a scroll as if you 
had a particular spell on your class spell list. The DC is equal to 20 + the caster 
level of the spell you are trying to cast from the scroll. In addition, casting 
a spell from a scroll requires a minimum score (10 + spell level) in the appropriate 
ability. If you don't have a sufficient score in that ability, you must emulate 
the ability score with a separate Use Magic Device check (see above).

This use of the skill also applies to other spell completion magic items.

\textit{Use a Wand:} Normally, to use a wand, you must have the wand's spell on 
your class spell list. This use of the skill allows you to use a wand as if you 
had a particular spell on your class spell list. This use of the skill also applies 
to other spell trigger magic items, such as staffs.

\textbf{Action:} None. The Use Magic Device check is made as part of the action 
(if any) required to activate the magic item.

\textbf{Try Again:} Yes, but if you ever roll a natural 1 while attempting to activate 
an item and you fail, then you can't try to activate that item again for 24 hours.

\textbf{Special:} You cannot take 10 with this skill.

You can't aid another on Use Magic Device checks. Only the user of the item may 
attempt such a check.

If you have the \linkfeat{Magical Aptitude} feat, you get a +2 bonus on Use Magic Device checks.

\textbf{Synergy:} If you have 5 or more ranks in \linkskill{Spellcraft}, you get a +2 bonus 
on Use Magic Device checks related to scrolls.

If you have 5 or more ranks in \linkskill{Decipher Script}, you get a +2 bonus on Use Magic 
Device checks related to scrolls.

If you have 5 or more ranks in Use Magic Device, you get a +2 bonus to Spellcraft 
checks made to decipher spells on scrolls.

%%%%%%%%%%%%%%%%%%%%%%%%%
\skillentry{Use Rope}{(Dex)}
%%%%%%%%%%%%%%%%%%%%%%%%%

\textbf{Check:} Most tasks with a rope are relatively simple. The DCs for various 
tasks utilizing this skill are summarized on the table below.

\begin{table}[htb]
\rowcolors{1}{white}{offyellow}
\caption{Tumble Surface Modifiers}
\centering
\begin{tabular}{c l}
\textbf{Use Rope DC} & \textbf{Task}\\
10 & Tie a firm knot\\
10\textsuperscript{1} & Secure a grappling hook\\
15 & Tie a special knot, such as one that slips, slides slowly, or loosens with a tug\\
15 & Tie a rope around yourself one-handed\\
15 & Splice two ropes together\\
Varies & Bind a character\\
\multicolumn{2}{l}{\textsuperscript{1} Add 2 to the DC for every 10 feet the hook is thrown; see below.}\\
\end{tabular}
\end{table}

\textit{Secure a Grappling Hook:} Securing a grappling hook requires a Use Rope 
check (DC 10, +2 for every 10 feet of distance the grappling hook is thrown, to 
a maximum DC of 20 at 50 feet). Failure by 4 or less indicates that the hook fails 
to catch and falls, allowing you to try again. Failure by 5 or more indicates that 
the grappling hook initially holds, but comes loose after 1d4 rounds of supporting 
weight. This check is made secretly, so that you don't know whether the rope will 
hold your weight.

\textit{Bind a Character:} When you bind another character with a rope, any Escape 
Artist check that the bound character makes is opposed by your Use Rope check.

You get a +10 bonus on this check because it is easier to bind someone than to 
escape from bonds. You don't even make your Use Rope check until someone tries 
to escape.

\textbf{Action:} Varies. Throwing a grappling hook is a standard action that provokes 
an attack of opportunity. Tying a knot, tying a special knot, or tying a rope around 
yourself one-handed is a full-round action that provokes an attack of opportunity. 
Splicing two ropes together takes 5 minutes. Binding a character takes 1 minute.

\textbf{Special:} A silk rope gives you a +2 circumstance bonus on Use Rope checks. 
If you cast an \linkspell{Animate Rope} spell on a rope, you get a +2 circumstance 
bonus on any Use Rope checks you make when using that rope.

These bonuses stack.

If you have the \linkfeat{Deft Hands} feat, you get a +2 bonus on Use Rope checks.

\textbf{Synergy:} If you have 5 or more ranks in Use Rope, you get a +2 bonus on 
\linkskill{Climb} checks made to climb a rope, a knotted rope, or a rope-and-wall combination.

If you have 5 or more ranks in Use Rope, you get a +2 bonus on \linkskill{Escape Artist} checks 
when escaping from rope bonds.

If you have 5 or more ranks in Escape Artist, you get a +2 bonus on checks made 
to bind someone.

