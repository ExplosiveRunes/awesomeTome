\spellentry{Binding}

Enchantment (Compulsion) [Mind-Affecting]

\textbf{Level:} Sor/Wiz 8

\textbf{Components:} V, S, M

\textbf{Casting Time:} One minute

\textbf{Range:} Close (25 ft. + 5 ft./2 levels)

\textbf{Target:} One living creature

\textbf{Duration:} See text (D)

\textbf{Saving Throw:} Will negates; see text

\textbf{Spell Resistance:} Yes

A binding spell creates a magical restraint to hold a creature. The target 
gets an initial saving throw only if its Hit Dice equal at least one-half your 
caster level.

You may have as many as six assistants help you with the spell. For each assistant 
who casts \linkspell{Suggestion}, your caster level for this casting of binding 
increases by 1. For each assistant who casts \linkspell{Dominate Animal}, \linkspell{Dominate Person}, or \linkspell{Dominate Monster}, your caster level for this casting of binding increases by a number equal to one-third of that assistant's level, provided that 
the spell's target is appropriate for a binding spell. Since the assistants' 
spells are cast simply to improve your caster level for the purpose of the binding spell, saving throws and spell resistance against the assistants' spells are irrelevant. 
Your caster level determines whether the target gets an initial Will saving throw 
and how long the binding lasts. All binding spells are dismissible.

Regardless of the version of binding you cast, you can specify triggering 
conditions that end the spell and release the creature whenever they occur. These 
triggers can be as simple or elaborate as you desire, but the condition must be 
reasonable and have a likelihood of coming to pass. The conditions can be based 
on a creature's name, identity, or alignment but otherwise must be based on observable 
actions or qualities. Intangibles such as level, class, Hit Dice, or hit points 
don't qualify. Once the spell is cast, its triggering conditions cannot be changed. 
Setting a release condition increases the save DC (assuming a saving throw is allowed) 
by 2.

If you are casting any of the first three versions of binding (those with 
limited durations), you may cast additional binding spells to prolong 
the effect, since the durations overlap. If you do so, the target gets a saving 
throw at the end of the first spell's duration, even if your caster level was high 
enough to disallow an initial saving throw. If the creature succeeds on this save, 
all the binding spells it has received are broken.

The binding spell has six versions. Choose one of the following versions 
when you cast the spell.

\textit{Chaining:} The subject is confined by restraints that generate an \linkspell{Antipathy}
spell affecting all creatures who approach the subject, except you. The duration 
is one year per caster level. The subject of this form of Binding is confined 
to the spot it occupied when it received the spell.

\textit{Slumber:} This version causes the subject to become comatose for as long 
as one year per caster level. The subject does not need to eat or drink while slumbering, 
nor does it age. This form of binding is more difficult to cast than chaining, 
making it slightly easier to resist. Reduce the spell's save DC by 1.

\textit{Bound Slumber:} This combination of chaining and slumber 
lasts for as long as one month per caster level. Reduce the save DC by 2.

\textit{Hedged Prison:} The subject is transported to or otherwise brought within 
a confined area from which it cannot wander by any means. The effect is permanent. 
Reduce the save DC by 3.

\textit{Metamorphosis:} The subject assumes gaseous form, except for its head or 
face. It is held harmless in a jar or other container, which may be transparent 
if you so choose. The creature remains aware of its surroundings and can speak, 
but it cannot leave the container, attack, or use any of its powers or abilities. 
The binding is permanent. The subject does not need to breathe, eat, or 
drink while metamorphosed, nor does it age. Reduce the save DC by 4.

\textit{Minimus Containment:} The subject is shrunk to a height of 1 inch or even 
less and held within some gem, jar, or similar object. The binding is 
permanent. The subject does not need to breathe, eat, or drink while contained, 
nor does it age. Reduce the save DC by 4.

You can't dispel a binding spell with \linkspell{Dispel Magic} or a similar 
effect, though an \linkspell{Antimagic Field} or \linkspell{Mage's Disjunction} affects 
it normally. A bound extraplanar creature cannot be sent back to its home plane 
due to \linkspell{Dismissal}, \linkspell{Banishment}, or a similar effect.

\textit{Components:} The components for a binding spell vary according 
to the version of the spell, but they always include a continuous chanting utterance 
read from the scroll or spellbook page containing the spell, somatic gestures, 
and materials appropriate to the form of binding used. These components 
can include such items as miniature chains of special metals, soporific herbs of 
the rarest sort (for slumber bindings), a bell jar of the finest crystal, 
and the like.

In addition to the specially made props suited to the specific type of binding (cost 500 gp), the spell requires opals worth at least 500 gp for each HD of the 
target and a vellum depiction or carved statuette of the subject to be captured.

