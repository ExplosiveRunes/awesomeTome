\spellentry{Freezing Sphere}

Evocation [Cold]

\textbf{Level:} Sor/Wiz 6

\textbf{Components:} V, S, F

\textbf{Casting Time:} 1 standard action

\textbf{Range:} Long (400 ft. + 40 ft./level)

\textbf{Target, Effect, or Area:} See text

\textbf{Duration:} Instantaneous or 1 round/level; see text

\textbf{Saving Throw:} Reflex half; see text

\textbf{Spell Resistance:} Yes

\textit{Freezing sphere} creates a frigid globe of cold energy that streaks from 
your fingertips to the location you select, where it explodes in a 10-foot-radius 
burst, dealing 1d6 points of cold damage per caster level (maximum 15d6) to each 
creature in the area. An elemental (water) creature instead takes 1d8 points of 
cold damage per caster level (maximum 15d8).

If the \textit{freezing sphere} strikes a body of water or a liquid that is principally 
water (not including water-based creatures), it freezes the liquid to a depth of 
6 inches over an area equal to 100 square feet (a 10- foot square) per caster level 
(maximum 1,500 square feet). This ice lasts for 1 round per caster level. Creatures 
that were swimming on the surface of frozen water become trapped in the ice. Attempting 
to break free is a full-round action. A trapped creature must make a DC 25 Strength 
check or a DC 25 Escape Artist check to do so.

You can refrain from firing the globe after completing the spell, if you wish. 
Treat this as a touch spell for which you are holding the charge. You can hold 
the charge for as long as 1 round per level, at the end of which time the \textit{freezing 
sphere} bursts centered on you (and you receive no saving throw to resist its effect). 
Firing the globe in a later round is a standard action.

\textit{Focus:} A small crystal sphere.

