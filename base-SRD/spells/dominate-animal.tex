\spellentry{Dominate Animal}

Enchantment (Compulsion) [Mind-Affecting]

\textbf{Level:} Animal 3, Drd 3

\textbf{Components:} V, S

\textbf{Casting Time:} 1 round

\textbf{Range:} Close (25 ft. + 5 ft./2 levels)

\textbf{Target:} One animal

\textbf{Duration:} 1 round/level

\textbf{Saving Throw:} Will negates

\textbf{Spell Resistance:} Yes

You can enchant an animal and direct it with simple commands such as "Attack," 
"Run," and "Fetch." Suicidal or self-destructive commands (including an order 
to attack a creature two or more size categories larger than the \textit{dominated 
}animal) are simply ignored.

\textit{Dominate animal} establishes a mental link between you and the subject 
creature. The animal can be directed by silent mental command as long as it remains 
in range. You need not see the creature to control it. You do not receive direct 
sensory input from the creature, but you know what it is experiencing. Because 
you are directing the animal with your own intelligence, it may be able to undertake 
actions normally beyond its own comprehension. You need not concentrate exclusively 
on controlling the creature unless you are trying to direct it to do something 
it normally couldn't do. Changing your instructions or giving a \textit{dominated 
}creature a new command is the equivalent of redirecting a spell, so it is a move 
action.

