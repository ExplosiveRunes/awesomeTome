\spellentry{Whirlwind}

Evocation [Air]

\textbf{Level:} Air 8, Drd 8

\textbf{Components:} V, S, DF

\textbf{Casting Time:} 1 standard action

\textbf{Range:} Long (400 ft. + 40 ft./level)

\textbf{Effect:} Cyclone 10 ft. wide at base, 30 ft. wide at top, and 30 ft. tall

\textbf{Duration:} 1 round/level (D)

\textbf{Saving Throw:} Reflex negates; see text

\textbf{Spell Resistance:} Yes

This spell creates a powerful cyclone of raging wind that moves through the air, 
along the ground, or over water at a speed of 60 feet per round. You can concentrate 
on controlling the cyclone's every movement or specify a simple program. Directing 
the cyclone's movement or changing its programmed movement is a standard action 
for you. The cyclone always moves during your turn. If the cyclone exceeds the 
spell's range, it moves in a random, uncontrolled fashion for 1d3 rounds and then 
dissipates. (You can't regain control of the cyclone, even if comes back within 
range.)

Any Large or smaller creature that comes in contact with the spell effect must 
succeed on a Reflex save or take 3d6 points of damage. A Medium or smaller creature 
that fails its first save must succeed on a second one or be picked up bodily by 
the cyclone and held suspended in its powerful winds, taking 1d8 points of damage 
each round on your turn with no save allowed. You may direct the cyclone to eject 
any carried creatures whenever you wish, depositing the hapless souls wherever 
the cyclone happens to be when they are released.

