\spellentry{Magic Circle Against Evil}

Abjuration [Good]

\textbf{Level:} Clr 3, Good 3, Pal 3, Sor/Wiz 3

\textbf{Components:} V, S, M/DF

\textbf{Casting Time:} 1 standard action

\textbf{Range:} Touch

\textbf{Area:} 10-ft.-radius emanation from touched creature

\textbf{Duration:} 10 min./level

\textbf{Saving Throw:} Will negates (harmless)

\textbf{Spell Resistance:} No; see text

All creatures within the area gain the effects of a \textit{protection from evil 
}spell, and no nongood summoned creatures can enter the area either. You must overcome 
a creature's spell resistance in order to keep it at bay (as in the third function 
of \textit{protection from evil}), but the deflection and resistance bonuses and 
the protection from mental control apply regardless of enemies' spell resistance.

This spell has an alternative version that you may choose when casting it. A \textit{magic 
circle against evil} can be focused inward rather than outward. When focused inward, 
the spell binds a nongood called creature (such as those called by the \textit{lesser 
planar binding, planar binding}, and \textit{greater planar binding} spells) for 
a maximum of 24 hours per caster level, provided that you cast the spell that calls 
the creature within 1 round of casting the \textit{magic circle}. The creature 
cannot cross the circle's boundaries. If a creature too large to fit into the spell's 
area is the subject of the spell, the spell acts as a normal \textit{protection 
from evil} spell for that creature only.

A \textit{magic circle} leaves much to be desired as a trap. If the circle of powdered 
silver laid down in the process of spellcasting is broken, the effect immediately 
ends. The trapped creature can do nothing that disturbs the circle, directly or 
indirectly, but other creatures can. If the called creature has spell resistance, 
it can test the trap once a day. If you fail to overcome its spell resistance, 
the creature breaks free, destroying the circle. A creature capable of any form 
of dimensional travel (\textit{astral projection, blink, dimension door, etherealness, 
gate, plane shift, shadow walk, teleport}, and similar abilities) can simply leave 
the circle through that means. You can prevent the creature's extradimensional 
escape by casting a \textit{dimensional anchor} spell on it, but you must cast 
the spell before the creature acts. If you are successful, the \textit{anchor} effect 
lasts as long as the \textit{magic circle} does. The creature cannot reach across 
the \textit{magic circle}, but its ranged attacks (ranged weapons, spells, magical 
abilities, and the like) can. The creature can attack any target it can reach with 
its ranged attacks except for the circle itself.

You can add a special diagram (a two-dimensional bounded figure with no gaps along 
its circumference, augmented with various magical sigils) to make the \textit{magic 
circle} more secure. Drawing the diagram by hand takes 10 minutes and requires 
a DC 20 Spellcraft check. You do not know the result of this check. If the check 
fails, the diagram is ineffective. You can take 10 when drawing the diagram if 
you are under no particular time pressure to complete the task. This task also 
takes 10 full minutes. If time is no factor at all, and you devote 3 hours and 
20 minutes to the task, you can take 20.

A successful diagram allows you to cast a \textit{dimensional anchor} spell on 
the \textit{magic circle} during the round before casting any summoning spell. 
The \textit{anchor} holds any called creatures in the \textit{magic circle} for 
24 hours per caster level. A creature cannot use its spell resistance against a 
\textit{magic circle} prepared with a diagram, and none of its abilities or attacks 
can cross the diagram. If the creature tries a Charisma check to break free of 
the trap (see the \textit{lesser planar binding} spell), the DC increases by 5. 
The creature is immediately released if anything disturbs the diagram -- even a 
straw laid across it. However, the creature itself cannot disturb the diagram either 
directly or indirectly, as noted above.

This spell is not cumulative with \textit{protection from evil} and vice versa.

\textit{Arcane Material Component:} A little powdered silver with which you trace 
a 3-footdiameter circle on the floor (or ground) around the creature to be warded.

