%%%%%%%%%%%%%%%%%%%%%%%%%%%%%%%%%%%%%%%%%%%%%%%%%%
\section{Movement, Position, and Distance}
%%%%%%%%%%%%%%%%%%%%%%%%%%%%%%%%%%%%%%%%%%%%%%%%%%

Miniatures are on the 30mm scale -- a miniature figure of a six-foot-tall human 
is approximately 30mm tall. A square on the battle grid is 1 inch across, representing 
a 5-foot-by-5-foot area.

%%%%%%%%%%%%%%%%%%%%%%%%%
\subsection{Tactical Movement In Combat}\index{Movement}
%%%%%%%%%%%%%%%%%%%%%%%%%

%%%
\subsubsection{How Far Can Your Character Move?}
%%%

Your speed is determined by your race and your armor (see Table: Tactical Speed). 
Your speed while unarmored is your base land speed.

\textbf{\gameterm{Encumbrance}:} A character encumbered by carrying a large amount of gear, 
treasure, or fallen comrades may move slower than normal.

\textbf{\gameterm{Hampered Movement}:} Difficult terrain, obstacles, or poor visibility can 
hamper movement.

\textbf{Movement in Combat:} Generally, you can move your speed in a round and 
still do something (take a move action and a standard action).

If you do nothing but move (that is, if you use both of your actions in a round 
to move your speed), you can move double your speed.

If you spend the entire round running, you can move quadruple your speed. If you 
do something that requires a full round you can only take a 5-foot step.

\textbf{Bonuses to Speed:} A barbarian has a +10 foot bonus to his speed (unless 
he's wearing heavy armor). Experienced monks also have higher speed (unless they're 
wearing armor of any sort). In addition, many spells and magic items can affect 
a character's speed. Always apply any modifiers to a character's speed before adjusting 
the character's speed based on armor or encumbrance, and remember that multiple 
bonuses of the same type to a character's speed don't stack.

\begin{table}[htb]
\rowcolors{1}{white}{offyellow}
\caption{Tactical Speed}
\centering
\begin{tabular}{l l l}
\textbf{Race} & \textbf{No Armor or Light Armor} & \textbf{Medium or Heavy Armor}\\
Human, Elf, Half-Elf, Half-Orc & 30ft (6 squares) & 20ft (4 squares)\\
Dwarf & 20ft (4 squares) & 20ft (4 squares)\\
Halfling & 20ft (4 squares) & 15ft (3 squares)\\
\end{tabular}
\end{table}

%%%
\subsubsection{Measuring Distance}
%%%

\textbf{Diagonals:}\index{Movement!Diagonals} When measuring distance, the first diagonal counts as 1 square, 
the second counts as 2 squares, the third counts as 1, the fourth as 2, and so 
on.

You can't move diagonally past a corner (even by taking a 5-foot step). You can 
move diagonally past a creature, even an opponent.

You can also move diagonally past other impassable obstacles, such as pits.

\textbf{Closest Creature:} When it's important to determine the closest square 
or creature to a location, if two squares or creatures are equally close, randomly 
determine which one counts as closest by rolling a die.

%%%
\subsubsection{Moving through a Square}
%%%

\textbf{Friend:} You can move through a square occupied by a friendly character, 
unless you are charging. When you move through a square occupied by a friendly 
character, that character doesn't provide you with cover.

\textbf{Opponent:} You can't move through a square occupied by an opponent, unless 
the opponent is helpless. You can move through a square occupied by a helpless 
opponent without penalty. (Some creatures, particularly very large ones, may present 
an obstacle even when helpless. In such cases, each square you move through counts 
as 2 squares.)

\textbf{Ending Your Movement:} You can't end your movement in the same square as 
another creature unless it is helpless.

\textbf{Overrun:} During your movement you can attempt to move through a square 
occupied by an opponent.

\textbf{Tumbling:} A trained character can attempt to tumble through a square occupied 
by an opponent (see the Tumble skill).

\textbf{Very Small Creature:} A Fine, Diminutive, or Tiny creature can move into 
or through an occupied square. The creature provokes attacks of opportunity when 
doing so.

\textbf{Square Occupied by Creature Three Sizes Larger or Smaller:} Any creature 
can move through a square occupied by a creature three size categories larger than 
it is.

A big creature can move through a square occupied by a creature three size categories 
smaller than it is.

\textbf{Designated Exceptions:} Some creatures break the above rules. A creature 
that completely fills the squares it occupies cannot be moved past, even with the 
\linkskill{Tumble} skill or similar special abilities.

%%%
\subsubsection{Terrain and Obstacles}
%%%

\textbf{\gameterm{Difficult Terrain}:} Difficult terrain hampers movement. Each square of 
difficult terrain counts as 2 squares of movement. (Each diagonal move into a difficult 
terrain square counts as 3 squares.) You can't run or charge across difficult terrain.

If you occupy squares with different kinds of terrain, you can move only as fast 
as the most difficult terrain you occupy will allow.

Flying and incorporeal creatures are not hampered by difficult terrain.

\textbf{Obstacles:} Like difficult terrain, obstacles can hamper movement. If an 
obstacle hampers movement but doesn't completely block it each obstructed square 
or obstacle between squares counts as 2 squares of movement. You must pay this 
cost to cross the barrier, in addition to the cost to move into the square on the 
other side. If you don't have sufficient movement to cross the barrier and move 
into the square on the other side, you can't cross the barrier. Some obstacles 
may also require a skill check to cross.

On the other hand, some obstacles block movement entirely. A character can't move 
through a blocking obstacle.

Flying and incorporeal creatures can avoid most obstacles

\textbf{\gameterm{Squeezing}:} In some cases, you may have to squeeze into or through an area 
that isn't as wide as the space you take up. You can squeeze through or into a 
space that is at least half as wide as your normal space. Each move into or through 
a narrow space counts as if it were 2 squares, and while squeezed in a narrow space 
you take a -4 penalty on attack rolls and a -4 penalty to AC.

When a Large creature (which normally takes up four squares) squeezes into a space 
that's one square wide, the creature's miniature figure occupies two squares, centered 
on the line between the two squares. For a bigger creature, center the creature 
likewise in the area it squeezes into.

A creature can squeeze past an opponent while moving but it can't end its movement 
in an occupied square.

To squeeze through or into a space less than half your space's width, you must 
use the Escape Artist skill. You can't attack while using Escape Artist to squeeze 
through or into a narrow space, you take a -4 penalty to AC, and you lose any Dexterity 
bonus to AC.

%%%
\subsubsection{Special Movement Rules}
%%%

These rules cover special movement situations.

\textbf{Accidentally Ending Movement in an Illegal Space:} Sometimes a character 
ends its movement while moving through a space where it's not allowed to stop. 
When that happens, put your miniature in the last legal position you occupied, 
or the closest legal position, if there's a legal position that's closer.

\textbf{Double Movement Cost:} When your movement is hampered in some way, your 
movement usually costs double. For example, each square of movement through difficult 
terrain counts as 2 squares, and each diagonal move through such terrain counts 
as 3 squares (just as two diagonal moves normally do).

If movement cost is doubled twice, then each square counts as 4 squares (or as 
6 squares if moving diagonally). If movement cost is doubled three times, then 
each square counts as 8 squares (12 if diagonal) and so on. This is an exception 
to the general rule that two doublings are equivalent to a tripling.

\textbf{Minimum Movement:} Despite penalties to movement, you can take a full-round 
action to move 5 feet (1 square) in any direction, even diagonally. (This rule 
doesn't allow you to move through impassable terrain or to move when all movement 
is prohibited.) Such movement provokes attacks of opportunity as normal (despite 
the distance covered, this move isn't a 5-foot step).

%%%%%%%%%%%%%%%%%%%%%%%%%
\subsection{Big And Little Creatures In Combat}
%%%%%%%%%%%%%%%%%%%%%%%%%

Creatures smaller than Small or larger than Medium have special rules relating 
to position. 

\textbf{Tiny, Diminutive, and Fine Creatures:} Very small creatures take up less 
than 1 square of space. This means that more than one such creature can fit into 
a single square. A Tiny creature typically occupies a space only 2-1/2 feet across, 
so four can fit into a single square. Twenty-five Diminutive creatures or 100 Fine 
creatures can fit into a single square. Creatures that take up less than 1 square 
of space typically have a natural reach of 0 feet, meaning they can't reach into 
adjacent squares. They must enter an opponent's square to attack in melee. This 
provokes an attack of opportunity from the opponent. You can attack into your own 
square if you need to, so you can attack such creatures normally. Since they have 
no natural reach, they do not threaten the squares around them. You can move past 
them without provoking attacks of opportunity. They also can't flank an enemy.

\textbf{Large, Huge, Gargantuan, and Colossal Creatures:} Very large creatures 
take up more than 1 square.

Creatures that take up more than 1 square typically have a natural reach of 10 
feet or more, meaning that they can reach targets even if they aren't in adjacent 
squares.

Unlike when someone uses a reach weapon, a creature with greater than normal natural 
reach (more than 5 feet) still threatens squares adjacent to it. A creature with 
greater than normal natural reach usually gets an attack of opportunity against 
you if you approach it, because you must enter and move within the range of its 
reach before you can attack it. (This attack of opportunity is not provoked if 
you take a 5-foot step.)

Large or larger creatures using reach weapons can strike up to double their natural 
reach but can't strike at their natural reach or less. 

\begin{table}[htb]
\rowcolors{1}{white}{offyellow}
\caption{Creature Size and Scale}
\centering
\begin{tabular}{lcc}
\textbf{Creature Size} & \textbf{Space\textsuperscript{1}} & \textbf{Natural Reach\textsuperscript{1}}\\
Fine & 1/2ft & 0ft\\
Diminutive & 1ft & 0ft\\
Tiny & 2.5ft & 0ft\\
Small & 5ft & 5ft\\
Medium & 5ft & 5ft\\
Large (long) & 10ft & 5ft\\
Large (tall) & 10ft & 10ft\\
Huge (long) & 15ft & 10ft\\
Huge (tall) & 15ft & 15ft\\
Gargantuan (long) & 20ft & 15ft\\
Gargantuan (tall) & 20ft & 20ft\\
Colossal (long) & 30ft & 20ft\\
Colossal (tall) & 30ft & 30ft\\
\multicolumn{3}{p{6cm}}{\textsuperscript{1} These values are typical for creatures of the indicated size. Some exceptions exist.}\\
\end{tabular}
\end{table}

%%%%%%%%%%%%%%%%%%%%%%%%%%%%%%%%%%%%%%%%%%%%%%%%%%
\section{Combat Modifiers}
%%%%%%%%%%%%%%%%%%%%%%%%%%%%%%%%%%%%%%%%%%%%%%%%%%

%%%%%%%%%%%%%%%%%%%%%%%%%
\subsection{Favorable And Unfavorable Conditions}
%%%%%%%%%%%%%%%%%%%%%%%%%

\begin{table}[htb]
\rowcolors{1}{white}{offyellow}
\caption{Attack Roll Modifiers}
\centering
\begin{tabular}{p{7cm}cc}
\textbf{Attacker is \ldots{}} & \textbf{Melee} & \textbf{Ranged}\\
Dazzled & -1 & -1\\
Entangled & -2\textsuperscript{1} & -2\textsuperscript{1}\\
Flanking defender & +2 & --\\
Invisible & +2\textsuperscript{2} & +2\textsuperscript{2}\\
On higher ground & +1 & +0\\
Prone & -4 & --\textsuperscript{3}\\
Shaken or frightened & -2 & -2\\
Squeezing through a space & -4 & -4\\
\multicolumn{3}{p{10cm}}{\textsuperscript{1} An entangled character also takes a -4 penalty to Dexterity, which may affect his attack roll.}\\
\multicolumn{3}{p{10cm}}{\textsuperscript{2} The defender loses any Dexterity bonus to AC. This bonus doesn't apply if the target is blinded.}\\
\multicolumn{3}{p{10cm}}{\textsuperscript{3} Most ranged weapons can't be used while the attacker is prone, but you can use a crossbow or shuriken while prone at no penalty.}\\
\end{tabular}
\end{table}

%perhaps convert to tablularx
\begin{table}[htb]
\rowcolors{1}{white}{offyellow}\mcinherit
\caption{Armor Class Modifiers}
\centering
\begin{tabular}{p{8.5cm}cc}
\textbf{Defender is \ldots{}} & \textbf{Melee} & \textbf{Ranged}\\
Behind cover & +4 & +4\\
Blinded & -2\textsuperscript{1} & -2\textsuperscript{1}\\
Concealed or invisible & \multicolumn{2}{c}{-- See Concealment --}\\
Cowering & -2\textsuperscript{1} & -2\textsuperscript{1}\\
Entangled & +0\textsuperscript{2} & +0\textsuperscript{2}\\
Flat-footed (such as surprised, balancing, climbing) & +0\textsuperscript{1} & +0\textsuperscript{1}\\
Grappling (but attacker is not) & +0\textsuperscript{1} & +0\textsuperscript{1,3}\\
Helpless (such as paralyzed, sleeping, or bound) & -4\textsuperscript{4} & +0\textsuperscript{4}\\
Kneeling or sitting & -2 & +2\\
Pinned & -4\textsuperscript{4} & +0\textsuperscript{4}\\
Prone & -4 & +4\\
Squeezing through a space & -4 & -4\\
Stunned & -2\textsuperscript{1} & -2\textsuperscript{1}\\
\multicolumn{3}{p{12cm}}{\textsuperscript{1} The defender loses any Dexterity bonus to AC.}\\
\multicolumn{3}{p{12cm}}{\textsuperscript{2} An entangled character takes a -4 penalty to Dexterity.}\\
\multicolumn{3}{p{12cm}}{\textsuperscript{3} Roll randomly to see which grappling combatant you strike. That defender loses any Dexterity bonus to AC.}\\
\multicolumn{3}{p{12cm}}{\textsuperscript{4} Treat the defender's Dexterity as 0 (-5 modifier). Rogues can sneak attack helpless or pinned defenders.}\\
\end{tabular}
\end{table}

%%%%%%%%%%%%%%%%%%%%%%%%%
\subsection{Cover}\index{Cover}
%%%%%%%%%%%%%%%%%%%%%%%%%

To determine whether your target has cover from your ranged attack, choose a corner 
of your square. If any line from this corner to any corner of the target's square 
passes through a square or border that blocks line of effect or provides cover, 
or through a square occupied by a creature, the target has cover (+4 to AC).

When making a melee attack against an adjacent target, your target has cover if 
any line from your square to the target's square goes through a wall (including 
a low wall). When making a melee attack against a target that isn't adjacent to 
you (such as with a reach weapon), use the rules for determining cover from ranged 
attacks.

\textbf{Low Obstacles and Cover:} A low obstacle (such as a wall no higher than 
half your height) provides cover, but only to creatures within 30 feet (6 squares) 
of it. The attacker can ignore the cover if he's closer to the obstacle than his 
target.

\textbf{Cover and Attacks of Opportunity:} You can't execute an attack of opportunity 
against an opponent with cover relative to you.

\textbf{Cover and Reflex Saves:} Cover grants you a +2 bonus on Reflex saves against 
attacks that originate or burst out from a point on the other side of the cover 
from you. Note that spread effects can extend around corners and thus negate this 
cover bonus.

\textbf{Cover and Hide Checks:} You can use cover to make a Hide check. Without 
cover, you usually need concealment (see below) to make a Hide check.

\textbf{Soft Cover:} Creatures, even your enemies, can provide you with cover against 
ranged attacks, giving you a +4 bonus to AC. However, such soft cover provides 
no bonus on Reflex saves, nor does soft cover allow you to make a Hide check.

\textbf{Big Creatures and Cover:} Any creature with a space larger than 5 feet 
(1 square) determines cover against melee attacks slightly differently than smaller 
creatures do. Such a creature can choose any square that it occupies to determine 
if an opponent has cover against its melee attacks. Similarly, when making a melee 
attack against such a creature, you can pick any of the squares it occupies to 
determine if it has cover against you.

\textbf{Total Cover:} If you don't have line of effect to your target he is considered 
to have total cover from you. You can't make an attack against a target that has 
total cover.

\textbf{Varying Degrees of Cover:} In some cases, cover may provide a greater bonus 
to AC and Reflex saves. In such situations the normal cover bonuses to AC and Reflex 
saves can be doubled (to +8 and +4, respectively). A creature with this improved 
cover effectively gains improved evasion against any attack to which the Reflex 
save bonus applies. Furthermore, improved cover provides a +10 bonus on Hide checks.

%%%%%%%%%%%%%%%%%%%%%%%%%
\subsection{Concealment}\index{Concealment}
%%%%%%%%%%%%%%%%%%%%%%%%%

To determine whether your target has concealment from your ranged attack, choose 
a corner of your square. If any line from this corner to any corner of the target's 
square passes through a square or border that provides concealment, the target 
has concealment.

When making a melee attack against an adjacent target, your target has concealment 
if his space is entirely within an effect that grants concealment. When making 
a melee attack against a target that isn't adjacent to you use the rules for determining 
concealment from ranged attacks.

In addition, some magical effects provide concealment against all attacks, regardless 
of whether any intervening concealment exists.

\textbf{Concealment Miss Chance:} Concealment gives the subject of a successful 
attack a 20\% chance that the attacker missed because of the concealment. If the 
attacker hits, the defender must make a miss chance percentile roll to avoid being 
struck. Multiple concealment conditions do not stack.

\textbf{Concealment and Hide Checks:} You can use concealment to make a Hide check. 
Without concealment, you usually need cover to make a Hide check.

\textbf{Total Concealment:} If you have line of effect to a target but not line 
of sight he is considered to have total concealment from you. You can't attack 
an opponent that has total concealment, though you can attack into a square that 
you think he occupies. A successful attack into a square occupied by an enemy with 
total concealment has a 50\% miss chance (instead of the normal 20\% miss chance 
for an opponent with concealment).

You can't execute an attack of opportunity against an opponent with total concealment, 
even if you know what square or squares the opponent occupies.

\textbf{Ignoring Concealment:} Concealment isn't always effective. A shadowy area 
or darkness doesn't provide any concealment against an opponent with darkvision. 
Characters with low-light vision can see clearly for a greater distance with the 
same light source than other characters. Although invisibility provides total concealment, 
sighted opponents may still make Spot checks to notice the location of an invisible 
character. An invisible character gains a +20 bonus on Hide checks if moving, or 
a +40 bonus on Hide checks when not moving (even though opponents can't see you, 
they might be able to figure out where you are from other visual clues).

\textbf{Varying Degrees of Concealment:} Certain situations may provide more or 
less than typical concealment, and modify the miss chance accordingly.

%%%%%%%%%%%%%%%%%%%%%%%%%
\subsection{Flanking}\index{Flanking}
%%%%%%%%%%%%%%%%%%%%%%%%%

When making a melee attack, you get a +2 flanking bonus if your opponent is threatened 
by a character or creature friendly to you on the opponent's opposite border or 
opposite corner.

When in doubt about whether two friendly characters flank an opponent in the middle, 
trace an imaginary line between the two friendly characters' centers. If the line 
passes through opposite borders of the opponent's space (including corners of those 
borders), then the opponent is flanked.

\textit{Exception:} If a flanker takes up more than 1 square, it gets the flanking 
bonus if any square it occupies counts for flanking.

Only a creature or character that threatens the defender can help an attacker get 
a flanking bonus.

Creatures with a reach of 0 feet can't flank an opponent.

%%%%%%%%%%%%%%%%%%%%%%%%%
\subsection{Helpless Defenders}\index{Helpless}
%%%%%%%%%%%%%%%%%%%%%%%%%

A helpless opponent is someone who is bound, sleeping, paralyzed, unconscious, 
or otherwise at your mercy.

\textbf{Regular Attack:} A helpless character takes a -4 penalty to AC against 
melee attacks, but no penalty to AC against ranged attacks.

A helpless defender can't use any Dexterity bonus to AC. In fact, his Dexterity 
score is treated as if it were 0 and his Dexterity modifier to AC as if it were 
-5 (and a rogue can sneak attack him).

\textbf{\gameterm{Coup de Grace}:} As a full-round action, you can use a melee weapon to deliver 
a coup de grace to a helpless opponent. You can also use a bow or crossbow, provided 
you are adjacent to the target.

You automatically hit and score a critical hit. If the defender survives the damage, 
he must make a Fortitude save (DC 10 + damage dealt) or die. A rogue also gets 
her extra sneak attack damage against a helpless opponent when delivering a coup 
de grace.

Delivering a coup de grace provokes attacks of opportunity from threatening opponents.

You can't deliver a coup de grace against a creature that is immune to critical 
hits. You can deliver a coup de grace against a creature with total concealment, 
but doing this requires two consecutive full-round actions (one to "find" the 
creature once you've determined what square it's in, and one to deliver the coup 
de grace).

%%%%%%%%%%%%%%%%%%%%%%%%%%%%%%%%%%%%%%%%%%%%%%%%%%
\section{Special Attacks}
%%%%%%%%%%%%%%%%%%%%%%%%%%%%%%%%%%%%%%%%%%%%%%%%%%

\begin{table}[htb]
\rowcolors{1}{white}{offyellow}\mcinherit
\caption{Special Attacks}
\centering
\begin{tabular}{ll}
\textbf{Special Attack} & \textbf{Brief Description}\\
Aid Another & Grant an ally a +2 bonus on attacks or AC\\
Bull Rush & Push an opponent back 5 feet or more\\
Charge & Move up to twice your speed and attack with +2 bonus\\
Disarm & Knock a weapon from your opponent's hands\\
Feint & Negate your opponent's Dex bonus to AC\\
Grapple & Wrestle with an opponent\\
Overrun & Plow past or over an opponent as you move\\
Sunder & Strike an opponent's weapon or shield\\
Throw splash weapon & Throw container of dangerous liquid at target\\
Trip & Trip an opponent\\
Turn (rebuke) undead & Channel positive (or negative) energy to turn away (or awe) undead\\
Two-weapon Fighting & Fight with a weapon in each hand\\
\end{tabular}
\end{table}

%%%%%%%%%%%%%%%%%%%%%%%%%
\subsection{Aid Another}\index{Aid Another}
%%%%%%%%%%%%%%%%%%%%%%%%%

In melee combat, you can help a friend attack or defend by distracting or interfering 
with an opponent. If you're in position to make a melee attack on an opponent that 
is engaging a friend in melee combat, you can attempt to aid your friend as a standard 
action. You make an attack roll against AC 10. If you succeed, your friend gains 
either a +2 bonus on his next attack roll against that opponent or a +2 bonus to 
AC against that opponent's next attack (your choice), as long as that attack comes 
before the beginning of your next turn. Multiple characters can aid the same friend, 
and similar bonuses stack.

You can also use this standard action to help a friend in other ways, such as when 
he is affected by a spell, or to assist another character's skill check.

%%%%%%%%%%%%%%%%%%%%%%%%%
\subsection{Bull Rush}\index{Bull Rush}
%%%%%%%%%%%%%%%%%%%%%%%%%

You can make a bull rush as a standard action (an attack) or as part of a charge 
(see Charge, below). When you make a bull rush, you attempt to push an opponent 
straight back instead of damaging him. You can only bull rush an opponent who is 
one size category larger than you, the same size, or smaller.

\textbf{Initiating a Bull Rush:} First, you move into the defender's space. Doing 
this provokes an attack of opportunity from each opponent that threatens you, including 
the defender. (If you have the Improved Bull Rush feat, you don't provoke an attack 
of opportunity from the defender.) Any attack of opportunity made by anyone other 
than the defender against you during a bull rush has a 25\% chance of accidentally 
targeting the defender instead, and any attack of opportunity by anyone other than 
you against the defender likewise has a 25\% chance of accidentally targeting you. 
(When someone makes an attack of opportunity, make the attack roll and then roll 
to see whether the attack went astray.) 

Second, you and the defender make opposed Strength checks. You each add a +4 bonus 
for each size category you are larger than Medium or a -4 penalty for each size 
category you are smaller than Medium. You get a +2 bonus if you are charging. The 
defender gets a +4 bonus if he has more than two legs or is otherwise exceptionally 
stable.

\textbf{Bull Rush Results:} If you beat the defender's Strength check result, you 
push him back 5 feet. If you wish to move with the defender, you can push him back 
an additional 5 feet for each 5 points by which your check result is greater than 
the defender's check result. You can't, however, exceed your normal movement limit. 
(\textit{Note:} The defender provokes attacks of opportunity if he is moved. So 
do you, if you move with him. The two of you do not provoke attacks of opportunity 
from each other, however.)

If you fail to beat the defender's Strength check result, you move 5 feet straight 
back to where you were before you moved into his space. If that space is occupied, 
you fall prone in that space.

%%%%%%%%%%%%%%%%%%%%%%%%%
\subsection{Charge}\index{Charge}
%%%%%%%%%%%%%%%%%%%%%%%%%

Charging is a special full-round action that allows you to move up to twice your 
speed and attack during the action. However, it carries tight restrictions on how 
you can move.

\textbf{Movement During a Charge:} You must move before your attack, not after. 
You must move at least 10 feet (2 squares) and may move up to double your speed 
directly toward the designated opponent.

You must have a clear path toward the opponent, and nothing can hinder your movement 
(such as difficult terrain or obstacles). Here's what it means to have a clear 
path. First, you must move to the closest space from which you can attack the opponent. 
(If this space is occupied or otherwise blocked, you can't charge.) Second, if 
any line from your starting space to the ending space passes through a square that 
blocks movement, slows movement, or contains a creature (even an ally), you can't 
charge. (Helpless creatures don't stop a charge.)

If you don't have line of sight to the opponent at the start of your turn, you 
can't charge that opponent.

You can't take a 5-foot step in the same round as a charge.

If you are able to take only a standard action or a move action on your turn, you 
can still charge, but you are only allowed to move up to your speed (instead of 
up to double your speed). You can't use this option unless you are restricted to 
taking only a standard action or move action on your turn.

\textbf{Attacking on a Charge:} After moving, you may make a single melee attack. 
You get a +2 bonus on the attack roll. and take a -2 penalty to your AC until the 
start of your next turn.

A charging character gets a +2 bonus on the Strength check made to bull rush an 
opponent (see Bull Rush, above).

Even if you have extra attacks, such as from having a high enough base attack bonus 
or from using multiple weapons, you only get to make one attack during a charge.

\textbf{Lances and Charge Attacks:} A lance deals double damage if employed by 
a mounted character in a charge.

\textbf{Weapons Readied against a Charge:} Spears, tridents, and certain other 
piercing weapons deal double damage when readied (set) and used against a charging 
character.

%%%%%%%%%%%%%%%%%%%%%%%%%
\subsection{Disarm}\index{Disarm}
%%%%%%%%%%%%%%%%%%%%%%%%%

As a melee attack, you may attempt to disarm your opponent. If you do so with a 
weapon, you knock the opponent's weapon out of his hands and to the ground. If 
you attempt the disarm while unarmed, you end up with the weapon in your hand.

If you're attempting to disarm a melee weapon, follow the steps outlined here. 
If the item you are attempting to disarm isn't a melee weapon the defender may 
still oppose you with an attack roll, but takes a penalty and can't attempt to 
disarm you in return if your attempt fails.

\textbf{Step 1:} Attack of Opportunity. You provoke an attack of opportunity from 
the target you are trying to disarm. (If you have the Improved Disarm feat, you 
don't incur an attack of opportunity for making a disarm attempt.) If the defender's 
attack of opportunity deals any damage, your disarm attempt fails.

\textbf{Step 2:} Opposed Rolls. You and the defender make opposed attack rolls 
with your respective weapons. The wielder of a two-handed weapon on a disarm attempt 
gets a +4 bonus on this roll, and the wielder of a light weapon takes a -4 penalty. 
(An unarmed strike is considered a light weapon, so you always take a penalty when 
trying to disarm an opponent by using an unarmed strike.) If the combatants are 
of different sizes, the larger combatant gets a bonus on the attack roll of +4 
per difference in size category. If the targeted item isn't a melee weapon, the 
defender takes a -4 penalty on the roll.

\textbf{Step Three:} Consequences. If you beat the defender, the defender is disarmed. 
If you attempted the disarm action unarmed, you now have the weapon. If you were 
armed, the defender's weapon is on the ground in the defender's square.

If you fail on the disarm attempt, the defender may immediately react and attempt 
to disarm you with the same sort of opposed melee attack roll. His attempt does 
not provoke an attack of opportunity from you. If he fails his disarm attempt, 
you do not subsequently get a free disarm attempt against him.

\textit{Note:} A defender wearing spiked gauntlets can't be disarmed. A defender 
using a weapon attached to a locked gauntlet gets a +10 bonus to resist being disarmed.

%%%
\subsubsection{Grabbing Items}
%%%

You can use a disarm action to snatch an item worn by the target. If you want to 
have the item in your hand, the disarm must be made as an unarmed attack.

If the item is poorly secured or otherwise easy to snatch or cut away the attacker 
gets a +4 bonus. Unlike on a normal disarm attempt, failing the attempt doesn't 
allow the defender to attempt to disarm you. This otherwise functions identically 
to a disarm attempt, as noted above.

You can't snatch an item that is well secured unless you have pinned the wearer 
(see Grapple). Even then, the defender gains a +4 bonus on his roll to resist the 
attempt.

%%%%%%%%%%%%%%%%%%%%%%%%%
\subsection{Feint}\index{Feint}
%%%%%%%%%%%%%%%%%%%%%%%%%

Feinting is a standard action. To feint, make a Bluff check opposed by a Sense 
Motive check by your target. The target may add his base attack bonus to this Sense 
Motive check. If your Bluff check result exceeds your target's Sense Motive check 
result, the next melee attack you make against the target does not allow him to 
use his Dexterity bonus to AC (if any). This attack must be made on or before your 
next turn.

When feinting in this way against a nonhumanoid you take a -4 penalty. Against 
a creature of animal Intelligence (1 or 2), you take a -8 penalty. Against a nonintelligent 
creature, it's impossible.

Feinting in combat does not provoke attacks of opportunity.

\textbf{Feinting as a Move Action:} With the Improved Feint feat, you can attempt 
a feint as a move action instead of as a standard action.

%%%%%%%%%%%%%%%%%%%%%%%%%
\subsection{Grapple}\index{Grapple}
%%%%%%%%%%%%%%%%%%%%%%%%%

%%%
\subsubsection{Grapple Checks}
%%%

Repeatedly in a grapple, you need to make opposed grapple checks against an opponent. 
A grapple check is like a melee attack roll. Your attack bonus on a grapple check 
is: Base attack bonus + Strength modifier + special size modifier

\textbf{Special Size Modifier:} The special size modifier for a grapple check is 
as follows: Colossal +16, Gargantuan +12, Huge +8, Large +4, Medium +0, Small -4, 
Tiny -8, Diminutive -12, Fine -16. Use this number in place of the normal size 
modifier you use when making an attack roll.

%%%
\subsubsection{Starting a Grapple}
%%%

To start a grapple, you need to grab and hold your target. Starting a grapple requires 
a successful melee attack roll. If you get multiple attacks, you can attempt to 
start a grapple multiple times (at successively lower base attack bonuses).

\textbf{Step 1:} Attack of Opportunity. You provoke an attack of opportunity from 
the target you are trying to grapple. If the attack of opportunity deals damage, 
the grapple attempt fails. (Certain monsters do not provoke attacks of opportunity 
when they attempt to grapple, nor do characters with the Improved Grapple feat.) 
If the attack of opportunity misses or fails to deal damage, proceed to Step 2.

\textbf{Step 2:} Grab. You make a melee touch attack to grab the target. If you 
fail to hit the target, the grapple attempt fails. If you succeed, proceed to Step 3.

\textbf{Step 3:} Hold. Make an opposed grapple check as a free action.

If you succeed, you and your target are now grappling, and you deal damage to the 
target as if with an unarmed strike.

If you lose, you fail to start the grapple. You automatically lose an attempt to 
hold if the target is two or more size categories larger than you are.

In case of a tie, the combatant with the higher grapple check modifier wins. If 
this is a tie, roll again to break the tie.

\textbf{Step 4:} Maintain Grapple. To maintain the grapple for later rounds, you 
must move into the target's space. (This movement is free and doesn't count as 
part of your movement in the round.)

Moving, as normal, provokes attacks of opportunity from threatening opponents, 
but not from your target.

If you can't move into your target's space, you can't maintain the grapple and 
must immediately let go of the target. To grapple again, you must begin at Step 1.

%%%
\subsubsection{Grappling Consequences}
%%%

While you're grappling, your ability to attack others and defend yourself is limited.

\textbf{No Threatened Squares:} You don't threaten any squares while grappling.

\textbf{No Dexterity Bonus:} You lose your Dexterity bonus to AC (if you have one) 
against opponents you aren't grappling. (You can still use it against opponents 
you are grappling.)

\textbf{No Movement:} You can't move normally while grappling. You may, however, 
make an opposed grapple check (see below) to move while grappling.

%%%
\subsubsection{If You're Grappling}
%%%

When you are grappling (regardless of who started the grapple), you can perform 
any of the following actions. Some of these actions take the place of an attack 
(rather than being a standard action or a move action). If your base attack bonus 
allows you multiple attacks, you can attempt one of these actions in place of each 
of your attacks, but at successively lower base attack bonuses.

\textbf{Activate a Magic Item:} You can activate a magic item, as long as the item 
doesn't require a spell completion trigger. You don't need to make a grapple check 
to activate the item.

\textbf{Attack Your Opponent:} You can make an attack with an unarmed strike, natural 
weapon, or light weapon against another character you are grappling. You take a 
-4 penalty on such attacks.

You can't attack with two weapons while grappling, even if both are light weapons.

\textbf{Cast a Spell:} You can attempt to cast a spell while grappling or even 
while pinned (see below), provided its casting time is no more than 1 standard 
action, it has no somatic component, and you have in hand any material components 
or focuses you might need. Any spell that requires precise and careful action\textit{ 
}is impossible to cast while grappling or being pinned. If the spell is one that 
you can cast while grappling, you must make a Concentration check (DC 20 + spell 
level) or lose the spell. You don't have to make a successful grapple check to 
cast the spell.

\textbf{Damage Your Opponent:} While grappling, you can deal damage to your opponent 
equivalent to an unarmed strike. Make an opposed grapple check in place of an attack. 
If you win, you deal nonlethal damage as normal for your unarmed strike (1d3 points 
for Medium attackers or 1d2 points for Small attackers, plus Strength modifiers). 
If you want to deal lethal damage, you take a -4 penalty on your grapple check.

\textit{Exception:} Monks deal more damage on an unarmed strike than other characters, 
and the damage is lethal. However, they can choose to deal their damage as nonlethal 
damage when grappling without taking the usual -4 penalty for changing lethal damage 
to nonlethal damage.

\textbf{Draw a Light Weapon:} You can draw a light weapon as a move action with 
a successful grapple check.

\textbf{Escape from Grapple:} You can escape a grapple by winning an opposed grapple 
check in place of making an attack. You can make an Escape Artist check in place 
of your grapple check if you so desire, but this requires a standard action. If 
more than one opponent is grappling you, your grapple check result has to beat 
all their individual check results to escape. (Opponents don't have to try to hold 
you if they don't want to.) If you escape, you finish the action by moving into 
any space adjacent to your opponent(s).

\textbf{Move:} You can move half your speed (bringing all others engaged in the 
grapple with you) by winning an opposed grapple check. This requires a standard 
action, and you must beat all the other individual check results to move the grapple.

\textit{Note:} You get a +4 bonus on your grapple check to move a pinned opponent, 
but only if no one else is involved in the grapple.

\textbf{Retrieve a Spell Component:} You can produce a spell component from your 
pouch while grappling by using a full-round action. Doing so does not require a 
successful grapple check.

\textbf{Pin Your Opponent:} You can hold your opponent immobile for 1 round by 
winning an opposed grapple check (made in place of an attack). Once you have an 
opponent pinned, you have a few options available to you (see below).

\textbf{Break Another's Pin:} If you are grappling an opponent who has another 
character pinned, you can make an opposed grapple check in place of an attack. 
If you win, you break the hold that the opponent has over the other character. 
The character is still grappling, but is no longer pinned.

\textbf{Use Opponent's Weapon:} If your opponent is holding a light weapon, you 
can use it to attack him. Make an opposed grapple check (in place of an attack). 
If you win, make an attack roll with the weapon with a -4 penalty (doing this doesn't 
require another action).

You don't gain possession of the weapon by performing this action.

%%%
\subsubsection{If You're Pinning an Opponent}
%%%

You can attempt to damage your opponent with an opposed grapple check, you can 
attempt to use your opponent's weapon against him, or you can attempt to move the 
grapple (all described above). At your option, you can prevent a pinned opponent 
from speaking.

You can use a disarm action to remove or grab away a well secured object worn by 
a pinned opponent, but he gets a +4 bonus on his roll to resist your attempt (see 
Disarm).

You may voluntarily release a pinned character as a free action; if you do so, 
you are no longer considered to be grappling that character (and vice versa).

You can't draw or use a weapon (against the pinned character or any other character), 
escape another's grapple, retrieve a spell component, pin another character, or 
break another's pin while you are pinning an opponent.

%%%
\subsubsection{If You're Pinned by an Opponent}
%%%

When an opponent has pinned you, you are held immobile (but not helpless) for 1 
round. While you're pinned, you take a -4 penalty to your AC against opponents 
other than the one pinning you. At your opponent's option, you may also be unable 
to speak. On your turn, you can try to escape the pin by making an opposed grapple 
check in place of an attack. You can make an Escape Artist check in place of your 
grapple check if you want, but this requires a standard action. If you win, you 
escape the pin, but you're still grappling.

%%%
\subsubsection{Joining a Grapple}
%%%

If your target is already grappling someone else, you can use an attack to start 
a grapple, as above, except that the target doesn't get an attack of opportunity 
against you, and your grab automatically succeeds. You still have to make a successful 
opposed grapple check to become part of the grapple.

If there are multiple opponents involved in the grapple, you pick one to make the 
opposed grapple check against.

%%%
\subsubsection{Multiple Grapplers}
%%%

Several combatants can be in a single grapple. Up to four combatants can grapple 
a single opponent in a given round. Creatures that are one or more size categories 
smaller than you count for half, creatures that are one size category larger than 
you count double, and creatures two or more size categories larger count quadruple.

When you are grappling with multiple opponents, you choose one opponent to make 
an opposed check against. The exception is an attempt to escape from the grapple; 
to successfully escape, your grapple check must beat the check results of each 
opponent.

%%%%%%%%%%%%%%%%%%%%%%%%%
\subsection{Mounted Combat}\index{Mounted Combat}
%%%%%%%%%%%%%%%%%%%%%%%%%

\textbf{Horses in Combat:} Warhorses and warponies can serve readily as combat 
steeds. Light horses, ponies, and heavy horses, however, are frightened by combat. 
If you don't dismount, you must make a DC 20 Ride check each round as a move action 
to control such a horse. If you succeed, you can perform a standard action after 
the move action. If you fail, the move action becomes a full round action and you 
can't do anything else until your next turn.

Your mount acts on your initiative count as you direct it. You move at its speed, 
but the mount uses its action to move.

A horse (not a pony) is a Large creature and thus takes up a space 10 feet (2 squares) 
across. For simplicity, assume that you share your mount's space during combat.

\textbf{Combat while Mounted:} With a DC 5 Ride check, you can guide your mount 
with your knees so as to use both hands to attack or defend yourself. This is a 
free action.

When you attack a creature smaller than your mount that is on foot, you get the 
+1 bonus on melee attacks for being on higher ground. If your mount moves more 
than 5 feet, you can only make a single melee attack. Essentially, you have to 
wait until the mount gets to your enemy before attacking, so you can't make a full 
attack. Even at your mount's full speed, you don't take any penalty on melee attacks 
while mounted.

If your mount charges, you also take the AC penalty associated with a charge. If 
you make an attack at the end of the charge, you receive the bonus gained from 
the charge. When charging on horseback, you deal double damage with a lance (see 
Charge).

You can use ranged weapons while your mount is taking a double move, but at a -4 
penalty on the attack roll. You can use ranged weapons while your mount is running 
(quadruple speed), at a -8 penalty. In either case, you make the attack roll when 
your mount has completed half its movement. You can make a full attack with a ranged 
weapon while your mount is moving. Likewise, you can take move actions normally

\textbf{Casting Spells while Mounted:} You can cast a spell normally if your mount 
moves up to a normal move (its speed) either before or after you cast. If you have 
your mount move both before and after you cast a spell, then you're casting the 
spell while the mount is moving, and you have to make a Concentration check due 
to the vigorous motion (DC 10 + spell level) or lose the spell. If the mount is 
running (quadruple speed), you can cast a spell when your mount has moved up to 
twice its speed, but your Concentration check is more difficult due to the violent 
motion (DC 15 + spell level).

\textbf{If Your Mount Falls in Battle:} If your mount falls, you have to succeed 
on a DC 15 Ride check to make a soft fall and take no damage. If the check fails, 
you take 1d6 points of damage.

\textbf{If You Are Dropped:} If you are knocked unconscious, you have a 50\% chance 
to stay in the saddle (or 75\% if you're in a military saddle). Otherwise you fall 
and take 1d6 points of damage.

Without you to guide it, your mount avoids combat.

%%%%%%%%%%%%%%%%%%%%%%%%%
\subsection{Overrun}\index{Overrun}
%%%%%%%%%%%%%%%%%%%%%%%%%

You can attempt an overrun as a standard action taken during your move. (In general, 
you cannot take a standard action during a move; this is an exception.) With an 
overrun, you attempt to plow past or over your opponent (and move through his square) 
as you move. You can only overrun an opponent who is one size category larger than 
you, the same size, or smaller. You can make only one overrun attempt per round.

If you're attempting to overrun an opponent, follow these steps.

\textbf{Step 1:} Attack of Opportunity. Since you begin the overrun by moving into 
the defender's space, you provoke an attack of opportunity from the defender.

\textbf{Step 2:} Opponent Avoids? The defender has the option to simply avoid you. 
If he avoids you, he doesn't suffer any ill effect and you may keep moving (You 
can always move through a square occupied by someone who lets you by.) The overrun 
attempt doesn't count against your actions this round (except for any movement 
required to enter the opponent's square). If your opponent doesn't avoid you, move 
to Step 3.

\textbf{Step 3:} Opponent Blocks? If your opponent blocks you, make a Strength 
check opposed by the defender's Dexterity or Strength check (whichever ability 
score has the higher modifier). A combatant gets a +4 bonus on the check for every 
size category he is larger than Medium or a -4 penalty for every size category 
he is smaller than Medium. The defender gets a +4 bonus on his check if he has 
more than two legs or is otherwise more stable than a normal humanoid. If you win, 
you knock the defender prone. If you lose, the defender may immediately react and 
make a Strength check opposed by your Dexterity or Strength check (including the 
size modifiers noted above, but no other modifiers) to try to knock you prone.

\textbf{Step 4:} Consequences. If you succeed in knocking your opponent prone, 
you can continue your movement as normal. If you fail and are knocked prone in 
turn, you have to move 5 feet back the way you came and fall prone, ending your 
movement there. If you fail but are not knocked prone, you have to move 5 feet 
back the way you came, ending your movement there. If that square is occupied, 
you fall prone in that square.

\textbf{Improved Overrun:} If you have the Improved Overrun feat, your target may 
not choose to avoid you.

\textbf{Mounted Overrun (Trample):} If you attempt an overrun while mounted, your 
mount makes the Strength check to determine the success or failure of the overrun 
attack (and applies its size modifier, rather than yours). If you have the Trample 
feat and attempt an overrun while mounted, your target may not choose to avoid 
you, and if you knock your opponent prone with the overrun, your mount may make 
one hoof attack against your opponent.

%%%%%%%%%%%%%%%%%%%%%%%%%
\subsection{Sunder}\index{Sunder}
%%%%%%%%%%%%%%%%%%%%%%%%%

You can use a melee attack with a slashing or bludgeoning weapon to strike a weapon 
or shield that your opponent is holding. If you're attempting to sunder a weapon 
or shield, follow the steps outlined here. (Attacking held objects other than weapons 
or shields is covered below.)

\textbf{Step 1:} Attack of Opportunity. You provoke an attack of opportunity from 
the target whose weapon or shield you are trying to sunder. (If you have the Improved 
Sunder feat, you don't incur an attack of opportunity for making the attempt.)

\textbf{Step 2:} Opposed Rolls. You and the defender make opposed attack rolls 
with your respective weapons. The wielder of a two-handed weapon on a sunder attempt 
gets a +4 bonus on this roll, and the wielder of a light weapon takes a -4 penalty. 
If the combatants are of different sizes, the larger combatant gets a bonus on 
the attack roll of +4 per difference in size category.

\textbf{Step 3:} Consequences. If you beat the defender, roll damage and deal it 
to the weapon or shield. See Table: Common Armor, Weapon, and Shield Hardness and 
Hit Points to determine how much damage you must deal to destroy the weapon or 
shield.

If you fail the sunder attempt, you don't deal any damage.

\textit{Sundering a Carried or Worn Object:} You don't use an opposed attack roll 
to damage a carried or worn object. Instead, just make an attack roll against the 
object's AC. A carried or worn object's AC is equal to 10 + its size modifier + 
the Dexterity modifier of the carrying or wearing character. Attacking a carried 
or worn object provokes an attack of opportunity just as attacking a held object 
does. To attempt to snatch away an item worn by a defender rather than damage it, 
see Disarm. You can't sunder armor worn by another character.

%%%%%%%%%%%%%%%%%%%%%%%%%
\subsection{Throw Splash Weapon}\index{Splash Weapons}
%%%%%%%%%%%%%%%%%%%%%%%%%

A splash weapon is a ranged weapon that breaks on impact, splashing or scattering 
its contents over its target and nearby creatures or objects. To attack with a 
splash weapon, make a ranged touch attack against the target. Thrown weapons require 
no weapon proficiency, so you don't take the -4 nonproficiency penalty. A hit deals 
direct hit damage to the target, and splash damage to all creatures within 5 feet 
of the target.

You can instead target a specific grid intersection. Treat this as a ranged attack 
against AC 5. However, if you target a grid intersection, creatures in all adjacent 
squares are dealt the splash damage, and the direct hit damage is not dealt to 
any creature. (You can't target a grid intersection occupied by a creature, such 
as a Large or larger creature; in this case, you're aiming at the creature.)

If you miss the target (whether aiming at a creature or a grid intersection), roll 
1d8. This determines the misdirection of the throw, with 1 being straight back 
at you and 2 through 8 counting clockwise around the grid intersection or target 
creature. Then, count a number of squares in the indicated direction equal to the 
range increment of the throw.

After you determine where the weapon landed, it deals splash damage to all creatures 
in adjacent squares.

%%%%%%%%%%%%%%%%%%%%%%%%%
\subsection{Trip}\index{Trip}
%%%%%%%%%%%%%%%%%%%%%%%%%

You can try to trip an opponent as an unarmed melee attack. You can only trip an 
opponent who is one size category larger than you, the same size, or smaller.

\textbf{Making a Trip Attack:} Make an unarmed melee touch attack against your 
target. This provokes an attack of opportunity from your target as normal for unarmed 
attacks.

If your attack succeeds, make a Strength check opposed by the defender's Dexterity 
or Strength check (whichever ability score has the higher modifier). A combatant 
gets a +4 bonus for every size category he is larger than Medium or a -4 penalty 
for every size category he is smaller than Medium. The defender gets a +4 bonus 
on his check if he has more than two legs or is otherwise more stable than a normal 
humanoid. If you win, you trip the defender. If you lose, the defender may immediately 
react and make a Strength check opposed by your Dexterity or Strength check to 
try to trip you.

\textit{Avoiding Attacks of Opportunity:} If you have the Improved Trip feat, or 
if you are tripping with a weapon (see below), you don't provoke an attack of opportunity 
for making a trip attack.

\textbf{Being Tripped (Prone):} A tripped character is prone. Standing up is a 
move action.

\textbf{Tripping a Mounted Opponent:} You may make a trip attack against a mounted 
opponent. The defender may make a Ride check in place of his Dexterity or Strength 
check. If you succeed, you pull the rider from his mount.

\textbf{Tripping with a Weapon:} Some weapons can be used to make trip attacks. 
In this case, you make a melee touch attack with the weapon instead of an unarmed 
melee touch attack, and you don't provoke an attack of opportunity.

If you are tripped during your own trip attempt, you can drop the weapon to avoid 
being tripped.

%%%%%%%%%%%%%%%%%%%%%%%%%
\subsection{Turn or Rebuke Undead}\index{Turn Undead}\index{Rebuke Undead}
%%%%%%%%%%%%%%%%%%%%%%%%%

Good clerics and paladins and some neutral clerics can channel positive energy, 
which can halt, drive off (rout), or destroy undead.

Evil clerics and some neutral clerics can channel negative energy, which can halt, 
awe (rebuke), control (command), or bolster undead.

Regardless of the effect, the general term for the activity is "turning." When 
attempting to exercise their divine control over these creatures, characters make 
turning checks.

%%%
\subsubsection{Turning Checks}
%%%

Turning undead is a supernatural ability that a character can perform as a standard 
action. It does not provoke attacks of opportunity.

You must present your holy symbol to turn undead. Turning is considered an attack.

\textbf{Times per Day:} You may attempt to turn undead a number of times per day 
equal to 3 + your Charisma modifier. You can increase this number by taking the 
Extra Turning feat.

\textbf{Range:} You turn the closest turnable undead first, and you can't turn 
undead that are more than 60 feet away or that have total cover relative to you. 
You don't need line of sight to a target, but you do need line of effect.

\textbf{Turning Check:} The first thing you do is roll a turning check to see how 
powerful an undead creature you can turn. This is a Charisma check (1d20 + your 
Charisma modifier). Table: Turning Undead gives you the Hit Dice of the most powerful 
undead you can affect, relative to your level. On a given turning attempt, you 
can turn no undead creature whose Hit Dice exceed the result on this table.

\textbf{Turning Damage:} If your roll on Table: Turning Undead is high enough to 
let you turn at least some of the undead within 60 feet, roll 2d6 + your cleric 
level + your Charisma modifier for turning damage. That's how many total Hit Dice 
of undead you can turn.

If your Charisma score is average or low, it's possible to roll fewer Hit Dice 
of undead turned than indicated on Table: Turning Undead.

You may skip over already turned undead that are still within range, so that you 
do not waste your turning capacity on them.

\textbf{Effect and Duration of Turning:} Turned undead flee from you by the best 
and fastest means available to them. They flee for 10 rounds (1 minute). If they 
cannot flee, they cower (giving any attack rolls against them a +2 bonus). If you 
approach within 10 feet of them, however, they overcome being turned and act normally. 
(You can stand within 10 feet without breaking the turning effect -- you just can't 
approach them.) You can attack them with ranged attacks (from at least 10 feet 
away), and others can attack them in any fashion, without breaking the turning 
effect.

\textbf{Destroying Undead:} If you have twice as many levels (or more) as the undead 
have Hit Dice, you destroy any that you would normally turn.

\begin{table}[htb]
\rowcolors{1}{white}{offyellow}\mcinherit
\caption{Turning Undead}
\centering
\begin{tabular}{c c}
\multicolumn{1}{p{3.5cm}}{\textbf{Turning Check Result}} & \multicolumn{1}{p{5cm}}{\textbf{Most Powerful Undead Affected (Maximum Hit Dice)}}\\
0 or lower & Cleric's Level -4\\
1-3 & Cleric's Level -3\\
4-6 & Cleric's Level -2\\
7-9 & Cleric's Level -1\\
10-12 & Cleric's Level\\
13-15 & Cleric's Level +1\\
16-18 & Cleric's Level +2\\
19-21 & Cleric's Level +3\\
22 or higher & Cleric's Level +4\\
\end{tabular}
\end{table}

%%%
\subsubsection{Evil Clerics and Undead}
%%%

Evil clerics channel negative energy to rebuke (awe) or command (control) undead 
rather than channeling positive energy to turn or destroy them. An evil cleric 
makes the equivalent of a turning check. Undead that would be turned are rebuked 
instead, and those that would be destroyed are commanded.

\textbf{Rebuked:} A rebuked undead creature cowers as if in awe (attack rolls against 
the creature get a +2 bonus). The effect lasts 10 rounds.

\textbf{Commanded:} A commanded undead creature is under the mental control of 
the evil cleric. The cleric must take a standard action to give mental orders to 
a commanded undead. At any one time, the cleric may command any number of undead 
whose total Hit Dice do not exceed his level. He may voluntarily relinquish command 
on any commanded undead creature or creatures in order to command new ones.

\textbf{Dispelling Turning:} An evil cleric may channel negative energy to dispel 
a good cleric's turning effect. The evil cleric makes a turning check as if attempting 
to rebuke the undead. If the turning check result is equal to or greater than the 
turning check result that the good cleric scored when turning the undead, then 
the undead are no longer turned. The evil cleric rolls turning damage of 2d6 + 
cleric level + Charisma modifier to see how many Hit Dice worth of undead he can 
affect in this way (as if he were rebuking them).

\textbf{Bolstering Undead:} An evil cleric may also bolster undead creatures against 
turning in advance. He makes a turning check as if attempting to rebuke the undead, 
but the Hit Dice result on Table: Turning Undead becomes the undead creatures' 
effective Hit Dice as far as turning is concerned (provided the result is higher 
than the creatures' actual Hit Dice). The bolstering lasts 10 rounds. An evil undead 
cleric can bolster himself in this manner.

%%%
\subsubsection{Neutral Clerics and Undead}
%%%

A cleric of neutral alignment can either turn undead but not rebuke them, or rebuke 
undead but not turn them. See Turn or Rebuke Undead for more information.

Even if a cleric is neutral, channeling positive energy is a good act and channeling 
negative energy is evil.

%%%
\subsubsection{Paladins and Undead}
%%%

Beginning at 4th level, paladins can turn undead as if they were clerics of three 
levels lower than they actually are.

%%%
\subsubsection{Turning Other Creatures}
%%%

Some clerics have the ability to turn creatures other than undead.

The turning check result is determined as normal.

%%%%%%%%%%%%%%%%%%%%%%%%%
\subsection{Two-Weapon Fighting}\index{Two-Weapon Fighting}
%%%%%%%%%%%%%%%%%%%%%%%%%

If you wield a second weapon in your off hand, you can get one extra attack per 
round with that weapon. You suffer a -6 penalty with your regular attack or attacks 
with your primary hand and a -10 penalty to the attack with your off hand when 
you fight this way. You can reduce these penalties in two ways:

\begin{itemize*}
\item If your off-hand weapon is light, the penalties are reduced by 2 each. (An unarmed strike is always considered light.)
\item The \linkfeat{Two-Weapon Fighting} feat lessens the primary hand penalty by 2, and the off-hand penalty by 6.
\end{itemize*}

Table: Two-Weapon Fighting Penalties summarizes the interaction of all these factors.

\begin{table}[htb]
\rowcolors{1}{white}{offyellow}\mcinherit
\caption{Two-Weapon Fighting Penalties}
\centering
\begin{tabular}{l c c}
\textbf{Circumstances} & \textbf{Primary Hand} & \textbf{Off Hand}\\
Normal penalties & -6 & -10\\
Off-hand weapon is light & -4 & -8\\
Two-Weapon Fighting feat & -4 & -4\\
\shortstack{Off-hand weapon is light and\\Two-Weapon Fighting feat} & -2 & -2\\
\end{tabular}
\end{table}

\textbf{Double Weapons}: You can use a double weapon to make an extra attack with 
the off-hand end of the weapon as if you were fighting with two weapons. The penalties 
apply as if the off-hand end of the weapon were a light weapon.

\textbf{Thrown Weapons:} The same rules apply when you throw a weapon from each 
hand. Treat a dart or shuriken as a light weapon when used in this manner, and 
treat a bolas, javelin, net, or sling as a one-handed weapon.

%%%%%%%%%%%%%%%%%%%%%%%%%%%%%%%%%%%%%%%%%%%%%%%%%%
\section{Special Initiative Actions}
%%%%%%%%%%%%%%%%%%%%%%%%%%%%%%%%%%%%%%%%%%%%%%%%%%

Here are ways to change when you act during combat by altering your place in the 
initiative order.

%%%%%%%%%%%%%%%%%%%%%%%%%
\subsection{Delay}\index{Delay Action}\index{Initiative!Delay}
%%%%%%%%%%%%%%%%%%%%%%%%%

By choosing to delay, you take no action and then act normally on whatever initiative 
count you decide to act. When you delay, you voluntarily reduce your own initiative 
result for the rest of the combat. When your new, lower initiative count comes 
up later in the same round, you can act normally. You can specify this new initiative 
result or just wait until some time later in the round and act then, thus fixing 
your new initiative count at that point.

You never get back the time you spend waiting to see what's going to happen. You 
can't, however, interrupt anyone else's action (as you can with a readied action).

\textbf{Initiative Consequences of Delaying:} Your initiative result becomes the 
count on which you took the delayed action. If you come to your next action and 
have not yet performed an action, you don't get to take a delayed action (though 
you can delay again).

If you take a delayed action in the next round, before your regular turn comes 
up, your initiative count rises to that new point in the order of battle, and you 
do not get your regular action that round.

%%%%%%%%%%%%%%%%%%%%%%%%%
\subsection{Ready}\index{Readied Action}\index{Initiative!Readied Action}
%%%%%%%%%%%%%%%%%%%%%%%%%

The ready action lets you prepare to take an action later, after your turn is over 
but before your next one has begun. Readying is a standard action. It does not 
provoke an attack of opportunity (though the action that you ready might do so).

\textbf{Readying an Action:} You can ready a standard action, a move action, or 
a free action. To do so, specify the action you will take and the conditions under 
which you will take it. Then, any time before your next action, you may take the 
readied action in response to that condition. The action occurs just before the 
action that triggers it. If the triggered action is part of another character's 
activities, you interrupt the other character. Assuming he is still capable of 
doing so, he continues his actions once you complete your readied action. Your 
initiative result changes. For the rest of the encounter, your initiative result 
is the count on which you took the readied action, and you act immediately ahead 
of the character whose action triggered your readied action.

You can take a 5-foot step as part of your readied action, but only if you don't 
otherwise move any distance during the round. 

\textbf{Initiative Consequences of Readying:} Your initiative result becomes the 
count on which you took the readied action. If you come to your next action and 
have not yet performed your readied action, you don't get to take the readied action 
(though you can ready the same action again). If you take your readied action in 
the next round, before your regular turn comes up, your initiative count rises 
to that new point in the order of battle, and you do not get your regular action 
that round.

\textbf{Distracting Spellcasters:} You can ready an attack against a spellcaster 
with the trigger "if she starts casting a spell." If you damage the spellcaster, 
she may lose the spell she was trying to cast (as determined by her Concentration 
check result).

\textbf{Readying to Counterspell:} You may ready a counterspell against a spellcaster 
(often with the trigger "if she starts casting a spell"). In this case, when 
the spellcaster starts a spell, you get a chance to identify it with a Spellcraft 
check (DC 15 + spell level). If you do, and if you can cast that same spell (are 
able to cast it and have it prepared, if you prepare spells), you can cast the 
spell as a counterspell and automatically ruin the other spellcaster's spell. Counterspelling 
works even if one spell is divine and the other arcane.

A spellcaster can use \linkspell{Dispel Magic} to counterspell another spellcaster, 
but it doesn't always work.

\textbf{Readying a Weapon against a Charge:} You can ready certain piercing weapons, 
setting them to receive charges. A readied weapon of this type deals double damage 
if you score a hit with it against a charging character.
