%%%%%%%%%%%%%%%%%%%%%%%%%
\skillentry{Jump}{(Str; Armor Check Penalty)}
%%%%%%%%%%%%%%%%%%%%%%%%%

\textbf{Check:} The DC and the distance you can cover vary according to the type 
of jump you are attempting (see below).

Your Jump check is modified by your speed. If your speed is 30 feet then no modifier 
based on speed applies to the check. If your speed is less than 30 feet, you take 
a -6 penalty for every 10 feet of speed less than 30 feet. If your speed is greater 
than 30 feet, you gain a +4 bonus for every 10 feet beyond 30 feet.

All Jump DCs given here assume that you get a running start, which requires that 
you move at least 20 feet in a straight line before attempting the jump. If you 
do not get a running start, the DC for the jump is doubled.

Distance moved by jumping is counted against your normal maximum movement in a 
round.

If you have ranks in Jump and you succeed on a Jump check, you land on your feet 
(when appropriate). If you attempt a Jump check untrained, you land prone unless 
you beat the DC by 5 or more.

\textit{Long Jump:} A long jump is a horizontal jump, made across a gap like a 
chasm or stream. At the midpoint of the jump, you attain a vertical height equal 
to one-quarter of the horizontal distance. The DC for the jump is equal to the 
distance jumped (in feet).

If your check succeeds, you land on your feet at the far end. If you fail the check 
by less than 5, you don't clear the distance, but you can make a DC 15 Reflex save 
to grab the far edge of the gap. You end your movement grasping the far edge. If 
that leaves you dangling over a chasm or gap, getting up requires a move action 
and a DC 15 Climb check.

\textit{High Jump:} A high jump is a vertical leap made to reach a ledge high above 
or to grasp something overhead. The DC is equal to 4 times the distance to be cleared.

If you jumped up to grab something, a successful check indicates that you reached 
the desired height. If you wish to pull yourself up, you can do so with a move 
action and a DC 15 Climb check. If you fail the Jump check, you do not reach the 
height, and you land on your feet in the same spot from which you jumped. As with 
a long jump, the DC is doubled if you do not get a running start of at least 20 
feet.


Obviously, the difficulty of reaching a given height varies according to the size 
of the character or creature. The maximum vertical reach (height the creature can 
reach without jumping) for an average creature of a given size is shown on the 
table below. (As a Medium creature, a typical human can reach 8 feet without jumping.)

Quadrupedal creatures don't have the same vertical reach as a bipedal creature; 
treat them as being one size category smaller.

\begin{table}[htb]
\rowcolors{1}{white}{offyellow}
\caption{Vertical Reach By Size}
\centering
\begin{tabular}{l c}
\textbf{Size} & \textbf{Vertical Reach} \\
Colossal & 128ft\\
Gargantuan & 64ft\\
Huge & 32ft\\
Large & 16ft\\
Medium & 8ft\\
Small & 4ft\\
Tiny & 2ft\\
Diminutive & 1ft\\
Fine & \sfrac{1}{2}ft\\
\end{tabular}
\end{table}

\textit{Hop Up:} You can jump up onto an object as tall as your waist, such as 
a table or small boulder, with a DC 10 Jump check. Doing so counts as 10 feet of 
movement, so if your speed is 30 feet, you could move 20 feet, then hop up onto 
a counter. You do not need to get a running start to hop up, so the DC is not doubled 
if you do not get a running start.

\textit{Jumping Down:} If you intentionally jump from a height, you take less damage 
than you would if you just fell. The DC to jump down from a height is 15. You do 
not have to get a running start to jump down, so the DC is not doubled if you do 
not get a running start.

If you succeed on the check, you take falling damage as if you had dropped 10 fewer 
feet than you actually did.

\textbf{Action:} None. A Jump check is included in your movement, so it is part 
of a move action. If you run out of movement mid-jump, your next action (either 
on this turn or, if necessary, on your next turn) must be a move action to complete 
the jump.

\textbf{Special:} Effects that increase your movement also increase your jumping 
distance, since your check is modified by your speed.

If you have the \linkfeat{Run} feat, you get a +4 bonus on Jump checks for any jumps made 
after a running start.

A \linkrace{Halfling} has a +2 racial bonus on Jump checks because halflings are agile and 
athletic.

If you have the \linkfeat{Acrobatic} feat, you get a +2 bonus on Jump checks.

\textbf{Synergy:} If you have 5 or more ranks in \linkskill{Tumble}, you get a +2 bonus on 
Jump checks.

If you have 5 or more ranks in Jump, you get a +2 bonus on Tumble checks.
