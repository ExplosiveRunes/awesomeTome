%%%%%%%%%%%%%%%%%%%%%%%%%
\skillentry{Sleight of Hand}{(Dex; Trained Only; Armor Check Penalty)}\label{skill:Sleight Of Hand}
%%%%%%%%%%%%%%%%%%%%%%%%%

\textbf{Check:} A DC 10 Sleight of Hand check lets you palm a coin-sized, unattended 
object. Performing a minor feat of legerdemain, such as making a coin disappear, 
also has a DC of 10 unless an observer is determined to note where the item went.

When you use this skill under close observation, your skill check is opposed by 
the observer's Spot check. The observer's success doesn't prevent you from performing 
the action, just from doing it unnoticed.

You can hide a small object (including a light weapon or an easily concealed ranged 
weapon, such as a dart, sling, or hand crossbow) on your body. Your Sleight of 
Hand check is opposed by the Spot check of anyone observing you or the Search check 
of anyone frisking you. In the latter case, the searcher gains a +4 bonus on the 
Search check, since it's generally easier to find such an object than to hide it. 
A dagger is easier to hide than most light weapons, and grants you a +2 bonus on 
your Sleight of Hand check to conceal it. An extraordinarily small object, such 
as a coin, shuriken, or ring, grants you a +4 bonus on your Sleight of Hand check 
to conceal it, and heavy or baggy clothing (such as a cloak) grants you a +2 bonus 
on the check.

Drawing a hidden weapon is a standard action and doesn't provoke an attack of opportunity.

If you try to take something from another creature, you must make a DC 20 Sleight 
of Hand check to obtain it. The opponent makes a Spot check to detect the attempt, 
opposed by the same Sleight of Hand check result you achieved when you tried to 
grab the item. An opponent who succeeds on this check notices the attempt, regardless 
of whether you got the item.

You can also use Sleight of Hand to entertain an audience as though you were using 
the Perform skill. In such a case, your "act" encompasses elements of legerdemain, 
juggling, and the like.

\begin{table}[htb]
\rowcolors{1}{white}{offyellow}
\caption{Sleight of Hand DCs}
\centering
\begin{tabular}{c l}
\textbf{Sleight of Hand DC} & \textbf{Task}\\
10 & Palm a coin-sized object, make a coin disappear\\
20 & Lift a small object from a person\\
\end{tabular}
\end{table}

\textbf{Action:} Any Sleight of Hand check normally is a standard action. However, 
you may perform a Sleight of Hand check as a free action by taking a -20 penalty 
on the check.

\textbf{Try Again:} Yes, but after an initial failure, a second Sleight of Hand 
attempt against the same target (or while you are being watched by the same observer 
who noticed your previous attempt) increases the DC for the task by 10.

\textbf{Special:} If you have the \linkfeat{Deft Hands} feat, you get a +2 bonus on Sleight 
of Hand checks.

\textbf{Synergy:} If you have 5 or more ranks in \linkskill{Bluff}, you get a +2 bonus on Sleight 
of Hand checks.

\textbf{Untrained:} An untrained Sleight of Hand check is simply a Dexterity check. 
Without actual training, you can't succeed on any Sleight of Hand check with a 
DC higher than 10, except for hiding an object on your body.
